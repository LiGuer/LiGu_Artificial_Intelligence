\documentclass{article} 
\usepackage{amsmath}
\usepackage[UTF8]{ctex}
\title{}\date{} \setlength{\parindent}{0pt} \linespread{1.25}
\usepackage{ntheorem}\usepackage{amssymb}
\usepackage{graphicx}
\usepackage{geometry} \geometry{a4paper,left=2cm,right=2cm,top=1cm,bottom=1.5cm}
\usepackage{paralist}
\let\itemize\compactitem
\let\enumerate\compactenum

\newcommand{\env}[2]{\begin{#1}#2\end{#1}}
\newcommand{\defi}[2]{\textbf{#1}, #2}
\newcommand{\proof}[1]{\textbf{证明} #1}
\newcommand{\bb}{\boldsymbol}
\newcommand{\P}{\mathbb P}
\newcommand{\d}{\mathrm d}
\newcommand{\l}{\left}
\newcommand{\r}{\right}

\begin{document}
\tableofcontents

\section{微积分}
	\section{极限}
		\section{数列极限}
			\defi{数列极限}{
				设$\{x_n\}$为一数列,如果存在常数a,对于任意给定的正数ε (不论它多么小),总存在正整数N,使得当n>N时,不等式$\left|x_{n}-a\right|<\varepsilon$都成立,那么就称常数a是数列$\{x_n\}$的极限, 或者称数列$\{x_n\}$收敛于a,记为$\lim _{n \rightarrow \infty} x_{n}=a$. 若不存在常数a, 则数列无极限, 数列\textbf{发散}.
				$$\lim _{n \rightarrow \infty} x_{n}=a \Leftrightarrow \forall \varepsilon>0, \exists \text { 正整数 } N \text {, 当 } n>N \text { 时, 有 }\left|x_{n}-a\right|<\varepsilon$$
			}

				\textbf{性质}: 
					\env{itemize}{
					\item 极限唯一性
					\item 收敛数列有界性
					\item 收敛数列保号性
					}

		\section{函数极限}
			\defi{函数极限}{
				设函数$f(x)$在点$x_{0}$的某一去心邻域内有定义. 如果存在常数$A$ , 对 于任意给定的正数$\varepsilon$(不论它多么小 ), 总存在正数$\delta$ , 使得当$x$满足不等式$0<\left|x-x_{0}\right|<\delta$时, 对应的函数值$f(x)$都满足不等式$|f(x)-A|<\varepsilon$, 那么常数$A$就叫做函数$f(x)$当$x \rightarrow x_{0}$时的极限, 记作$\lim _{x \rightarrow x_{0}} f(x)=A \text { 或 } f(x) \rightarrow A\left(\text { 当 } x \rightarrow x_{0}\right) .$
			}

    \section{微分}

    	\section{多元函数微分}

    \section{积分}

    	\section{多元函数积分}
		
\section{微分方程}



    *                    导数  N阶
    *	[定义]:
            导数: df/dx = lim_(x->0)  [ f(x+Δx) - f(x-Δx) ] / (2 Δx)
            N阶偏导:
                  d^n f/dx ^n = lim_(x->0)  [ f^(n - 1)(x+Δx) - f^(n - 1)(x-Δx) ] / (2 Δx)
    *	[算法]: 中心差分公式
            f'(x) = ( f(x+Δx) -  f(x-Δx) ) / (2 Δx) + Err(f,Δx)
            ·截断误差: Err(f,Δx) = h² f^(3)(c) / 6 = O(h²)
            ·精度: O(h²)
            f'(x) = ( -f(x+2Δx) + 8·f(x+Δx) - 8·f(x-Δx) + f(x-2Δx) ) / (12 Δx) + Err(f,Δx)
            ·截断误差: Err(f,Δx) = h^4 f^(5)(c) / 6 = O(h^4)
            ·精度: O(h^4)


*                    偏导数
*	[定义]:
		偏导: ∂f/∂x_i = [ f(..,xi+Δxi,..) -  f(..,xi-Δxi,..) ] / (2 Δxi)
		二阶偏导:
			∂²f/∂x_i² = [ f'(..,xi+Δxi,..) -  f'(..,xi-Δ,..) ] / Δxi
					  = f(..,xi+Δxi) - 2·f'(..,x) + f'(..,xi-Δxi) / Δxi²



*                    Hamilton 算子、Laplace 算子
*	[定义]: 
		Hamilton: ▽  ≡ ∂/∂x \vec x + ∂/∂y \vec y + ∂/∂z \vec z + ... (看成一个矢量)
		Laplace:  ▽² ≡ ∂²/∂x² + ∂²/∂y² + ∂²/∂z² + ...
		▽² = ▽·▽
*	[算法]: 有限差分法
		[u(x+1,...) - 2·u(x,...) + u(x-1,...)] / Δx² + ...
*	[注]: Hamilton算子程序,见<梯度、散度、旋度>                      
       


*                    梯度、散度、旋度
*	[定义]:
		梯度: ▽f		: 矢量, 函数在该点处变化率最大的方向.
		散度: ▽·\vec f: 标量, 矢量场在该点发散的程度, 表征场的有源性(>0源,<0汇,=0无源)
		旋度: ▽×\vec f: 矢量, 矢量场在该点旋转的程度, 
							方向是旋转度最大的环量的旋转轴, 旋转的方向满足右手定则,
							大小是绕该旋转轴旋转的环量与旋转路径围成的面元面积之比.
*	[公式]: (直角坐标系)
		▽f        = ∂f/∂x \vec x + ∂f/∂y \vec y + ∂f/∂z \vec z + ...
		▽·\vec f =  ∂fx/∂x + ∂fy/∂y + ∂fz/∂z + ...
		▽×\vec f = (∂fz/∂y - ∂fy/∂z) \vec x
				   + (∂fx/∂z - ∂fz/∂x) \vec y
				   + (∂fy/∂x - ∂fx/∂y) \vec z


级数展开


*                    Taylor 展开
*	[定义]: 
		f(x) = f(x0)/0! + f'(x0)/1!·(x-x0) + ... + f^(n)(x0)/n!·(x-x0)
*	[Example]:
		Calculus::TaylorFormula(0, [](double x) { return sin(x); }, Coeff, 10);



*					Fast Fourier Transform 快速Fourier变换
*	[定义]: 离散Fourier变换的高效算法
*	[公式]:
		离散Fourier变换: X[k] = Σ_(n=0)^(N-1)  e^(-j2πnk/N)·x[n]
*	[时间复杂度]: O(N·logN)
*	[算法]:
		N-Point Model:
				___________
		x[0] —| N/2-Point |—> E[0] —×××> X[0]  (E[0],+O[0])
		x[2] —|   DFT	   |—> E[1] —×××> X[1]  (E[1],+O[1])
		x[4] —|		   |—> E[2] —×××> X[2]  (E[2],+O[2])
		x[6] —|		   |—> E[3] —×××> X[3]  (E[3],+O[3])
                ___________
		x[1] —| N/2-Point |—> O[0] —×××> X[4]  (E[0],-O[0])
		x[3] —|   DFT	   |—> O[1] —×××> X[5]  (E[1],-O[1])
		x[5] —|		   |—> O[2] —×××> X[6]  (E[2],-O[2])
		x[7] —|		   |—> O[3] —×××> X[7]  (E[3],-O[3])

		(E[k],±O[k]) = E[k] ± W_N^K·O[k]
*	[Reference]:
		Thanks for https://www.math.wustl.edu/~victor/mfmm/fourier/fft.c


*                    曲率
*	[定义]: 单位弧段弯曲的角度. 也即等效曲率圆的半径的倒数.
*	[公式]: K = |Δα/Δs| = 1 / R = |y''| / (1 + y'²)^(3/2)


    *                    积分
    *	[定义]:
    *	[算法]: NewtonCotes 公式
            ∫_a^b f(x) = (b - a) Σ_(k=0)^n  C_k^(n) f(xi)
            C_k^(n) = (-1)^(n-k) / (n·k!(n-k)!) ∫_0^n Π_(k≠j) (t-j)dt 
            n = 1: C = {1/2, 1/2}
            n = 2: C = {1/6, 4/6, 1/6}
            n = 4: C = {7/90, 32/90, 12/90, 32/90, 7/90}
            * NewtonCotes 公式在 n > 8 时不具有稳定性
            复合求积法: 将积分区间分成若干个子区间, 再在每个子区间使用低阶求积公式.


*                    重积分
*	[定义]: ∫∫∫ f(r) dr³


*                    解常微分方程组: Runge Kutta 方法
*	[公式]:           ->   ->       ->      ->
		常微分方程组: y' = f(x, y)	y(x0) = y0
		迭代求解 \vec y(x) 的一点/一区间的数值解.
		y(x + dx) = y(x) + dx/6·(k1 + 2·k2 + 2·k3 + k4)
		k1 = f(xn , yn)						//区间开始斜率
		k2 = f(xn + dx/2, yn + dx/2·k1)	//区间中点斜率,通过欧拉法采用k1决定y在xn+dx/2值
		k3 = f(xn + dx/2, yn + dx/2·k2)	//区间中点斜率,采用k2决定y值
		k4 = f(xn + dx	, yn + dx  ·k3)	//区间终点斜率
*	[目的]: 解常微分方程组
		[ y1'(x) = f1(y1 , ... , yn , x)          ->   ->
		| y2'(x) = f2(y1 , ... , yn , x)    =>    y' = f(x , y)
		| ...
		[ yn'(x) = fn(y1 , ... , yn , x)
*	[性质]:
		* RK4法是四阶方法,每步误差是h⁵阶,总积累误差为h⁴阶


*                    Poisson's方程
*	[定义]: Δφ = f
			▽²φ = f  (Euclidean空间)
		三维直角坐标系中 (∂²/∂x² + ∂²/∂y² + ∂²/∂z²) φ(x,y,z) = f(x,y,z)
		当f ≡ 0, 得到 Laplace's方程
	[解法]:  Green's函数  φ(r) = - ∫∫∫ f(rt) / 4π|r-rt| d³rt    ,r rt为矢量


*                    波动方程
*	[定义]: a ▽²u = ∂²u/∂t²
*	[算法]: 有限差分法
		u(t+1,...) = 2·u(t,...) - u(t-1,...)
				+ Δt²·a{[u(x+1,...) - 2·u(x,...) + u(x-1,...)]/Δx² + ...}
*                    扩散方程
*	[定义]: a ▽²u = ∂u/∂t
	[算法]: 有限差分法
		u(t+1,...) = u(t,r)
				+ Δt·a{[u(x+1,...) - 2·u(x,...) + u(x-1,...)]/Δx² + ...}



*					Lagrange插值
	[原理]:
		f(x) = Σ_(i=1)^n  y_i · f_i(x)
		f_i(x) = Π_(j=1,i≠j)^n  (x - x_j) / (x_i - x_j)
		第N点y = 基函数1×第1点y + 基函数2×第2点y + 基函数3×第3点y
		基函数状态2 = (输入X-第1点x)(输入X-第3点x) / (第2点x-第1点x)(第2点x-第3点x)


*					样条插值
*	[算法]: 过求解三弯矩方程组得出曲线函数组的过程





\end{document}
