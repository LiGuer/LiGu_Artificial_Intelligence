\documentclass{article} 
\usepackage{amsmath}
\usepackage[UTF8]{ctex}
\title{}\date{} \setlength{\parindent}{0pt} \linespread{1.25}
\usepackage{ntheorem}\usepackage{amssymb}
\usepackage{graphicx}
\usepackage{geometry} \geometry{a4paper,left=2cm,right=2cm,top=1cm,bottom=1.5cm}

\begin{document}
\tableofcontents

\section{数论}

组合 / 排列 
*	[定义]: 组合: 从n个不同元素中任取m个,不管其顺序合成一组的,所有组合的个数.记C_n^m.
			排列: 从n个不同元素中任取m个,按一定顺序合成一组的,所有组合的个数.记A_n^m.
*	[公式]: C_n^m = n! / ((n - m)!·m!)
			A_n^m = n! / (n - m)!
			杨辉恒等式: C_n^m = C_(n-1)^(m-1) + C_(n-1)^m
---------------------------------------------------------------------------------
*	[问题]: 求解C_n^m mod p
*	[公式]: Lucas定理: C_n^m mod p = C_(n/p)^(m/p)·C_(n mod p)^(m mod p) mod p
			将n, m 表示为p进制形式, ni, mi是该进制模式下的第i位.
			则 C_n^m = Π_(i=0)^k  C_ni^mi mod p

Fibonacci数列
[定义]: F[n] = F[n-1] + F[n-2]		(n≥2, F[0]=0, F[1]=1)
[算法]: 递推式

最大公约数 / 最小公倍数
[算法]: 辗转相除法 GCD(a,b) == GCD(b, a mod b)
[算法]: 最小公倍数 = a·b / GCD(a,b)
[注]: 先除后乘,减少中间数位数

[拓展Euclid算法]
	[算法]: 辗转相除法的拓展
			除计算a、b最大公约数, 还能找到满足a x + b y = gcd(a,b)的(x,y)的一组解(其中一个很可能是负数)
	[线性方程定理]: 
		非零正整数a b, 总存在x y满足 a x + b y = gcd(a ,b)
		若a b互质, 则 a x + b y = 1

[						素数判断 / 素数筛
*	[定义]: 素数: 只被1和本身整除的数. (1不是素数)
	[算法]: 素数判断: 遍历测试所有整数直至sqrt(a), 
			因为a不会再被更大数整除, 能和更大数相乘等于a的小整数已被测试过。
			时间复杂度 O(sqrt(n))
	[算法]: 素数筛: 从小到大将素数的倍数筛掉, 时间复杂度 O(n log(n))


[幂次模]
[问题]: 求 b = a^k (mod m)
[算法]: 逐次平方法
[步骤]:
    [1] 将 k 二进制展开  k = u0·2⁰ + u1·2¹ + u2·2² + ... + ur·2^r
            (计算机里, k内存天然是二进制)
    [2] 逐次平方制作模m的a幂次表, i∈[0,r]
            a^(2^0) = a ≡ A0 (mod m)
            a^(2^i) ≡ (a^2^(i-1))² ≡ A²(i-1) ≡ Ai (mod m)
    [3] 乘积  A0^u0·A1^u1·...·Ar^ur  (mod m)
[证明]: a^k = a^(u0·2⁰ + u1·2¹ + u2·2² + ... + ur·2^r)

[RSA 密码]
	[特征]: 非对称密码. 公钥加密, 私钥解密.
			大数质因数分解的时间复杂度极高.
	[步骤]:
		(1) 选2个大素数 p,q 
		(2) n = p q
			φ(n) = (p - 1)(q - 1)
		(3) 选公钥 a	(>1, 小于且互质于φ(n), ≠p、q)
		(4) 计算私钥 b	(a b = 1 mod φ(n))
			利用拓展Euclid算法
		(5) 公钥(n, a)		密文 = 明文^a (mod n)
			私钥(n, b)		明文 = 密文^b (mod n)
			加/解密, 利用幂次模算法
	[原理]:
		(1) Euler函数φ(n): 小于且互质于n的数的数目
			若n为素数, 则φ(n) = (n - 1)
		(2) a b = 1 mod φ(n)  =>  a b + c φ(n) = 1
			拓展Euclid算法, 可计算 a x + b y 的(x,y)的一组解
		(3) Euler公式: a n 互质 => a ^ φ(n) ≡ 1 (mod n)
			=> 明 ^ (kφ(n) + 1) = 明 ^ (ab) ≡ 1 (mod n)
			=> 若 密 = 明^a (mod n), 则 密^b = 明^(ab) = 明 (mod n)
	[例]:
		私钥: RSAPrivateKey(p, q, a)
		加密: PowMod(message, a, n);
		解密: PowMod(message, b, n);

\end{document}
