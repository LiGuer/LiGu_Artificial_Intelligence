\iffalse
Copyright 2020 LiGuer. All Rights Reserved.
Licensed under the Apache License, Version 2.0 (the "License");
you may not use this file except in compliance with the License.
You may obtain a copy of the License at
	http://www.apache.org/licenses/LICENSE-2.0
Unless required by applicable law or agreed to in writing, software
distributed under the License is distributed on an "AS IS" BASIS,
WITHOUT WARRANTIES OR CONDITIONS OF ANY KIND, either express or implied.
See the License for the specific language governing permissions and
limitations under the License.
==============================================================================\fi
\documentclass{article} 
\usepackage{latexsym}
\usepackage{amssymb}
\usepackage[UTF8]{ctex}
\usepackage{geometry}\geometry{left=2cm, right=2cm, top=2cm, bottom=2cm}
\setlength{\parindent}{0pt}
\title{Physics 物理}\author{}\date{}
\usepackage{paralist}
\let\itemize\compactitem
\let\enumerate\compactenum

\newcommand{\env}[2]{\begin{#1}#2\end{#1}}
\newcommand{\defi}[2]{\textbf{#1}, #2}
\newcommand{\proof}[1]{\textbf{证明} #1}

\begin{document}
\maketitle
\tableofcontents


\section{Mechanics力学}
    \subsection{力学之描述——运动方程}
        (对于s个自由度的系统,)
        
        \defi{广义坐标}{完全刻画其位置的任意s个变量 $q_{1},q_{2},\cdots,q_{s}$}
        
        \defi{广义速度}{广义坐标对时间的一阶导$\dot q_{1},\dot q_{2},\cdots,\dot q_{s}$}
        
        · 经验表明, 同时给定广义坐标、广义速度, 就可确定系统状态, 原则上可预测以后动作.(其中,加速度$\ddot q$, 可由$\dot q,q$唯一确定.)
        
        \defi{运动方程}{加速度与坐标、速度的关系式. (二阶微分方程,原则上积分得q(t)可确定系统运动轨迹.)}
        
        \defi{Lagrange函数}{每一个力学系统可以用一个确定函数L()表征.}
        
        $$L(q_{1},q_{2},\cdots,q_{s},\dot q_{1},\dot q_{2},\cdots,\dot q_{s},t)$$
        
        \defi{最小作用量原理}{系统在两个时刻的, 位置之间的运动, 使得Lagrange函数积分(作用量S)取最小值.
            $$S = \int_{t1}^{t2} L(q,\dot q,t)dt$$
            $$\delta S = \delta \int_{t1}^{t2} L(q,\dot q,t)dt = \int_{t1}^{t2} (\frac{\partial L}{\partial q}\delta q + \frac{\partial L}{\partial \dot q}\delta \dot q) dt = \frac{\partial L}{\partial \dot q} \delta q \arrowvert_{t1}^{t2} + \int_{t1}^{t2} (\frac{\partial L}{\partial q} - \frac{d}{dt} \frac{\partial L}{\partial \dot q}) \delta q dt = 0$$
        }
        
        \defi{Lagrange方程}{运动微分方程.
            $$\Rightarrow \frac{d}{dt}\frac{\partial L}{\partial \dot q_i} - \frac{\partial L}{\partial q_i} = 0\quad(i=1,\dots,s)$$
        }


    \subsection{力学之框架——参考系}
        · 研究力学现象必须选择参考系.
        
        \textbf{惯性参考系}:空间相对它均匀且各向同性,时间相对它均匀.(特别的,某时刻静止的自由物体将永远保持静止.)\\
        $\Rightarrow$ Lagrange函数不显含$\vec r,t$,不依赖$\vec v$的矢量方向,即
            $$L = L(v^2)$$
        
        $\Rightarrow$ Lagrange方程有$\frac{d}{dt}\frac{\partial L}{\partial \vec v} = -\frac{\partial L}{\partial \vec r} = 0 \Rightarrow \frac{\partial L}{\partial \vec v}=const$
        
        \defi{惯性定律}{在惯性参考系中,质点任何自由运动的速度的大小和方向都不改变.}
            $$\Rightarrow \vec v = const$$


    \subsection{力学之框架——参考系间相对性}
        \defi{Galilean相对性原理}{存在无穷个相互作匀速直线运动的惯性参考系,这些惯性系之间时空性质相同,所有力学规律等价.}
        
        (今后不特别声明,默认惯性参考系.)
        
        \defi{Galilean变换}{两个不同参考系K、K'之间的的坐标变换(K'相对K以速度$\vec V$运动).}
        
        \defi{绝对时间假设}{我们认为两个参考系的时间相同.}
            \begin{displaymath}
                \left\{ \begin{array}{ll}
                \vec r = \vec r' + \vec V t\\
                t = t'
                \end{array} \right.
            \end{displaymath}
        
        \defi{Galilean相对性原理}{力学系统在Galilean变换下具有不变性.}
    
    
    \subsection{力学之对象——质点、质点系}
        \textbf{质点}:有质量,无体积形状的,理想的点.
        
        \textbf{质点的Lagrange函数}:
        
        \textbf{质量 m}:
            $$L = \frac{m}{2} v^2$$
        
        \textbf{质点系}:2个及以上相互作用的质点,组成的力学系统.
        
        \textbf{质点系的Lagrange函数}:
        
        \textbf{动能 T}: \quad \textbf{势能 U}:描述质点之间相互作用,而增加的关于坐标的函数(由相互作用性质决定)(限于经典力学).
            $$L = \sum \frac{m_i v_i^2}{2} - U(\vec r_1,\vec r_2,\cdots ) = T(v_1^2,v_2^2,\cdots) - U(\vec r_1,\vec r_2,\cdots )$$


\subsection{力学之不变性——守恒定律}
    \subsubsection{能量守恒——时间均匀性}
        时间均匀性: 封闭系统Lagrange函数不显含时间.\quad $\Rightarrow L(\vec q,\dot \vec q)$
            $$\frac{dL}{dt} = \sum \frac{\partial L}{\partial q_i} \dot q_i + \sum \frac{\partial L}{\partial \dot q_i} \ddot q_i = \sum \frac{d}{dt}(\frac{\partial L}{\partial \dot q_i}\dot q_i) 
            \Rightarrow \frac{d}{dt}(\sum \dot q_i \frac{\partial L}{\partial \dot q_i} - L) = 0$$
            
        \textbf{能量E}: 
        $$\Rightarrow E = \sum \dot q_i \frac{\partial L}{\partial \dot q_i} - L  = T(q,\dot q) + U(q) = const.$$
    · 封闭系统、定常外场系统,即当Lagrange函数不显含时间时,能量守恒成立.


    \subsubsection{动量守恒——空间均匀性、力}
        空间均匀性: 空间平移不变性.
            $$\delta L = \sum \frac{\partial L}{\partial \vec r_i}\cdot \delta \vec r_i = \vec \epsilon \cdot \sum \frac{\partial L}{\partial \vec r_i} = 0\quad(\forall \vec \epsilon)
            \Rightarrow \sum \frac{\partial L}{\partial \vec r_i} = \sum \frac{d}{dt} \frac{\partial L}{\partial \vec v_i} = \frac{d}{dt} \sum \frac{\partial L}{\partial \vec v_i} = 0$$
    
        \textbf{动量$\vec P$: }
            $$\Rightarrow \vec P = \sum \frac{\partial L}{\partial \vec v_i} = \sum m_i \vec v_i = const.$$
    
        \textbf{力$\vec F$}:动量对时间的一阶导. \quad 封闭系统合外力为零.
            $$\Rightarrow \sum \frac{\partial L}{\partial \vec r_i} = \sum \vec F_i = 0 \quad , \quad \vec F = \frac{\partial L}{\partial \vec r} = \dot{ \vec P }$$
    
        \textbf{广义动量、广义力}:
            $$P_i = \frac{\partial L}{\partial \dot q_i},\quad F_i = \frac{\partial L}{\partial \dot q_i} = \dot P_i$$


    \subsubsection{角动量守恒——空间各向同性}
        空间各向同性: 空间旋转不变性.
            \begin{displaymath}
            \begin{array}{ll}
            \delta L = \sum (\frac{\partial L}{\partial \vec r_i} \cdot \delta \vec r_i + \frac{\partial L}{\partial \vec v_i} \cdot \delta \vec v_i) = \sum [\dot{\vec P_i} \cdot (\delta \vec \varphi \times \vec r_i) + \vec P_i \cdot ( \delta \vec \varphi \times \vec v_i )]
             = \delta \vec \varphi \cdot \sum (\vec r_i \tims \dot{\vec P_i} + \vec v_i \tims \vec P_i) \\
             \quad\  = \delta \vec \varphi \cdot \frac{d}{dt} \sum \vec r_i \times \vec P_i = 0\quad(\forall \vec \varphi) \Rightarrow \frac{d}{dt}\sum \vec r_i \times \vec P_i = 0
            \end{array}
            \end{displaymath}
        
        \textbf{角动量$\vec M$: }
            $$\Rightarrow \vec M = \sum \vec r_i \times \vec P_i = const.$$


    \subsubsection{$E,\vec P,\vec M$参考系间变换}
        不同惯性参考系中(K'相对K以速度$\vec V$运动)(对于$\vec M$且K'相对K坐标原点相差$\vec R$)\\
            \begin{displaymath}
                \left\{ \begin{array}{ll}
                E = \frac{1}{2} \sum m_i(\vec v_i + \vec V)^2 + U = \frac{m_c V^2}{2} + \vec V \cdot \sum m_i \vec v'_i + \frac{1}{2} \sum m_i v'_i^2 + U = E' + \vec V \cdot \vec P' + \frac{m_c V^2}{2}\\
                \vec P = \sum m_i (\vec v'_i + \vec V) = \sum m_i \vec v'_i + \vec V \sum m_i = \vec P' + \vec V \sum m_i\\
                \vec M = \sum m_i (\vec r'_i + \vec R) \times (\vec v'_i + \vec V) = \sum m_i( \vec r'_i \times \vec v'_i +  \vec R \times \vec v'_i + \vec r'_i \times \vec V + \vec R \times \vec V)
             = \vec M' + \vec R \times \vec P'_c + m_c \vec r'_c \times \vec V + m_c \vec R \times \vec V\\
                \end{array} \right.
            \end{displaymath}


\subsection{力学之相似性}
    * Lagrange函数乘任意常数,不会改变运动方程.
        $$\vec r_i \to \alpha \vec r_i, t \to \beta t 
        \Rightarrow \vec v_i = \frac{d\vec r_i}{dt} \to \frac{\alpha}{\beta}\vec v_i,T \to \frac{\alpha^2}{\beta^2}T,U \to  \alpha^k U$$
    
    若$\frac{\alpha^2}{\beta^2} = \alpha ^ k $,即$\beta = \alpha^{1-k/2}$,则Lagrange函数乘const.,运动方程不变.前后运动轨迹相似,只是尺寸不同.
    
    且各力学量之比,满足:\quad(l:轨迹线度)
        $$\frac{t'}{t} = (\frac{l'}{l})^{1-k/2},\frac{v'}{v} = (\frac{l'}{l})^{k/2},\frac{E'}{E} = (\frac{l'}{l})^k,\frac{M'}{M} = (\frac{l'}{l})^{1+k/2}$$
        
    · 例1: 均匀力场,势能与坐标成线性,$\Rightarrow k=1,\Rightarrow \frac{t'}{t} = \sqrt{\frac{l'}{l}}$,重力场自由落体,下落时间平方与初始高度成正比.
    
    · 例2: Kepler's第三定律,Newton引力、Coulomb力,势能与两点间距离成反比,$\Rightarrow k=-1,\Rightarrow \frac{t'}{t} = (\frac{l'}{l})^{3/2}$,轨道运动周期的平方与轨道尺寸的立方成正比.


    \subsection{力学之对象——质心}
        \textbf{质心系}:\exists 速度$\vec V$使得系统相对K'静止($\vec P' = 0$),且K'原点为系统质量中心,K'即质心系:
            $$\vec V = \frac{\vec P}{\sum m_i} = \frac{\sum m_i \vec v_i}{\sum m_i}$$
            
        \textbf{质心}: 质点系的质量中心,并将质点系$\Leftrightarrow$位于质心的质点.
            \begin{displaymath}
                \left\{ \begin{array}{ll}
                m_c = \sum m_i\\
                \vec r_c = \frac{\sum m_i \vec r_i}{\sum m_i}\\
                \vec v_c = \frac{\sum m_i \vec v_i}{\sum m_i}
                \end{array} \right.
            \end{displaymath}
            
        \textbf{质心力学量}:
            内能$E_{int}$: 整体静止的(质心系内)力学系统的能量,包括系统内相对运动动能 + 相互作用势能.\\
            \begin{displaymath}
                \left\{ \begin{array}{ll}
                \vec P_c = m_c \vec v_c\\
                E_c = \frac{m_c V^2}{2} + \vec V \cdot \vec P' |_{\vec P'=0} + E_{int} = E_{int} + \frac{m_c v_c^2}{2}\\
                \vec M_c = \vec M' + \vec R \times \vec P'_c + m_c \vec r'_c \times \vec V + m'_c \vec R \times \vec V|_{\vec P'_c =\vec r'_c = 0, \vec R =\vec r_c, \vec V =\vec v_c} = \vec M_{int} + \vec r_c \times \vec P_c 
                \end{array} \right.
            \end{displaymath}
            
        \textbf{质心组合关系}: 将质点系分成若干小系,各小系质心构成新的质点系之质心即为原质点系的质心.


    \subsection{情景: 一维运动}
        一维运动,定常外部条件下,
        
        \textbf{[0] Lagrange函数}:
            $$L = \frac{1}{2} = a(q) \dot q^2 - U(x)\quad (CartesianCoord)\Rightarrow L = \frac{m \dot x^2}{2} - U(x)$$
        
        \textbf{[1] 运动不变性 —— 能量守恒}: 定常外场下能量守恒,有
            $$E = \frac{\partial L}{\partial \dot q}\dot q - L = \frac{1}{2}m\cdot 2 \dot x \cdot \dot x - (\frac{m \dot x^2}{2} - U(x)) = \frac{m \dot x^2}{2} + U(x) $$
        
        \textbf{[2] 运动方程}:
            $$\dot x = \frac{dx}{dt} = \sqrt{\frac{2}{m}[E - U(x)]}\quad \Rightarrow \quad t = \sqrt{\frac{m}{2}} \int \frac{dx}{\sqrt{E - U(x)}} + const$$
            $\because$动能恒正,故运动只能发生在$U(x) \leqslant E$的空间区域.


    \subsection{情景: 二体问题}
        \textbf{二体问题}: 两个相互作用的质点,组成的系统的运动.(相互作用的两个质点的势能仅依赖于它们之间的距离.)\\
        二体问题Lagrange函数:
            $$L = \frac{m \vec \dot r_1^2}{2} + \frac{m \vec \dot r_1^2}{2} - U(|\vec r_1 - \vec r_2|)$$
        
        \textbf{核心思想}: 将问题分解为\textbf{质心运动}和\textbf{相对质心运动},以质心为原点:
            $$m_1\vec r_1 + m_2 \vec r_2 =0,\quad \vec r_{12} = \vec r_1 - \vec r_2, \quad \Rightarrow \quad \vec r_1 = \frac{m_2}{m_1 + m_2}\vec r_{12}, \quad \vec r_2 = - \frac{m_1}{m_1 + m_2}\vec r_{12} $$
            $$\Rightarrow L = \frac{m_{12} \vec \dot r_{12}^2}{2} - U(|\vec r_{12}|), \quad m_{12} = \frac{m_1 m_2}{m_1 + m_2}$$
            $\Rightarrow$"二体问题"等效为一个质量$m_{12}$的质点,在外场$U(\vec r_{12})$下的运动, 而分运动$\vec r_1, \vec r_2$, 可由$\vec r_{12}$分别解出.
        
        
    \subsection{情景: 有心力场}
        \textbf{有心力场}: 质点势能只与质点到某一固定点的距离有关的外场.
        
        \textbf{有心力}: 始终指向or背离与质点到某一固定点的方向,且大小只依赖r的力.
            $$\vec F = -\frac{\partial U(r)}{\partial \vec r} = -\frac{d U(r)}{d r} \hat r$$
    
        \textbf{[1] 运动不变性 —— 角动量守恒}:中心对称外场下(即势能仅依赖到空间某特定点(中心)距离),系统角动量在任意过中心的轴上投影都守恒.
            $$\Rightarrow \vec M = \vec r \times \vec P = const.$$
            $\Rightarrow$ 质点运动在垂直于$\vec M$的平面内.\quad $\Rightarrow$ 有心力场,\textbf{[0] Lagrange函数}:
            $$L = \frac{1}{2}m v^2 - U(r) = \frac{m}{2} (\dot r^2 + r^2 \dot \varphi ^2) - U(r)$$
            $\varphi$的广义动量:
            $$P_\varphi = \frac{\partial L}{\partial \dot \varphi} = m r^2 \dot \varphi \quad , \quad \frac{d P_\varphi}{d t} = \frac{d}{d t}\frac{\partial L}{\partial \dot \varphi} = \frac{\partial L}{\partial \varphi} \frac{d \frac{d \varphi}{d \varphi / d t}}{d t} = \frac{\partial L}{\partial \varphi} = 0 \quad , \quad |\vec M| = M_z = \sum \frac{\partial L}{\partial  \dot \varphi} = P_\varphi$$
            $$\Rightarrow |\vec M| =| \vec r \times \vec P |= P_\varphi = m r^2 \dot \varphi = const.$$
    
        \textbf{Kepler's第二定律}:质点矢径在相同时间内扫过的面积相等.
            $$\Rightarrow M = m r^2 \dot \varphi = 2 m \dot S_{sector} = const.\quad \Rightarrow \dot S_{sector} = \frac{1}{2} r \cdot r d\varphi = const.$$
    
        \textbf{[2]运动不变性——能量守恒}: 定常外场下能量守恒,有
            $$E = \frac{\partial L}{\partial \dot q}\dot q - L = \frac{m}{2} (\dot r^2 + r^2 \dot \varphi ^2) + U(r) = \frac{m \dot r^2}{2} + \frac{M^2}{2mr^2} + U(r)$$
    
        \textbf{[3]运动方程}:
            $$\Rightarrow t = \int \frac{d r}{\sqrt{\frac{2}{m}[E-U(r)] - \frac{M^2}{m^2 r^2}}} + const. \quad \varphi = \int \frac{M}{m r^2} d t  + const.= \int \frac{M/r^2\ dr}{\sqrt{2m [E-U(r)] - M^2/r^2}} + const.$$
    
        \textbf{[4]结果讨论}: 有心力场径向运动,和一维运动的联系.
    
        \textbf{等效势能}
            $$U_{eff} = U(r) + \frac{M^2}{2mr^2}$$
    
        \textbf{离心势能}
            $$U_{centrifuge} = \frac{M^2}{2mr^2}$$
            运动封闭条件:$\Delta \varphi$等于$2\pi$有理数倍.\quad($U(r) \propto \frac{1}{r}\ ,\ r^2$,则运动始终封闭.)
            $$\Delta \varphi = \int_{r_{min}} ^{r_{max}} \frac{M/r^2\ dr}{\sqrt{2m [E-U(r)] - M^2/r^2}} + const.$$
    

        \subsubsection{1/r有心力场}
        势能$U(r) \propto \frac{1}{r}$.\quad eg.引力场,库仑电场.
            $$U = - \alpha / r$$
            
        \textbf{[1]运动方程}: 焦点位于原点的圆锥曲线方程.\quad 偏心率e.
            $$\varphi = \int \frac{M/r^2\ dr}{\sqrt{2m (E+\frac{\alpha}{r}) - \frac{M^2}{r^2}}} + C. = \int \frac{-dk}{\sqrt{2mE + \frac{2m\alpha}{M}k - k^2}}|_{k = \frac{M}{r}} + C. = arccos\frac{M/r - m\alpha /M}{\sqrt{2mE + m^2 \alpha ^2 /M^2}} + C.$$
            $$\Rightarrow p/r = 1 + e \ cos \varphi \quad , \quad p = \frac{M^2}{m\alpha} , e = \sqrt{1 + \frac{2 E M^2}{m \alpha^2}}$$
            
        \textbf{[2]结果讨论}: \\
        
        \textbf{[2.1] $\alpha > 0\ and\ e<1, E<0$时,轨道为椭圆}\quad,半长轴a: \quad , 半短轴b: \quad , 周期T:
        $$a = \frac{p}{1-e^2} = \frac{\alpha}{2|E|} \quad , \quad b = \frac{p}{\sqrt{1-e^2}} = \frac{M}{\sqrt{2m|E|}} \quad , \quad T = \frac{2mS_{ellipse}}{M} = \frac{2\pi m \frac{\alpha}{2|E|} \frac{M}{\sqrt{2m|E|}}}{M} = \pi \alpha \sqrt{\frac{m}{2|E|^3}} $$
        
        \textbf{[2.2] $\alpha > 0\ and\ e=1, E=0$时,轨道为抛物线}
        
        \textbf{[2.3] $\alpha > 0\ and\ e>1, E>0$时,轨道为双曲线}
        
        \textbf{[2.4] $\alpha < 0$斥力场时,轨道为双曲线}



    \subsection{情景: 小振动}
        \subsubsection{一维小振动}
            \textbf{一维小振动}:
                设系统在势场$q_0$处平衡,即$F = -\frac{dU(q)}{dq}|_{q=q_0} = 0$,当平衡处发生微小偏移至$q(q\to q_0)$,Taylor展开,
                $$U(q) - U(q_0) = [U(q_0) - U(q_0)] + [\frac{d U(q)}{dq}|_{q=q_0} (q - q_0)]+ [\frac{d^2 U(q)}{dq^2}|_{q=q_0} \frac{(q - q_0)^2}{2}] + o((q - q_0)^2) \approx \frac{d^2 U(q)}{dq^2}|_{q=q_0} \frac{(q - q_0)^2}{2}$$
            
            等效势能:
            $$U(x) = - \frac{k x^2}{2} \quad , \quad x = q-q_0 \quad ,\quad k = \frac{d^2 U(q)}{dq^2}|_{q=q_0}$$
            
            \textbf{[0]Lagrange函数}:
            $$L = T - U = \frac{m \dot x^2}{2} - \frac{k x^2}{2}$$
            
            \textbf{[1]运动方程}: 系统在平衡位置附近作正弦振动.
            $$\frac{d}{dt} \frac{\partial L}{\partial \dot x} = \frac{\partial L}{\partial x} = \frac{d m\dot x}{dt} = -kx \quad \Rightarrow m \ddot x = -kx \quad \Rightarrow x = A cos(\omega t + \alpha) \quad , \omega = \sqrt{k/m}$$
            
            \textbf{[2]运动不变性——能量守恒}:
            $$E = \frac{\partial L}{\partial \dot q}\dot q - L = \frac{m \dot x^2}{2} + \frac{k x^2}{2} = \frac{1}{2} m \omega^2 A^2$$


        \subsubsection{一维强迫小振动、共振}
            \textbf{强迫振动}:外力下振动系统发生的振动. 对于微小偏移,强迫力势场,Taylor展开,有
            
            \textbf{强迫力$F(t)$}:
                $$U_e(x,t) \approx U_e(0,t) + \frac{\partial U_e}{x}|_{x=0}x = U_e(0,t) + xF(t)$$
            
            \textbf{[0]Lagrange函数}:
                $$L = T - (U_0 + U_e) = \frac{m \dot x^2}{2} - \frac{k x^2}{2} + xF(t)$$
            
            \textbf{[1]运动方程}:
                $$\ddot x + \omega^2 x = \frac{F(t)}{m}$$
            
            \textbf{[2]结果讨论}:
            
            \textbf{[2.1]} 若强迫力是正弦函数$F(t) = f cos(\gamma t + \beta)$
                $$\Rightarrow x = A\ cos(\omega t + \alpha) + \frac{f}{m(\omega^2 - \gamma^2)} cos(\gamma t + \beta)$$
                
            \textbf{[2.2]} \textbf{共振}:若强迫力是正弦函数,且$\gamma = \omega$.\quad 振幅随时间线性增大,直至不再是小量,理论不再适用为止.
            $$\Rightarrow \lim_{\gamma \to \omega}x = A'cos(\omega t + \alpha) + \frac{f t}{2m\omega} sin(\omega t + \beta)$$
            
            \textbf{[2.3]} 共振附近:若强迫力是正弦函数,且$\gamma = \omega + \epsilon$.\quad 幅度以频率$\epsilon$变化(\textbf{拍频})的小振动.
            $$\Rightarrow x = (A' + B' e^{i\epsilon t})e^{i\omega t} \quad ,A' = Ae^{i\alpha},B' = Be^{i\beta},|A' + B' e^{i\epsilon t}| \in \{|A-B|,A+B\} $$
            
            \textbf{[2.3]} 任意强迫力
            $$let\ \xi = \dot x + i\omega x\ \Rightarrow \dot \xi - i\omega \xi = \frac{F(t)}{m}  \ \Rightarrow x = \frac{1}{\omega}Im\{\int_0^t \frac{F(t)}{m}e^{-i\omega t}dt + const.\}$$


        \subsubsection{多自由度小振动}


    \subsection{力学之对象——刚体}
        \textbf{刚体}:质点间距离保持不变的质点组成的系统.
        
        \textbf{角速度$\vec \omega$}:
        
        \textbf{惯性张量}:
    
    
    
    \subsection{力学之对象——理想流体}
        \textbf{理想流体}:不可压缩、不计粘性的流体.
        
        · 给定5个量:速度$(v_x,v_y,v_z)$,压强$p$,密度$\rho$,可完全确定运动流体的状态.
        
        \textbf{[1] 运动不变性 —— 质量守恒}
            区域体流出质量 = 区域封闭面流出质量
            $$\Rightarrow \oint \rho \vec v \cdot d \vec f = -\frac{\partial}{\partial t}\int \rho dV \quad \Rightarrow \int \nabla \cdot (\rho \vec v) dV  = -\frac{\partial}{\partial t}\int \rho dV \Rightarrow  \int (\nabla \cdot (\rho \vec v) + \frac{\partial \rho}{\partial t}) dV  = 0$$
        
        \textbf{连续性方程}:质量流流出速率 = 流体密度减少速率
            $$\Rightarrow \nabla \cdot (\rho \vec v) +  \frac{\partial \rho}{\partial t} = 0$$
        
        \textbf{质量流密度}:
            $$\vec j = \rho \vec v$$
        
        \textbf{[2]运动方程}
            合力:
            $$-\oint p d \vec f = -\int \nabla p\ dV$$
            $$\rho \frac{d \vec v}{t} = -\nabla p$$
        
        \textbf{Euler方程}:
            $$\frac{\partial \vec v}{\partial t} + (\vec v \cdot \nabla)\vec v = - \frac{1}{\rho}\nabla p$$


\section{Relativity Mechanics 相对论力学}
    \subsection{相对性原理,相互作用传播速度}
        \textbf{相对性原理}: 所有物理定律,在所有惯性参考系中都相同.
        
        · 实验表明, 相对性原理是有效的.
        
        · 实验表明, 瞬时相互作用在自然界不存在,相互作用的传播需要时间.
        
        \textbf{相互作用的传播速度},在所有惯性参考系中都一样(相对性原理可得).\quad 电动力学中证明,这个速度是光在真空中的速度.
            $$c = 2.998 \times 10^8 m/s$$
            (取$c\to \infty$,即可过渡到经典力学.)


    \subsection{相对时间}
        · <1881年Michelson-Morley干涉实验>表明, 光速与其传播方向无关. (而按经典力学,光应在地球速度同方向(v+c),比反方向(v-c)更快一点.)
        
        $\Rightarrow$Galilean变换的绝对时间假设(t=t')错了.\quad  $\Rightarrow$不同参考系,时间流逝的速度不同.
        
        \textbf{事件}:由事件发生的位置(x,y,z)和时间(t)决定.
        
        \textbf{事件间隔}:
            $$S_{12} = [(ct_2-ct_1)^2 - (x_2-x_1)^2 - (y_2-y_1)^2 - (z_2-z_1)^2]^{1/2}$$
            $\Rightarrow$ 两个事件的间隔在所有参考系中都一样.\quad 这个不变性,就是光速不变的数学表示.
        
        \textbf{固有时}:与物体一同运动的钟所指示的时间.
        
        \textbf{固有长度}:物体在相对静止参考系内的长度.


    \subsection{参考系间变换}
        \textbf{Lorentz变换}: 参考系间变换
            $$ x = \frac{x' + V t'}{\sqrt{1 - \frac{V^2}{c^2}}},\quad y=y',\quad z=z', \quad t = \frac{t'+ \frac{V}{c^2}x'}{\sqrt{1 - \frac{V^2}{c^2}}}$$
            $$ \Rightarrow dx = \frac{dx' + V dt'}{\sqrt{1 - \frac{V^2}{c^2}}},\quad dy=dy',\quad dz=dz', \quad dt = \frac{dt'+ \frac{V}{c^2}dx'}{\sqrt{1 - \frac{V^2}{c^2}}}$$
    
        \textbf{速度变换}: $\vec v = \frac{d\vec r}{dt},\quad v' = \frac{d\vec r'}{dt}$
            $$\Rightarrow v_x = \frac{v'_x + V}{1 + v'_x \frac{V}{c^2}}, \quad v_y = \frac{v'_y \sqrt{1 - \frac{V^2}{c^2}}}{1 + v'_x \frac{V}{c^2}},\quad v_z = \frac{v'_z \sqrt{1 - \frac{V^2}{c^2}}}{1 + v'_x \frac{V}{c^2}}$$
            
        · 例1: \textbf{钟慢}:
    
        · 例2: \textbf{尺缩}:


    \subsection{力学量}
        \textbf{Lagrange函数}:
            $$L = -m c^2 \sqrt{1 - \frac{v^2}{c^2}}$$
    
        \textbf{动量$\vec P$}:
            $$\vec P = \frac{\partial L}{\partial \vec v}= (\frac{\partial}{\partial v_x} \hat{v_x}, \frac{\partial}{\partial v_y} \hat{v_y}, \frac{\partial}{\partial v_z} \hat{v_z})(-m c^2 \sqrt{1 - \frac{v_x^2 + v_y^2 + v_z^2}{c^2}}) = \frac{m \vec v}{\sqrt{1 - \frac{v^2}{c^2}}}$$
        
        \textbf{力$\vec F$}:
            $$\vec F = \frac{d\vec P}{dt} \to \frac{m}{\sqrt{1 - \frac{v^2}{c^2}}} \frac{d\vec v}{dt}(\vec F \perp \vec v) \quad or\quad  \frac{m}{(1 - \frac{v^2}{c^2})^{3/2}} \frac{d\vec v}{dt} (\vec F \parallel \vec v)$$
    
        \textbf{能量E}:
            $$E = \sum \dot q_i \frac{\partial L}{\partial \dot q_i} - L = \vec P \cdot \vec v - L = \frac{m v^2}{\sqrt{1 - \frac{v^2}{c^2}}} + m c^2 \sqrt{1 - \frac{v^2}{c^2}} = \frac{mc^2}{\sqrt{1 - \frac{v^2}{c^2}}}$$
    
        \textbf{静能}:\quad $E(v=0) = mc^2$\\


\section{Electromagnetics 电磁学}
    \subsection{电磁场方程}
        · 事实表明,粒子同电磁场相互作用的性质,由粒子电荷$q$所决定.
    
        \textbf{四维势$A_{i}$: \quad 标势$\varphi$: \quad 矢势$\vec A$:}
            $$A^{i}=(\varphi,\vec A)$$
            
        运动方程:
            $$\frac{d\vec P}{dt} = - \frac{e}{c} \frac{\partial\vec A}{\partial t} - e \nabla \varphi + \frac{e}{c} \vec v \times \nabla \times \vec A$$
            
        \textbf{电场强度$\vec E$:\quad 磁场强度$\vec H$:}
            \begin{displaymath}
                \left\{ \begin{array}{ll}
                \vec E = -\frac{1}{c} \frac{\partial \vec A}{\partial t} - \nabla \varphi\\
                \vec H = \nabla \times \vec A
                \end{array} \right.
            \end{displaymath}
            $$\Rightarrow \frac{d\vec P}{dt} = e \vec E + \frac{e}{c} \vec v \times \vec H$$
            
        对$\vec E,\vec H$取旋散度, 有
        
        \textbf{Maxwell's方程组}:
            \begin{displaymath}
                \left\{ \begin{array}{ll}
                \nabla \cdot \vec E = 4\pi\rho\\
                \nabla \cdot \vec H = 0\\
                \nabla \times \vec E = - \frac{1}{c} \frac{\partial \vec H}{\partial t}\\
                \nabla \times \vec H = - \frac{1}{c} \frac{\partial \vec E}{\partial t} + \frac{4\pi}{c}\vec j
                \end{array} \right.
            \end{displaymath}


\subsection{特解: 静电场}
静电场  /  恒磁场
*	[公式]: 
		▽²A  = -4π/c·J		磁矢势 (恒磁场)
		▽²φ = -ρ/ε0			电  势 (静电场)
		电场强度: E = -▽ φ
		磁场强度: H = -▽×A
		▽·E = ρ/ε0			(静电场)
		▽×E = 0
		▽·H = 0				(恒磁场)
		▽×H = 4π/c·J
*	[算法]:	Poisson's方程		
		当ρ=0时, ▽²φ = 0		Laplace's方程
		解Poisson's方程,Green's函数,得 φ(r) = - 4π/ε0 ∫∫∫ f(rt) / |r-rt| d³rt
*	[静电场唯一性定理]:
		对于各种边界条件,Poisson's方程有许多种解,但每个解梯度相同.
		静电场下, 意味边界条件下满足Poisson's方程的势函数,所解得电场唯一确定.

\subsection{特解: 恒磁场}


\subsection{特解: 真空电磁波}



\section{流体力学}
    Navier Stokes 流体方程
    ∂\vec u/∂t + (\vec v·▽)\vec v =  - 1/ρ·▽p  + \vec g + η/ρ▽²\vec v + (ζ + η/3)/ρ·▽(▽·\vec v)
			η: 粘度    ρ: 密度
			不可压缩流: ▽·\vec v ≡ 0
			压强: ▽·▽p = ▽·v·ρ/ dt

*                   Eular 理想流体方程
*	[公式]: ∂\vec u/∂t + (\vec v·▽)\vec v = - 1/ρ·▽p + \vec g
		分量式:
			∂u_x/∂t + v_x·∂v_x/∂x+ v_y·∂v_x/∂y + v_z·∂v_x/∂z = - 1/ρ·∂p/∂x + g_x


\section{Gravitational Field 引力场}



\section{Quantum Mechanics 量子力学}
*                    Schrödinger 方程
*	[定义]: iℏ·∂ψ(r,t)/∂t = [-ℏ/2m▽² + U(r,t)]·ψ(r,t)
		or	iℏ·d/dt·|ψ(t)> = H |ψ(t)>
		[符号]:
			ℏ: 约化Planck常量 = h / 2π    U(r,t): 势
			H: Hamiltonian算子 = -ℏ/2m▽² + U(r)
			ψ: the state vector of the quantum system, letter psi.
		[* Time-independent Schrödinger]:
			[-ℏ/2m▽² + U(r)]·ψ(r)  = Eψ(r)
		or	H|ψ> = E|ψ>
		[符号]: E: 系统的能级, 常量.
		* In the language of linear algebra, this equation is an eigenvalue equation.
		  Therefore, the wave function is an eigenfunction of the Hamiltonian operator
		  with corresponding eigenvalue(s) E.
*	[算法]: 有限差分法
		* ∂ψ/∂x   = [ψ(x+1,...) - ψ(x-1,...)] / 2Δt
		* ∂²ψ/∂x² = [ψ(x+1,...) - 2·ψ(x,...) + ψ(x-1,...)] / Δt²


\section{Statistical Mechanics 统计力学}


\end{document}
