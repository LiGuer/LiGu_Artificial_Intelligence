
\section{凸优化}
    \section{仿射、凸、锥}
        \item \def{仿射集}  {C是仿射集 $\Leftrightarrow$ 任意两点间直线仍在C内. $\forall \vec x_i \in C,\ \theta_i \in R, \sum \theta_i = 1 , then\ \sum \theta_i x_i \in C $}
        \item \def{仿射包}  {$ \text{aff}\ C =   \{\sum \theta_i x_i\ |\ x_i\in C,\theta_i \in R, \sum \theta_i = 1  \}$}
        \item \def{凸集}    {C是凸集 $\Leftrightarrow$ 任意两点间线段仍在C内.$\forall \vec x_i \in C, \theta_i \in [0,1], \sum \theta_i = 1 , then\ \sum \theta_i x_i \in C $}
        \item \def{凸包}    {$\text{conv}\ C =   \{\sum \theta_i x_i\ |\ x_i\in C, \theta_i \in [0,1], \sum \theta_i = 1  \}$}
        \item \def{锥}      {C是锥 $\Leftrightarrow \forall \vec x \in C,\ \theta > 0,\ then\ \theta \vec x \in$ C.}
        \item \def{凸锥}    {C是凸锥 $\Leftrightarrow \forall \vec x_1, \vec x_2 \in C,\ \theta_1,\theta_2 \ge 0,\ then\ \theta_1 \vec x_1 + \theta_2 \vec x_2 \in$ C.}
        
    \section{凸函数}
        \def{凸函数}{函数$f: \mathbf{R}^n \to \mathbf{R}$是凸的$\Leftrightarrow$ 若 $\mathbf{dom} f$是凸集, 对$\forall x,y \in \mathbf{dom} f, \forall \theta \in [0,1]$有 $f(\theta x + (1 - \theta) y) \le \theta f(x) + (1 - \theta)f(y)$}
            
        \bf{证明条件}:
        \item \bf{一阶条件}: $f(y) \geqslant f(x)+\nabla f(x)^{T}(y-x)$
        \item \bf{二阶条件}: $\nabla^{2} f(x) \succeq 0$
        
        \def{拟凸函数}{函数$f: \mathbf{R}^n \to \mathbf{R}$是拟凸函数$\Leftrightarrow$ 对$\forall \alpha \in R ,\ \{x \in \mathbf{dom} f \mid f(x) \le \alpha\}$都是凸集.}
        
    \section{凸优化}
        \def{优化问题}{
            目标函数$f_0(x)$, 不等式约束$f_i(x)$, 等式约束$h_i(x)$.
            $
                \min \quad& f_0(x) 
                s.t. \quad& f_i(x) \le 0
                          & h_i(x) = 0
            $
            \bf{最优解}: $p^{\star}=\inf  \{f_{0}(x) \mid f_i(x) \le 0, h_i(x) = 0 \}$\
        }
            \bf{例子}
                \def{可行性问题}{
                    如果目标函数恒等于零,则最优解是0或∞ (若可行集非空或空集).
                    $
                        \min \quad& x 
                        s.t. \quad& f_i(x) \le 0
                                  & h_i(x) = 0
                    $
                }

        \def{凸优化问题}{
            对凸函数$f_0, f_i$
            $
                \min \quad& x
                s.t. \quad& f_i(x) \le 0
                          & a_i^T x = b_i
            $
        }
            \bf{性质}
            \item 目标函数必须是凸.
            \item 不等式约束函数必须是凸.
            \item 等式约束函数$h_i(x)= a_i^T x- b_i$必须是仿射.

            \bf{例子}
            \item \def{线性规划}{
                    目标函数和约束函数都是放射的优化问题.
                    $
                        \min \quad& a^T x + b
                        s.t. \quad& Gx \preceq 0
                                  & Ax = b
                    $
                    几何意义: 可行解集是多面体, 等位曲线是与向量$a^T$正交的超平面, 最优解是多面体中在$-a^T$方向最远的点.
                }
            \item \def{线性分式规划}{
                    $
                        \min \quad& \frac{a^T x+ b}{c^T x + d}
                        s.t. \quad& Gx \preceq 0
                                  & Ax = b
                    $
                    该问题可以等价转化为线性规划.
                }
            \item \def{二次规划}{
                    $
                        \min \quad& \frac{1}{2} x^T P x + q^T x + r
                        s.t. \quad& Gx \preceq 0
                                  & Ax = b
                    $
                }
                    \df{例子}
                        最小二乘法$\min ||Ax + b||_2^2$
            \item \def{二次约束二次规划}{
                    $
                        \min \quad& \frac{1}{2} x^T P_0 x + q_0^T x + r_0
                        s.t. \quad& \frac{1}{2} x^T P_i x + q_i^T x + r_i \preceq 0
                                  & Ax = b
                    $
                }
            \item \def{二次锥规划}{
                    $
                        \min \quad& f^T x
                        s.t. \quad& ||A_i x + b_i|| \le c_i^T + d_i
                                  & Fx = g
                    $
                }
            \item \def{几何规划}{
                    $

                    $
                }
            \item \def{半正定规划}{
                    $

                    $
                }

                \df{关系} $二次锥规划 \supset \{二次规划 \supset \{ 线性规划 \} , 二次约束二次规划\}$
            
    \section{对偶}
        \def{Lagrange函数}{$L(x, \lambda, \nu) = f_0(x) + \sum_i \lambda_i f_i(x) + \sum_i \nu_i h_i(x)$}
        \def{Lagrange对偶函数}{$g(\lambda, \nu)=\inf _{x \in \mathcal{D}} L(x, \lambda, \nu)=\inf _{x \in \mathcal{D}} (f_{0}(x)+\sum_{i=1}^{m} \lambda_i f_i(x)+\sum_{i=1}^{p} \nu_i h_i(x) )$}
            \bf{性质}
            \item Lagrange对偶函数是凹函数.
            \item 若$\lambda \succeq 0$, 则$g(\lambda, \upsilon) \le p^*$, 即给出原问题最优解的一个不平凡下界.
        \def{对偶问题}{
            $
                \max \quad& g(\lambda, \upsilon)
                s.t. \quad& \lambda \succeq 0
            $
        }
        
\end{document}
