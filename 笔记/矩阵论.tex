\documentclass{article} 
\usepackage{amsmath}
\usepackage[UTF8]{ctex}
\title{}\date{} \setlength{\parindent}{0pt} \linespread{1.25}
\usepackage{ntheorem}\usepackage{amssymb}
\usepackage{graphicx}
\usepackage{geometry} \geometry{a4paper,left=2cm,right=2cm,top=1cm,bottom=1.5cm}

\begin{document}
\tableofcontents

\section{矩阵论}
    \subsection{线性空间}
        \textbf{线性空间}: 一个非空集合, 带有加法和数乘, 且满足下列几项条件.
            \begin{enumerate}
                \item 加法   $x+y \in V$
                \item 数乘   $k x \in V$
                \item 存在零元 $x+0=x$
                \item 存在负元 $x+(-x) = 0$
                \item $1x = x$
                \item 交换律 $x+y = y+x$
                \item 分配律 $(k+l)x = kx+lx$
                \item 加法结合律 $x+(y+z) = (x+y) +z$
                \item 数乘结合律 $k(Lx) = (kl)x$
            \end{enumerate}
            
            \textbf{例子}:
                (1) 一个平面 $\mathbb R^2 = {(x,y): x, y\in \mathbb R}$
                (2) 一个三维空间 $\mathbb R^3 = {(x,y,z): x, y, z\in \mathbb R}$

            \textbf{性质}:


        \textbf{线性组合}: $x = a_1 x_1 +a_2 x_2+ ...+a_m x_m, \text{且}\ x \in V$

            \textbf{性质}: 
                \textbf{线性无关/线性相关}: $\nexists / \exists\ \boldsymbol c \neq \boldsymbol 0, let x = a_1 x_1 + a_2 x_2+ ...+a_m x_m = 0$ 

        \textbf{维数}: 线性空间中, 线性无关向量组所含向量最大的个数.
        
        \textbf{基}: 一个向量组, 且 (1) 线性无关; (2) 线性空间中$\forall x \in V$都是该向量组的线性组合. $\forall x \in V, and x = \sum\limits_{i=1}^{D} a_i x_i$
            \textbf{基向量}: 基中的一个向量. $x_i$
            \textbf{坐标}: 向量在基中线性表示的系数. $a_i \in \boldsymbol a$

        \textbf{性质}: 
            \textbf{基变换}: 新旧基之间的变换矩阵. $\boldsymbol y = \boldsymbol x \boldsymbol C$
                \textbf{性质}: 基变换矩阵是非奇异矩阵.

            \textbf{坐标变换}: $\boldsymbol a = \boldsymbol C \boldsymbol b$.
                \textbf{证明}: $v = \boldsymbol x \boldsymbol a = \boldsymbol y \boldsymbol b = \boldsymbol x \boldsymbol C \boldsymbol a \quad \Rightarrow \quad \boldsymbol a = \boldsymbol C \boldsymbol b$

            
        \textbf{线性子空间}: 线性空间中的一个非空集合, 且 (1)加法 $x,y\in V_1 ,\quad x+y \in V_1$; (2)数乘 $x \in V_1, k x \in V_1$
            \textbf{性质}: 
                (1) 线性子空间也是线性空间.这是因为$V_1$是V的子集合,所以$V_1$中的向量不仅对线性空间V已定义的线性运算封闭,而且还满足相应的8条运算律.
                (2) 每个非零线性空间至少有两个子空间,一个是自身,另一个是仅由零向量所构成的子集合,称后者为零子空间.             
                (3) $dim\ V_1 \le dim\ V$

        \textbf{张成}: $span(\vec x_1,...,\vec x_n ) = {\sum_{i=1}^n a_i \vec_i}$ 
            
        \textbf{列空间}: 设 $A = (a_{ij})\in R^{m×n}$, 以ai(i= 1, 2,..., n)表示
        A的第i个列向量,称子空间span(a1, a2,...,a_n)为矩阵A的值域(列空间). R(A = span(a1, a2,...,a_n))
            \textbf{性质}: $rand A = dim R(A)$

        \textbf{零空间}: 设 $A = (a_{ij})\in R^{m×n}, \{x | Ax = 0\}$
        
        \textbf{交}: 
        \textbf{和}: $V_{1}+V_{2}=\left\{z \mid z=x+y, x \in V_{1}, y \in V_{2}\right\}$
        \textbf{直和}: 若V1+V2中的任一向量只能唯一地表示为子空间V1的一个向量与子空间V2的一个向量的和.
            \textbf{性质}: 
            (1) 子空间的交、和, 也是子空间. $V_1, V_2 \subseteq V,\ V_1 \cap V_2 \subseteq V,\ V_1 + V_2 \subseteq V$
            (2) $\operatorname{dim} V_{1}+\operatorname{dim} V_{2}=\operatorname{dim}\left(V_{1}+V_{2}\right)+\operatorname{dim}\left(V_{1} \cap V_{2}\right)$
            (3)  $U=V_{1} + V_{2}, then\ \quad U=V_{1} \oplus V_{2} \quad \Leftrightarrow \quad \operatorname{dim} U=\operatorname{dim}\left(V_{1}+V_{2}\right)=\operatorname{dim} V_{1}+\operatorname{dim} V_{2} $

        
    \subsection{线性变换}
        \textbf{变换}: 线性空间到自身的映射, 且$\forall x \in V$都有唯一$y \in V$与之对应.

        \textbf{线性变换}: 一种变换, 且$T(k \boldsymbol{x}+l \boldsymbol{y})=k(T \boldsymbol{x})+l(T \boldsymbol{y})$

        \textbf{值域}: $R(T)=\{T \boldsymbol{x} \mid \boldsymbol{x} \in V\}$

            \textbf{性质}: 线性空间V的线性变换T的值域和核都是V的线性子空间.

        \textbf{线性变换矩阵}: $T \boldsymbol X = \boldsymbol X \boldsymbol A$ 
            \textbf{性质}: 
            (1)运算
                \begin{itemize}
                    \item $(T_1 + T_2) X = X (A + B)$
                    \item $(k\ T_1) X = X (k\ A)$
                    \item $(T_1 T_2) X = X AB$
                    \item $T_1^{-1} X = X A^{-1}$
                \end{itemize}
            (2) 坐标变换: $\boldsymbol b = \boldsymbol A \boldsymbol a$

        \textbf{相似}: $\exists \text{非奇异矩阵}P, then \quad \boldsymbol A \sim \boldsymbol B \quad \Leftrightarrow \quad \boldsymbol B = \boldsymbol P^{-1} \boldsymbol A \boldsymbol P$
            \textbf{性质}: 
            (1) $\boldsymbol A \sim \boldsymbol A$ 
            (2) $\boldsymbol A \sim \boldsymbol B \Leftrightarrow \boldsymbol B \sim \boldsymbol A$ 
            (3) $\boldsymbol A \sim \boldsymbol B, \boldsymbol B \sim \boldsymbol C \Leftrightarrow \boldsymbol A \sim \boldsymbol C$

        \textbf{特征值 \& 特征向量}: $T \boldsymbol x = \lambda_0 \boldsymbol x$
        

\end{document}
