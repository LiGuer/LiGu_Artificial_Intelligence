\documentclass{article} 
\usepackage{amsmath}
\usepackage[UTF8]{ctex}
\title{信息论}\date{} \linespread{1.25}
\usepackage{ntheorem}\usepackage{amssymb}
\usepackage{graphicx}
\usepackage{geometry} \geometry{a4paper,left=2cm,right=2cm,top=1cm,bottom=1.5cm}
\usepackage{paralist}
\let\itemize\compactitem
\let\enumerate\compactenum

\newcommand{\env}[2]{\begin{#1}#2\end{#1}}
\newcommand{\defi}[2]{\textbf{#1}, #2}
\newcommand{\proof}[1]{\textbf{证明} #1}
\newcommand{\bb}{\boldsymbol}
\begin{document}
\tableofcontents


\defi{矩阵}
    \defi{对称矩阵}
    \defi{正交矩阵}

\section{线性空间}
    \defi{线性空间}{一个非空集合, 带有加法和数乘, 且满足下列几项条件.}
        \begin{enumerate}
            \item 加法   $x+y \in V$
            \item 数乘   $k x \in V$
            \item 存在零元 $x+0=x$
            \item 存在负元 $x+(-x) = 0$
            \item $1x = x$
            \item 交换律 $x+y = y+x$
            \item 分配律 $(k+l)x = kx+lx$
            \item 加法结合律 $x+(y+z) = (x+y) +z$
            \item 数乘结合律 $k(Lx) = (kl)x$
        \end{enumerate}

    \textbf{例子}:
        (1) 一个平面 $\mathbb R^2 = {(x,y): x, y\in \mathbb R}$
        (2) 一个三维空间 $\mathbb R^3 = {(x,y,z): x, y, z\in \mathbb R}$

        \defi{Euclid空间}{定义了内积的实线性空间, 且满足 \evn(enumerate){
            \item 交换律:$(\bb x , \bb y )=(\bb y , \bb x )$
            \item 分配律:$(\bb x , \bb y +\bb z )=(\bb x , \bb y )+(\bb x , \bb z )$
            \item 齐次性:$(k \bb x , \bb y )=k(\bb x , \bb y )(\forall k \in \bb R )$
            \item 非负性:$(\bb x , \bb x ) \ge 0$, 当且仅当 $\bb x =\bb 0 ,(\bb x , \bb x )=0$
        }}
            \defi{度量矩阵}
            \defi{Cauchy-Бyнияковскнй不等式}{$|(\boldsymbol{x}, \boldsymbol{y})| \leqslant|\boldsymbol{x}||\boldsymbol{y}|$}
        \defi{unitary空间}


    \defi{线性子空间}{线性空间中的一个非空集合, 且 (1)加法 $x,y\in V_1 ,\quad x+y \in V_1$; (2)数乘 $x \in V_1, k x \in V_1$}

        \textbf{性质}: 
            (1) 线性子空间也是线性空间.这是因为$V_1$是V的子集合,所以$V_1$中的向量不仅对线性空间V已定义的线性运算封闭,而且还满足相应的8条运算律.
            (2) 每个非零线性空间至少有两个子空间,一个是自身,另一个是仅由零向量所构成的子集合,称后者为零子空间.             
            (3) $dim\ V_1 \le dim\ V$


\defi{最小多项式}

\defi{特征值 \& 特征向量}{线性变换前后, 方向不改变的向量, 称特征向量$\lambda$, 特征向量在线性变换后长度变化的倍率, 称特征值$\bb x$. $$T \bb x = \lambda \bb x$$.}

Hamilton-Cayley定理 $$\varphi(\lambda)=\operatorname{det}(\lambda-\bb{A})=\lambda^{n}+a_{1} \lambda^{n-1}+\cdots+a_{n-1} \lambda+a_{n}$$ $$\varphi(\bb{A})=\bb{A}^{n}+a_{1} \bb{A}^{n-1}+\cdots+a_{n-1} \bb{A}+a_{n} \bb{I}=\bb{O}$$


\section{范数}:
    \defi{范数}{}

    \defi{向量范数}{一类函数, 且满足\env{
        \item $||\bb A|| \ge 0$, 当且仅当$\bb A = \bb 0, ||\bb A|| = 0$
        \item $||k \bb A|| = |k| ||\bb A||$
        \item $||\bb A + \bb B|| \le ||\bb A|| + ||\bb B||$
    }}
    \defi{矩阵范数}{一类函数, 且满足\env{
        \item $||\bb A|| \ge 0$, 当且仅当$\bb A = \bb 0, ||\bb A|| = 0$
        \item $||k \bb A|| = |k| ||\bb A||$
        \item $||\bb A + \bb B|| \le ||\bb A|| + ||\bb B||$
        \item $||\bb A \bb B|| \le ||\bb A||\ ||\bb B||$
    }}

\section{矩阵分解}:
    \defi{上下三角分解}{将矩阵A化成上三角矩阵R与下三角矩阵L的乘积.$A = L R$}
    \defi{上下三角对角分解}{将矩阵A化成上三角矩阵R, 对角矩阵D, 下三角矩阵L的乘积.$A = L D R$}

    \defi{正交三角分解}{将实(复)非奇异矩阵A化成正交(酉)矩阵Q与实(复)非奇异上三角矩阵R的乘积.$A = Q R$}
        \textbf{算法}: \env{enumerate}{
            \item $\bb A = [\bb a_1, ..., \bb a_n]$
            \item Schmidt正交化 \[\bb b_i = \bb a_i - \sum\limits_{k=1}^{i-1} \frac{<\bb a_i,\bb b_j>}{<\bb b_j,\bb b_j>}\bb b_j\]
            \item 正交矩阵Q
                \[\bb Q = \left[ \frac{\bb b_1}{|\bb b_1|}, ... , \frac{\bb b_n}{|\bb b_n|} \right]\]
            \item 非奇异上三角矩阵R
                \[\bb R = \begin{bmatrix} |\bb{b}_{1}|\\ & \ddots\\ && |\bb{b}_{n}| \end{bmatrix} \bb C\]\
        }

    \defi{满秩分解}{}
    \defi{奇异值分解}{}
        \textbf{算法}: \env{enumerate}{
            \item $\bb A^T \bb A$ 计算特征值 $\lambda$, 特征向量$\bb x$
            \item $V = \left[ \frac{\bb x_1}{|\bb x_1|}, ... ,\frac{\bb x_n}{|\bb x_n|} \right], \quad \Sigma = diag(\lambda_1, ... ,\lambda_n)$
            \item $U_1 = A V \Sigma^{-1}$, 计算正交矩阵$U$
            \item 分解结果 $A = U \Sigma V^T$
        }


    \defi{初等旋转变换}{
        \[T_{ij} = I + \begin{pmatrix}\mathbf 0\\ & \cos\theta|_{(i,i)}& \mathbf 0 & \sin\theta|_{(i,j)}& \\ & \mathbf 0&\mathbf 0&\mathbf 0\\& \sin\theta|_{(j,i)}& \mathbf 0 & \cos\theta|_{(j,j)}\\ &&&&\mathbf 0\end{pmatrix} - \begin{pmatrix}\mathbf 0\\ &1|_{(i,i)}\\&&\mathbf 0\\&&& 1 |_{(j,j)}\\& &&\mathbf 0\end{pmatrix}\]
    }
        \textbf{性质}: \env{enumerate}{
            \item $T$是正交矩阵.
            \item 设$\bb{x}=\left(\xi_{1}, \cdots, \xi_{n}\right)^{\mathrm{T}}, \quad \bb{y}=\bb{T}_{ij}\ \bb{x}=\left(\eta_{1}, \cdots, \eta_{n},\right)$, 则
                \[\left\{\begin{array}{l}
                \eta_{i}=c \xi_{i}+s \xi_{j} \\
                \eta_{j}=-s \xi_{i}+c \xi_{j} \\
                \eta_{k}=\xi_{k} \quad(k \neq i, j)
                \end{array}\right.\]
            且当$\xi_{i}^{2}+\xi_{j}^{2} \neq 0$时, $c=\frac{\xi_{i}}{\sqrt{\xi_{i}^{2}+\xi_{j}^{2}}}, \quad s=\frac{\xi_{j}}{\sqrt{\xi_{i}^{2}+\xi_{j}^{2}}}$, 可使$\eta_{i}=\sqrt{\xi_{i}^{2}+\xi_{j}^{2}}>0, \eta_{j}=0$.
        }

    \defi{初等反射变换}{$\bb{H} \bb{x} = \left(\bb{I}-2 \bb{e}_{2} \bb{e}_{2}^{\mathrm{T}}\right) \bb{x}$}
        \textbf{性质}: \env{enumerate}{
            \item 对称矩阵$\bb H^T = \bb H$, 正交矩阵$\bb H^T \bb H = I$, 对合$\bb H^2 = \bb I$, 自逆$\bb H^{-1} = \bb H$, $|\bb H| = -1$.
        }

    \textbf{关系}: 初等旋转矩阵是两个初等反射矩阵的乘积.









\section{矩阵论}
    \subsection{矩阵}
        \subsubsection{加减乘除}

        \subsubsection{内积}

        \subsubsection{叉乘}

        \subsubsection{范数}


    \subsection{线性空间}
        \textbf{线性空间}: 一个非空集合, 带有加法和数乘, 且满足下列几项条件.
            \begin{enumerate}
                \item 加法   $x+y \in V$
                \item 数乘   $k x \in V$
                \item 存在零元 $x+0=x$
                \item 存在负元 $x+(-x) = 0$
                \item $1x = x$
                \item 交换律 $x+y = y+x$
                \item 分配律 $(k+l)x = kx+lx$
                \item 加法结合律 $x+(y+z) = (x+y) +z$
                \item 数乘结合律 $k(Lx) = (kl)x$
            \end{enumerate}
            
            \textbf{例子}:
                (1) 一个平面 $\mathbb R^2 = {(x,y): x, y\in \mathbb R}$
                (2) 一个三维空间 $\mathbb R^3 = {(x,y,z): x, y, z\in \mathbb R}$

            \textbf{性质}:


        \textbf{线性组合}: $x = a_1 x_1 +a_2 x_2+ ... +a_m x_m = [x_1 ... x_n] \begin{bmatrix} a_1 = \bb X \bb a\\ \vdots \\ a_3 \end{bmatrix}, \text{且}\ x \in V$

            \textbf{性质}: 
                \textbf{线性无关/线性相关}: $\nexists / \exists\ \bb a \neq \bb 0, let x = \sum\limits_{i=1}^n a_i x_i = 0$ 

        \textbf{维数}: 线性空间中, 线性无关向量组所含向量最大的个数.
        
        \textbf{基}: 一个向量组, 且 (1) 线性无关; (2) 线性空间中$\forall x \in V$都是该向量组的线性组合. $\forall x \in V, and x = \sum\limits_{i=1}^{D} a_i x_i$. 其中, $x_i$称\textbf{基向量}, $a_i \in \bb a$称\textbf{坐标}.

            \textbf{性质}: 
                \textbf{基变换}: 新旧基之间的变换矩阵. $\bb Y = \bb X \bb C$
                    \textbf{性质}: 基变换矩阵是非奇异矩阵.

                \textbf{坐标变换}: $\bb a = \bb C \bb b$.
                    \textbf{证明}: $v = \bb x \bb a = \bb y \bb b = \bb x \bb C \bb a \quad \Rightarrow \quad \bb a = \bb C \bb b$

            
        \textbf{线性子空间}: 线性空间中的一个非空集合, 且 (1)加法 $x,y\in V_1 ,\quad x+y \in V_1$; (2)数乘 $x \in V_1, k x \in V_1$
            \textbf{性质}: 
                (1) 线性子空间也是线性空间.这是因为$V_1$是V的子集合,所以$V_1$中的向量不仅对线性空间V已定义的线性运算封闭,而且还满足相应的8条运算律.
                (2) 每个非零线性空间至少有两个子空间,一个是自身,另一个是仅由零向量所构成的子集合,称后者为零子空间.             
                (3) $dim\ V_1 \le dim\ V$

        \textbf{张成}: $span(\vec x_1,...,\vec x_n ) = {\sum_{i=1}^n a_i \vec_i}$ 
            
        \textbf{列空间}: 设 $A = (a_{ij})\in R^{m×n}$, 以ai(i= 1, 2,..., n)表示
        A的第i个列向量,称子空间span(a1, a2,...,a_n)为矩阵A的值域(列空间). R(A = span(a1, a2,...,a_n))
            \textbf{性质}: $rand A = dim R(A)$

        \textbf{零空间}: 设 $A = (a_{ij})\in R^{m×n}, \{x | Ax = 0\}$
        
        \textbf{交}: 
        \textbf{和}: $V_{1}+V_{2}=\left\{z \mid z=x+y, x \in V_{1}, y \in V_{2}\right\}$
        \textbf{直和}: 若V1+V2中的任一向量只能唯一地表示为子空间V1的一个向量与子空间V2的一个向量的和.
            \textbf{性质}: 
            (1) 子空间的交、和, 也是子空间. $V_1, V_2 \subseteq V,\ V_1 \cap V_2 \subseteq V,\ V_1 + V_2 \subseteq V$
            (2) $\operatorname{dim} V_{1}+\operatorname{dim} V_{2}=\operatorname{dim}\left(V_{1}+V_{2}\right)+\operatorname{dim}\left(V_{1} \cap V_{2}\right)$
            (3)  $U=V_{1} + V_{2}, then\ \quad U=V_{1} \oplus V_{2} \quad \Leftrightarrow \quad \operatorname{dim} U=\operatorname{dim}\left(V_{1}+V_{2}\right)=\operatorname{dim} V_{1}+\operatorname{dim} V_{2} $

        
    \subsection{线性变换}
        \textbf{变换}: 线性空间到自身的映射, 且$\forall x \in V$都有唯一$y \in V$与之对应.

        \textbf{线性变换}: 一种变换, 且$T(k \bb{x}+l \bb{y})=k(T \bb{x})+l(T \bb{y})$

        \textbf{值域}: $R(T)=\{T \bb{x} \mid \bb{x} \in V\}$

            \textbf{性质}: 线性空间V的线性变换T的值域和核都是V的线性子空间.

        \textbf{线性变换矩阵}: $T \bb X = \bb X \bb A$ 
            \textbf{性质}: 
            (1)运算
                \begin{itemize}
                    \item $(T_1 + T_2) X = X (A + B)$
                    \item $(k\ T_1) X = X (k\ A)$
                    \item $(T_1 T_2) X = X AB$
                    \item $T_1^{-1} X = X A^{-1}$
                \end{itemize}
            (2) 坐标变换: $\bb b = \bb A \bb a$

        \textbf{相似}: $\exists \text{非奇异矩阵}P, then \quad \bb A \sim \bb B \quad \Leftrightarrow \quad \bb B = \bb P^{-1} \bb A \bb P$
            \textbf{性质}: 
            (1) $\bb A \sim \bb A$ 
            (2) $\bb A \sim \bb B \Leftrightarrow \bb B \sim \bb A$ 
            (3) $\bb A \sim \bb B, \bb B \sim \bb C \Leftrightarrow \bb A \sim \bb C$

        \textbf{特征值 \& 特征向量}: $T \bb x = \lambda_0 \bb x$
        

        \textbf{Jordan标准型}
            $\bb{J}=\left[\begin{array}{llll}
                \bb{J}_{1}\left(\lambda_{1}\right) & & & \\
                & \bb{J}_{2}\left(\lambda_{2}\right) & & \\
                & & \ddots & \\
                & & & \bb{J}_{s}\left(\lambda_{s}\right)
                \end{array}\right]$

\end{document}
