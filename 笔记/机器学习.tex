\section{主成分分析}
		[目标]:
			数据降维,提取数据的主要特征分量, 满足:
    			
    			    * 最近重构性: 样本点到该超平面的距离都足够近。
    			    * 最大可分性: 样本点在该超平面的投影尽可能分开。
    			
			
		[优化问题]:
    		$
    			\min_W  &\quad	tr( W^T x x^T W )\\
    			s.t.    &\quad	W^T W = I
    		$
			
		[流程]:
			* 数据中心化, $\sum \vec x_i = 0$
			* 计算协方差矩阵 C = X X^T
			* 对协方差矩阵C 特征值分解
			* 取最大d'个特征值所对应的特征向量{w1,w2,...,wd'},投影矩阵 W = (w1,w2,...,wd')
			* 样本点在超平面投影: y_i = W^T x_i
			
		[原理]:
				分别从目标(1, 2)可以推得同样的结果
			*	目标函数: 样本点到超空间投影 y = WT x 尽可能分开, 即.方差最大:$\max \sum W^T x x^T W$
				协方差矩阵:
					$D = 1/m Y Y^T = 1/m (PX) (PX)^T = 1/m P X X^T P^T = 1/m P C P^T$
				协方差矩阵对角化
			* 优化问题构造:
			$
				min_W		tr( W^T x x^T W )
				s.t.		W^T W = I
			$
			* 计算最优点:
				Lagrange函数 $L(W,λ) = W^T x x^T W + λ( W^T W - I )$
				Lagrange对偶 $G(λ) = inf L(W,λ) = inf (W^T x x^T W + λ( W^T W - I ))$
				L(W,λ)求导, 当导数为0时, 取得极值
				=>	$X X^T ω_i = λ_i ω_i$
				即.对协方差X XT, 特征值求解
			*	取特征值最大的yDim个特征向量, 即目标投影矩阵W


\section{K-Mean 聚类}
		[目标]:
			聚类. 对N维分布的数据点,可以将其聚类在 K 个关键簇内.
			
		[步骤]:
			* 随机选择 K 个簇心点 Center
			* 迭代开始
				* 归零 Cluster , Cluster: 簇,记录ith簇内的数据指针。
				* 计算每个xi到簇心μj的距离
					* 选择距离最小的簇心, 将该点加入其簇内
				* 对每个簇,计算其质心 Center'
				* Center≠Center' , 则更正Center为 Center'
				* 迭代重新开始
			* 一轮无更正时,迭代结束

\section{最小二乘法}
		[目标]:
			求一条直线,使得所有样本点到该直线的Euclid距离最小.
			
		[优化问题]: 最小化均方误差
    		$
    			\min_w  &\quad MSE(\tilde{\bb y}) · n = \sum (\tilde {\bb y} -\bb y)^2 = \sum (\bb w^T \bb x - \bb y)^2
    		$
    		
		[原理]:
			* 直线方程: $f(x) = \bb w^T \bb x	\quad (\bb x = [1, \bb x_0])$
			* 均方误差:
					$MSE(\tilde y) · n = \sum (\tilde {\bb y} -\bb y)^2 = \sum (\bb w^T \bb x - \bb y)^2 = (X w - y)^T (X w - y)$
					
			* 优化问题构造: 无约束凸优化问题
					$\min_W	\sum (\bb w^T \bb x - y)^2 = (X w - y)^T (X w - y)$
					
			* 计算最优点: 求导, 导数为0时取得极值
				$
					\frac{\partial MSE}{\partial w} &= 2·X^T·(\bb w^T \bb x - y) = 0\\
				\Rightarrow	w^* &= (X^T X)^-1 X^T y\\
					f(x) &= x^T (X^T X)^-1 X^T y
				$
				
			* 一维场景:
				优化问题:
				$
					min_{w,b}	\sum (\bb w^T \bb x_i + b - y_i)^2\\
				    \Rightarrow	\frac{\partial E}{\partial w} = 2( w^T \sumx_i^2 + \sum(x_i(b - y_i)) )	= 0\\
					\frac{\partial E}{\partial b} = 2( n b + \sum(y_i - w x_i) ) = 0
				$
				
				最优点:
				$
				    \Rightarrow w \sum x_i^2 &= \sum_i x_i y_i - \frac{1}{n} (\sum_i x_i ) (\sum_i y_i) + \frac{w}{n} (\sum_i x_i)^2\\
					w^* &= \frac{\sum y_i(x_i - \bar x)}{\sum x_i^2 - \frac{1}{n} (\sum x_i)^2}\\
					b^* &= \frac{1}{n}·\sum(y_i - w x_i)\\
				$

					
\section{核方法}
    \defi{核方法}{将样本从原始空间映射到更高维特征空间,使得其线性可分.}
        $
			\mathbf{K}=[\begin{array}{ccccc}
            \kappa(\bb x_1, \bb x_1) & \cdots & \kappa(\bb x_1, \bb x_j) & \cdots & \kappa(\bb x_1, \bb x_{m}) \\
            \vdots & \ddots & \vdots & \ddots & \vdots \\
            \kappa(\bb x_{i}, \bb x_1) & \cdots & \kappa(\bb x_{i}, \bb x_j) & \cdots & \kappa(\bb x_{i}, \bb x_{m}) \\
            \vdots & \ddots & \vdots & \ddots & \vdots \\
            \kappa(\bb x_{m}, \bb x_1) & \cdots & \kappa(\bb x_{m}, \bb x_j) & \cdots & \kappa(\bb x_{m}, \bb x_{m})
        \end{array}]
		$
        
        $\begin{array}{l}
            \kappa(\bb x_{i}, \bb x_j)=\bb x_{i}^T \bb x_j \\
            \kappa(\bb x_{i}, \bb x_j)=(\bb x_{i}^T \bb x_j)^{d} \\
            \kappa(\bb x_{i}, \bb x_j)=\exp (-\frac{\|\bb x_{i}-\bb x_j\|^{2}}{2 \sigma^{2}}) \\
            \kappa(\bb x_{i}, \bb x_j)=\exp (-\frac{\|\bb x_{i}-\bb x_j\|}{\sigma}) \\
            \kappa(\bb x_{i}, \bb x_j)=\tanh (\beta \bb x_{i}^T \bb x_j+\theta)
        \end{array}$
		

\section{支持向量机}
    \defi{支持向量机}{
        是一种二分类模型,目的是找到一个超平面, 使得超平面与任意样本点之间的最小距离最大. 约束条件是超平面能够使得所有样本点正确分类. 支持向量机的优化问题形式(凸二次规划问题)如下. 
        $
           \underset{\bb w, b}{\arg\max} &\quad \min\limits_i d_i = \frac{\min\limits_i\ |\bb w^T \bb x_i + b |}{||\bb w||}\\
           s.t. &\quad y_i (\bb w^T \bb x_i + b) \le 1
        $
        \qquad \Rightarrow \qquad 
	    $
	        \underset{\bb w, b}{\arg\min}     &\quad	\frac{||\bb w||^2}{2}\\
			s.t.    &\quad	y_i (\bb w^T \bb x_i + b) \le 1
	    $
    }
    
    \proof{	\par		
        超平面方程: $\bb w^T \bb x + b = 0$, \quad 点面距: $d = \frac{|\bb w^T \bb x + b|}{||w||}$
		
		分类
			$ \{ \begin{array}{cc}
				\bb w^T \bb x_i + b \ge +1    &\quad (y_i = +1)\\
				\bb w^T \bb x_i + b \le -1	&\quad (y_i = -1)\\
			    \end{array} .
				\quad\Rightarrow\quad y_i (\bb w^T \bb x_i + b) \ge 1
			$

		间隔: 离超平面最近的2个异类样本点到超平面的距离之和.
			$\Delta = \frac{2}{||w||} \tag{间隔}$
					
		优化问题构造: 使间隔最大化. 凸二次规划问题
		    $\max\quad  \frac{2}{||\bb w||} \quad \Rightarrow \quad \min\quad  \frac{||\bb w||^2}{2}$
			$s.t.\quad	y_i (\bb w^T \bb x_i + b) \ge 1$
	}
    
	\bf{解}, 解凸优化问题, 计算最优点: 先求Lagrange函数, Lagrange函数对$\bb w, b$求导, 令导数为0取得极值$\bb w^*, b^*$, 再求Lagrange对偶函数
		$L(\bb w, b,\bb \lambda) = \frac{||\bb w||^2}{2} + \sum_i \lambda_i (1 - y_i (\bb w^T \bb x_i + b)) \tag{Lagrange函数}$
		$\Rightarrow \{\begin{array}{rl} \bb w^* &= \sum\limits_i \lambda_i y_i x_i\\ 0 &= \sum\limits_i \lambda_i y_i \end{array}.\tag{$\bb w, b$极值}$
		$G(\bb \lambda) = L(\bb w^*, b^*, \bb \lambda) = \sum \lambda_i - \frac{1}{2} \sum_i \sum_j \lambda_i \lambda_j y_i y_j \bb x_i^T \bb x_j \tag{Lagrange对偶}$

		得到对偶问题, 为二次规划问题形式, 可利用Sequential Minimal Optimization算法求解得到$\bb \lambda^*$
		    $
		        \max\limits_{\bb \lambda} &\quad G(\bb \lambda)\\
			    s.t. &  \quad \lambda_i \ge 0\\
			        & \quad \sum_i \lambda_i y_i = 0
		    $
		    
		其中, KKT条件如下
		    $
		        \lambda &\ge 0\\
		        y_i (\bb w^T \bb x_i + b) - 1 &\ge 0\\
		        \lambda_i(y_i(\bb w^T \bb x_i + b) - 1) &= 0
		    $
					
	\bf{算法步骤}:
		* 计算核矩阵
		* 计算$\lambda^*$
		
			* 选择 i
			* 选择 j
			* 计算 $K_{ii}+K_{jj}-2K_{ij}, L, H$
			* 更新 $\lambda_j, \lambda_i$
			* 更新 b
		
		* 计算 w*, b*
        

    \bf{核方法非线性支持向量机}:
		\bf{核函数}:将样本从原始空间映射到更高维特征空间,使得其线性可分.
			=>	超平面方程: $w^T \phi(x) + b = 0$
				min		||w||^2 / 2
				s.t.	y_i (w^T Φ(x_i) + b) ≥ 1
				G(λ) = \sumλ_i - 1/2 \sum_i \sum_j λ_i λ_j y_i y_j Φ(x_i)^T Φ(x_j)
			=>	设 核函数к(x_i, x_j) = Φ(x_i)^T Φ(x_j)
				f(x) = w*^T x + b = \sum λ_i y_i к(x, x_i) + b
			*	к是核函数 <=> 核矩阵[a_ij = к(x_i, x_j)]总是半正定
			
		\bf{Sequential Minimal Optimization算法}
			∵	λ是n-1自由度, 确定前n-1个量, 则第n个由\sum λ_i y_i = 0自动确定.
			∴	每次选2个λ_i λ_j, 固定其他λ_k不变, 优化λ_i λ_j, 更新b
			优化λ_i λ_j:
			=>	λ_i = (-\sum_{k≠i≠j} λ_k y_k)·y_1 - λ_j y_i y_j = ζ y_1 - λ_j y_i y_j
				G(λ_j) = (λ_j + ζ y_1 - λ_j y_i y_j) + C - v_i (ζ-λ_j y_j) - v_j λ_j y_j
						- 1/2 к_ii (ζ-λ_j y_j)^2 - 1/2 к_jj λ_j^2 - к_ij λ_j y_j (ζ-λ_j y_j)^2
				其中 v_i = \sum_{k≠i≠j} λ_i y_i к_ki = f(x_i) - λ_i y_i к_ii -λ_j y_j к_ij - b, v_j同理
				∂G/∂λ_j = -(к_ii+к_jj-2к_ij)(λ_j^old - λ_j^new) + y_j(y_j-y_i+f(x_i)-f(x_j)) = 0
			=>	λ_j^new = λ_j^old + y_j (E_i - E_j)/(к_ii+к_jj-2к_ij)		其中E_i = f(x_i) - y_i
			修剪: 使得λ_j* 满足约束条件, λ_j应当∈[L,H]:
				y_i≠y_j时,	下界L = max(0,λ_j^old-λ_i^old)	,上界L = min(C,λ_j^old-λ_i^old + C)
				y_i= y_j时,	下界L = max(0,λ_j^old+λ_i^old - C),上界L = min(C,λ_j^old+λ_i^old)
			更新b:
			
	\bf{软间隔}:

\section{神经网络}
    \section{前馈神经网络}
        \section{正向传播}
            $\sigma()$: 激活函数, 使线性拟合非线性化, eg. relu(x), Sigmoid(x)
            $\bb y = \sigma (\bb w \bb x + \bb b)$


            误差·损失函数: $E_{total} = \sum (target_i - out_i)^2$
            $E(\bb{w})=\frac{1}{2} \sum_{n=1}^{N}\|\bb{y}(\bb x_{n}, \bb{w})-\bb{t}_{n}\|^{2}$

        \section{反向传播 --- 梯度下降最优化}
    		$\bb{w}^{(\tau+1)}=\bb{w}^{(\tau)}-\eta \nabla E(\bb{w}^{(\tau)})$
    		
    		设$z = \bb w^T \bb x + \bb b$
            $ \{ \begin{array}{lr}
                \delta_l = (\bb w_{l+1}^T·\delta_{l+1})·\sigma'(z_l) & \text{(每层误差)}\\
                \delta_L = \nabla E · \sigma'(z_{outl}) & \text{(输出层误差)}\\
                \nabla E(\bb w_l) = \delta_l \bb x_l^T & \text{(每层参数误差)}
            \end{array} .$

            \bf{证明}:
        		
        		每层参数误差, 由链式法则得(其中$L$指输出层): 
        		    $
        		        \frac{\partial E}{\partial w_l} 
        		        &= \frac{\partial z_l}{\partial w_l} (\frac{\partial y_l}{\partial z_l}\frac{\partial z_l}{\partial y_{l+1}}) ... (\frac{\partial y_{L-1}}{\partial z_{L-1}} \frac{\partial z_{L-1}}{\partial y_L} ) \frac{\partial y_L}{\partial z_L} \frac{\partial E}{\partial y_L}\\
        		        &= \frac{\partial z_l}{\partial w_l}  (\prod_{i = l}^{L} \frac{\partial y_i}{\partial z_i}\frac{\partial z_i}{\partial y_{i+1}} ) \frac{\partial y_L}{\partial z_L} \frac{\partial E}{\partial y_L}
        		    $
        		    
        	    其中:
        		    $
        		        \frac{\partial y_l}{\partial z_l} &= \sigma'(z_l)\\
        		        \frac{\partial z_l}{\partial y_{l+1}} &= \bb w_{l+1}^T
        		    $
        		    
        		令$\delta_l$为每层误差:
        		    $
        		        \delta_l &=(\prod_{i = l}^{L} \frac{\partial y_i}{\partial z_i}\frac{\partial z_i}{\partial y_{i+1}} ) \frac{\partial y_L}{\partial z_L} \frac{\partial E}{\partial y_L}\\
        		        &= (\prod_{i = l}^{L} \sigma'(z_i)\bb w_{i+1}^T) \sigma '(z_L)\frac{\partial E}{\partial y_L}
        		    $
        		    
        		得到每层参数误差结果:
        		    $
        		        \Rightarrow \frac{\partial E}{\partial w_l} &= \delta_l \frac{\partial z_l}{\partial w_l} = \delta_l x_l^T\\
        		        \delta_L &= \frac{\partial E}{\partial y_L} ·\sigma '(z_L)
        		    $
        		    
            \bf{附}:
            
        		激活函数的导函数:
        		    $
        		        relu(x) &= \max(0, x)\\
        		        relu'(x) &= x > 0 ? 1 : 0 = step(x) \quad \text{阶跃函数}\\
        		        sigmoid(x) &= \frac{1}{1+e^{-x}}\\
        		        sigmoid'(x) &= sigmoid(x) · (1 - sigmoid(x))
        		    $