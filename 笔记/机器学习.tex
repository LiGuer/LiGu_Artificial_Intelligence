\documentclass{article} 
\usepackage{amsmath}
\usepackage[UTF8]{ctex}
\title{}\date{} \setlength{\parindent}{0pt} \linespread{1.25}
\usepackage{ntheorem}\usepackage{amssymb}
\usepackage{graphicx}
\usepackage{listings}
\usepackage{geometry} \geometry{a4paper,left=1cm,right=1cm,top=1cm,bottom=1.5cm}
\usepackage{paralist}
\let\itemize\compactitem
\let\enumerate\compactenum

\newcommand{\env}[2]{\begin{#1}#2\end{#1}}
\newcommand{\defi}[2]{\textbf{#1}, #2}
\newcommand{\proof}[1]{\textbf{证明} #1}
\begin{document}

\section{主成分分析}
		[目标]:
			数据降维,提取数据的主要特征分量, 满足:
    			\begin{enumerate}
    			    \item 最近重构性: 样本点到该超平面的距离都足够近。
    			    \item 最大可分性: 样本点在该超平面的投影尽可能分开。
    			\end{enumerate}
			
		[优化问题]:
    		\begin{align*}
    			\min_W  &\quad	tr( W^T x x^T W )\\
    			s.t.    &\quad	W^T W = I
    		\end{align*}
			
		[流程]:
			(1) 数据中心化, \sum \vec x_i = 0
			(2) 计算协方差矩阵 C = X X^T
			(3) 对协方差矩阵C 特征值分解
			(4) 取最大d'个特征值所对应的特征向量{w1,w2,...,wd'},投影矩阵 W = (w1,w2,...,wd')
			(5) 样本点在超平面投影: y_i = W^T x_i
			
		[原理]:
				分别从目标(1, 2)可以推得同样的结果
			(3)	目标函数: 样本点到超空间投影 y = WT x 尽可能分开, 即.方差最大:$\max \sum W^T x x^T W$
				协方差矩阵:
					$D = 1/m Y Y^T = 1/m (PX) (PX)^T = 1/m P X X^T P^T = 1/m P C P^T$
				协方差矩阵对角化
			(4) 优化问题构造:
					min_W		tr( W^T x x^T W )
					s.t.		W^T W = I
			(5) 计算最优点:
				Lagrange函数 $L(W,λ) = W^T x x^T W + λ( W^T W - I )$
				Lagrange对偶 $G(λ) = inf L(W,λ) = inf (W^T x x^T W + λ( W^T W - I ))$
				L(W,λ)求导, 当导数为0时, 取得极值
				=>	$X X^T ω_i = λ_i ω_i$
				即.对协方差X XT, 特征值求解
			()	取特征值最大的yDim个特征向量, 即目标投影矩阵W


\section{K-Mean 聚类}
		[目标]:
			聚类. 对N维分布的数据点,可以将其聚类在 K 个关键簇内.
			
		[步骤]:
			(1) 随机选择 K 个簇心点 Center
			(2) 迭代开始
				(3) 归零 Cluster , Cluster: 簇,记录ith簇内的数据指针。
				(4) 计算每个xi到簇心μj的距离
					(5) 选择距离最小的簇心, 将该点加入其簇内
				(6) 对每个簇,计算其质心 Center'
				(7) Center≠Center' , 则更正Center为 Center'
				(8) 迭代重新开始
			(9) 一轮无更正时,迭代结束

\section{最小二乘法}
		[目标]:
			求一条直线,使得所有样本点到该直线的Euclid距离最小.
			
		[优化问题]: 最小化均方误差
    		\begin{align*}
    			\min_w  &\quad	MSE(\tilde{\boldsymbol y}) · n = \sum (\tilde {\boldsymbol y} -\boldsymbol y)^2 = \sum (\boldsymbol w^T \boldsymbol x - \boldsymbol y)^2
    		\end{align*}
    		
		[原理]:
			(1) 直线方程: $$f(x) = \boldsymbol w^T \boldsymbol x	\quad (\boldsymbol x = [1, \boldsymbol x_0])$$
			(2) 均方误差:
					$$MSE(\tilde y) · n = \sum (\tilde {\boldsymbol y} -\boldsymbol y)^2 = \sum (\boldsymbol w^T \boldsymbol x - \boldsymbol y)^2 = (X w - y)^T (X w - y)$$
					
			(3) 优化问题构造: 无约束凸优化问题
					$$\min_W	\sum (\boldsymbol w^T \boldsymbol x - y)^2 = (X w - y)^T (X w - y)$$
					
			(4) 计算最优点: 求导, 导数为0时取得极值
				\begin{align*}
					\frac{\partial MSE}{\partial w} &= 2·X^T·(\boldsymbol w^T \boldsymbol x - y) = 0\\
				\Rightarrow	w^* &= (X^T X)^-1 X^T y\\
					f(x) &= x^T (X^T X)^-1 X^T y
				\end{align*}
				
			(5) 一维场景:
				优化问题:
				\begin{align*}
					min_{w,b}	\sum (\boldsymbol w^T \boldsymbol x_i + b - y_i)^2\\
				    \Rightarrow	\frac{\partial E}{\partial w} = 2( w^T \sumx_i^2 + \sum(x_i(b - y_i)) )	= 0\\
					\frac{\partial E}{\partial b} = 2( n b + \sum(y_i - w x_i) ) = 0
				\end{align*}
				
				最优点:
				\begin{align*}
				    \Rightarrow w \sum x_i^2 &= \sum_i x_i y_i - \frac{1}{n} \left(\sum_i x_i \right) \left(\sum_i y_i\right) + \frac{w}{n} \left(\sum_i x_i\right)^2\\
					w^* &= \frac{\sum y_i(x_i - \bar x)}{\sum x_i^2 - \frac{1}{n} (\sum x_i)^2}\\
					b^* &= \frac{1}{n}·\sum(y_i - w x_i)\\
				\end{align*}

					
\section{核方法}
    \defi{核方法}{将样本从原始空间映射到更高维特征空间,使得其线性可分.}
        \[\mathbf{K}=\left[\begin{array}{ccccc}
            \kappa\left(\boldsymbol{x}_{1}, \boldsymbol{x}_{1}\right) & \cdots & \kappa\left(\boldsymbol{x}_{1}, \boldsymbol{x}_{j}\right) & \cdots & \kappa\left(\boldsymbol{x}_{1}, \boldsymbol{x}_{m}\right) \\
            \vdots & \ddots & \vdots & \ddots & \vdots \\
            \kappa\left(\boldsymbol{x}_{i}, \boldsymbol{x}_{1}\right) & \cdots & \kappa\left(\boldsymbol{x}_{i}, \boldsymbol{x}_{j}\right) & \cdots & \kappa\left(\boldsymbol{x}_{i}, \boldsymbol{x}_{m}\right) \\
            \vdots & \ddots & \vdots & \ddots & \vdots \\
            \kappa\left(\boldsymbol{x}_{m}, \boldsymbol{x}_{1}\right) & \cdots & \kappa\left(\boldsymbol{x}_{m}, \boldsymbol{x}_{j}\right) & \cdots & \kappa\left(\boldsymbol{x}_{m}, \boldsymbol{x}_{m}\right)
        \end{array}\right]\]
        
        \[\begin{array}{l}
            \kappa\left(\boldsymbol{x}_{i}, \boldsymbol{x}_{j}\right)=\boldsymbol{x}_{i}^{\mathrm{T}} \boldsymbol{x}_{j} \\
            \kappa\left(\boldsymbol{x}_{i}, \boldsymbol{x}_{j}\right)=\left(\boldsymbol{x}_{i}^{\mathrm{T}} \boldsymbol{x}_{j}\right)^{d} \\
            \kappa\left(\boldsymbol{x}_{i}, \boldsymbol{x}_{j}\right)=\exp \left(-\frac{\left\|\boldsymbol{x}_{i}-\boldsymbol{x}_{j}\right\|^{2}}{2 \sigma^{2}}\right) \\
            \kappa\left(\boldsymbol{x}_{i}, \boldsymbol{x}_{j}\right)=\exp \left(-\frac{\left\|\boldsymbol{x}_{i}-\boldsymbol{x}_{j}\right\|}{\sigma}\right) \\
            \kappa\left(\boldsymbol{x}_{i}, \boldsymbol{x}_{j}\right)=\tanh \left(\beta \boldsymbol{x}_{i}^{\mathrm{T}} \boldsymbol{x}_{j}+\theta\right)
        \end{array}\]
		

\section{支持向量机}
    \defi{支持向量机}{
        是一种二分类模型,目的是找到一个超平面, 使得超平面与任意样本点之间的最小距离最大. 约束条件是超平面能够使得所有样本点正确分类. 支持向量机的优化问题形式(凸二次规划问题)如下. 
        \[\begin{align*}
           \underset{\boldsymbol w, b}{\arg\max} &\quad \min\limits_i d_i = \frac{\min\limits_i\ |\boldsymbol w^T \boldsymbol x_i + b |}{||\boldsymbol w||}\\
           s.t. &\quad y_i (\boldsymbol w^T \boldsymbol x_i + b) \le 1
        \end{align*}
        \qquad \Rightarrow \qquad 
	    \begin{align*}
	        \underset{\boldsymbol w, b}{\arg\min}     &\quad	\frac{||\boldsymbol w||^2}{2}\\
			s.t.    &\quad	y_i (\boldsymbol w^T \boldsymbol x_i + b) \le 1
	    \end{align*}\]
    }
    
    \proof{	\par		
        超平面方程: $\boldsymbol w^T \boldsymbol x + b = 0$, \quad 点面距: $d = \frac{|\boldsymbol w^T \boldsymbol x + b|}{||w||}$
		
		分类
			\begin{align*}\left \{ \begin{array}{cc}
				\boldsymbol w^T \boldsymbol x_i + b \ge +1    &\quad (y_i = +1)\\
				\boldsymbol w^T \boldsymbol x_i + b \le -1	&\quad (y_i = -1)\\
			    \end{array} \right.
				\quad\Rightarrow\quad y_i (\boldsymbol w^T \boldsymbol x_i + b) \ge 1
			\end{align*}

		间隔: 离超平面最近的2个异类样本点到超平面的距离之和.
			\[\Delta = \frac{2}{||w||} \tag{间隔}\]
					
		优化问题构造: 使间隔最大化. 凸二次规划问题
		    \[\max\quad  \frac{2}{||\boldsymbol w||} \quad \Rightarrow \quad \min\quad  \frac{||\boldsymbol w||^2}{2}\]
			\[s.t.\quad	y_i (\boldsymbol w^T \boldsymbol x_i + b) \ge 1\]
	}
    
	\textbf{解}, 解凸优化问题, 计算最优点: 先求Lagrange函数, Lagrange函数对$\boldsymbol w, b$求导, 令导数为0取得极值$\boldsymbol w^*, b^*$, 再求Lagrange对偶函数
		\[L(\boldsymbol w, b,\boldsymbol \lambda) = \frac{||\boldsymbol w||^2}{2} + \sum_i \lambda_i (1 - y_i (\boldsymbol w^T \boldsymbol x_i + b)) \tag{Lagrange函数}\]
		\[\Rightarrow \left\{\begin{array}{rl} \boldsymbol w^* &= \sum\limits_i \lambda_i y_i x_i\\ 0 &= \sum\limits_i \lambda_i y_i \end{array}\right.\tag{$\boldsymbol w, b$极值}\]
		\[G(\boldsymbol \lambda) = L(\boldsymbol w^*, b^*, \boldsymbol \lambda) = \sum \lambda_i - \frac{1}{2} \sum_i \sum_j \lambda_i \lambda_j y_i y_j \boldsymbol x_i^T \boldsymbol x_j \tag{Lagrange对偶}\]

		得到对偶问题, 为二次规划问题形式, 可利用Sequential Minimal Optimization算法求解得到$\boldsymbol \lambda^*$
		    \begin{align*}
		        \max\limits_{\boldsymbol \lambda} &\quad G(\boldsymbol \lambda)\\
			    s.t. &  \quad \lambda_i \ge 0\\
			        & \quad \sum_i \lambda_i y_i = 0
		    \end{align*}
		    
		其中, KKT条件如下
		    \begin{align*}
		        \lambda &\ge 0\\
		        y_i (\boldsymbol w^T \boldsymbol x_i + b) - 1 &\ge 0\\
		        \lambda_i(y_i(\boldsymbol w^T \boldsymbol x_i + b) - 1) &= 0
		    \end{align*}
					
	\textbf{算法步骤}:
        \begin{enumerate}
            \item 计算核矩阵
            \item 计算$\lambda^*$
            \begin{enumerate}
                \item 选择 i
                \item 选择 j
                \item 计算 $K_{ii}+K_{jj}-2K_{ij}, L, H$
                \item 更新 $\lambda_j, \lambda_i$
                \item 更新 b
            \end{enumerate}
            \item 计算 w*, b*
        \end{enumerate}

    \textbf{核方法非线性支持向量机}:
		\textbf{核函数}:将样本从原始空间映射到更高维特征空间,使得其线性可分.
			=>	超平面方程: $w^T \phi(x) + b = 0$
				min		||w||^2 / 2
				s.t.	y_i (w^T Φ(x_i) + b) ≥ 1
				G(λ) = \sumλ_i - 1/2 \sum_i \sum_j λ_i λ_j y_i y_j Φ(x_i)^T Φ(x_j)
			=>	设 核函数к(x_i, x_j) = Φ(x_i)^T Φ(x_j)
				f(x) = w*^T x + b = \sum λ_i y_i к(x, x_i) + b
			*	к是核函数 <=> 核矩阵[a_ij = к(x_i, x_j)]总是半正定
			
		\textbf{Sequential Minimal Optimization算法}
			∵	λ是n-1自由度, 确定前n-1个量, 则第n个由\sum λ_i y_i = 0自动确定.
			∴	每次选2个λ_i λ_j, 固定其他λ_k不变, 优化λ_i λ_j, 更新b
			优化λ_i λ_j:
			=>	λ_i = (-\sum_{k≠i≠j} λ_k y_k)·y_1 - λ_j y_i y_j = ζ y_1 - λ_j y_i y_j
				G(λ_j) = (λ_j + ζ y_1 - λ_j y_i y_j) + C - v_i (ζ-λ_j y_j) - v_j λ_j y_j
						- 1/2 к_ii (ζ-λ_j y_j)^2 - 1/2 к_jj λ_j^2 - к_ij λ_j y_j (ζ-λ_j y_j)^2
				其中 v_i = \sum_{k≠i≠j} λ_i y_i к_ki = f(x_i) - λ_i y_i к_ii -λ_j y_j к_ij - b, v_j同理
				∂G/∂λ_j = -(к_ii+к_jj-2к_ij)(λ_j^old - λ_j^new) + y_j(y_j-y_i+f(x_i)-f(x_j)) = 0
			=>	λ_j^new = λ_j^old + y_j (E_i - E_j)/(к_ii+к_jj-2к_ij)		其中E_i = f(x_i) - y_i
			修剪: 使得λ_j* 满足约束条件, λ_j应当∈[L,H]:
				y_i≠y_j时,	下界L = max(0,λ_j^old-λ_i^old)	,上界L = min(C,λ_j^old-λ_i^old + C)
				y_i= y_j时,	下界L = max(0,λ_j^old+λ_i^old - C),上界L = min(C,λ_j^old+λ_i^old)
			更新b:
			
	\textbf{软间隔}:

\section{神经网络}
    \subsection{前馈神经网络}
        \subsubsection{正向传播}
            $\sigma()$: 激活函数, 使线性拟合非线性化, eg. relu(x), Sigmoid(x)
            $$\boldsymbol y = \sigma (\boldsymbol w \boldsymbol x + \boldsymbol b)$$


            误差·损失函数: $E_{total} = \sum (target_i - out_i)^2$
            $$E(\boldsymbol{w})=\frac{1}{2} \sum_{n=1}^{N}\left\|\boldsymbol{y}\left(\boldsymbol{x}_{n}, \boldsymbol{w}\right)-\boldsymbol{t}_{n}\right\|^{2}$$

        \subsubsection{反向传播 --- 梯度下降最优化}
    		$$\boldsymbol{w}^{(\tau+1)}=\boldsymbol{w}^{(\tau)}-\eta \nabla E\left(\boldsymbol{w}^{(\tau)}\right)$$
    		
    		设$z = \boldsymbol w^T \boldsymbol x + \boldsymbol b$
            $$\left \{ \begin{array}{lr}
                \delta_l = (\boldsymbol w_{l+1}^T·\delta_{l+1})·\sigma'(z_l) & \text{(每层误差)}\\
                \delta_L = \nabla E · \sigma'(z_{outl}) & \text{(输出层误差)}\\
                \nabla E(\boldsymbol w_l) = \delta_l \boldsymbol x_l^T & \text{(每层参数误差)}
            \end{array} \right.$$

            \textbf{证明}:
        		
        		每层参数误差, 由链式法则得(其中$L$指输出层): 
        		    \begin{align*}
        		        \frac{\partial E}{\partial w_l} 
        		        &= \frac{\partial z_l}{\partial w_l} \left(\frac{\partial y_l}{\partial z_l}\frac{\partial z_l}{\partial y_{l+1}}\right) ... \left(\frac{\partial y_{L-1}}{\partial z_{L-1}} \frac{\partial z_{L-1}}{\partial y_L} \right) \frac{\partial y_L}{\partial z_L} \frac{\partial E}{\partial y_L}\\
        		        &= \frac{\partial z_l}{\partial w_l}  \left(\prod_{i = l}^{L} \frac{\partial y_i}{\partial z_i}\frac{\partial z_i}{\partial y_{i+1}} \right) \frac{\partial y_L}{\partial z_L} \frac{\partial E}{\partial y_L}
        		    \end{align*}
        		    
        	    其中:
        		    \begin{align*}
        		        \frac{\partial y_l}{\partial z_l} &= \sigma'(z_l)\\
        		        \frac{\partial z_l}{\partial y_{l+1}} &= \boldsymbol w_{l+1}^T
        		    \end{align*}
        		    
        		令$\delta_l$为每层误差:
        		    \begin{align*}
        		        \delta_l &=\left(\prod_{i = l}^{L} \frac{\partial y_i}{\partial z_i}\frac{\partial z_i}{\partial y_{i+1}} \right) \frac{\partial y_L}{\partial z_L} \frac{\partial E}{\partial y_L}\\
        		        &= \left(\prod_{i = l}^{L} \sigma'(z_i)\boldsymbol w_{i+1}^T\right) \sigma '(z_L)\frac{\partial E}{\partial y_L}
        		    \end{align*}
        		    
        		得到每层参数误差结果:
        		    \begin{align*}
        		        \Rightarrow \frac{\partial E}{\partial w_l} &= \delta_l \frac{\partial z_l}{\partial w_l} = \delta_l x_l^T\\
        		        \delta_L &= \frac{\partial E}{\partial y_L} ·\sigma '(z_L)
        		    \end{align*}
        		    
            \textbf{附}:
            
        		激活函数的导函数:
        		    \begin{align*}
        		        relu(x) &= \max(0, x)\\
        		        relu'(x) &= x > 0 ? 1 : 0 = step(x) \quad \text{阶跃函数}\\
        		        sigmoid(x) &= \frac{1}{1+e^{-x}}\\
        		        sigmoid'(x) &= sigmoid(x) · (1 - sigmoid(x))
        		    \end{align*}

 
\end{document}