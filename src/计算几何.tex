\documentclass{article} 
\usepackage{amsmath}
\usepackage[UTF8]{ctex}
\title{计算几何}\date{} \linespread{1.25}
\usepackage{ntheorem}\usepackage{amssymb}
\usepackage{graphicx}
\usepackage{geometry} \geometry{a4paper,left=2cm,right=2cm,top=1cm,bottom=1.5cm}
\begin{document}
\tableofcontents

\section{计算几何}

\section{几何光学}
    \subsection{基本定律}
        \begin{enumerate}
            \item 光沿直线传播.
            \item 光路可逆.
            \item 两束光传播中互不干扰,会聚于同一点时光强简单相加.
            \item 介质分界面的光传播规律.
        \end{enumerate}
    
    
    \subsection{反射}
        \textbf{反射定律}: 入射角 = 反射角. $\theta_I = \theta_O$
        
        设面矢\boldsymbol F, 入射光L
			$$\boldsymbol L_O = \boldsymbol L_I - \boldsymbol F·2 \cos \theta_I$$
			
    \subsection{折射}
        \textbf{折射定律}: $n_I·\sin \theta_I = n_O·\sin \theta_O$
            
			设面矢\boldsymbol F, 入射光L $n = \frac{n_I}{n_O}, \theta_I = <\boldsymbol L,\boldsymbol F>$
			\begin{align*}
    			\boldsymbol L_O &= \boldsymbol L_I + \boldsymbol F·\frac{\sin(\theta_O - \theta_I)}{\sin\theta_O}\\
    			   &= \boldsymbol L_I + \boldsymbol F· \frac{\cos\theta_I·\sin\theta_O - \cos\theta_O·\sin\theta_I}{\sin\theta_O}\\
    			   &= \boldsymbol L_I + \boldsymbol F· \left(\cos\theta_I - \frac{\cos\theta_O}{n}\right)\\
    			   &= \boldsymbol L_I + \boldsymbol F· \left( \cos\theta_I - \frac{\sqrt{1 - n^2·\sin^2\theta_I}}{n} \right)
			\end{align*}
	

\section{外接圆}
	\subsection{外接圆半径}
		\textbf{Heron-秦九韶公式}
			$$S=\sqrt{p(p-a)(p-b)(p-c)}$$
			$$p = \frac{a+b+c}{2}$$

	\subsection{外接圆圆心}


	isInCircle 判断四点共圆
	*	[输出]: 圆外-1,圆上0,圆内1
	*	三点确定圆方程: 即 解行列式:
			| x1²+y1²  x1  y1  1 | ?= 0
			| x2²+y2²  x2  y2  1 |
			| x3²+y3²  x3  y3  1 |
			| x4²+y4²  x4  y4  1 |
	*	[几何解释]: 通过把平面点提升到三维的抛物面中,由于抛物面被平面所截的截面为圆形,四点共面即使共圆,也可以用四面体的体积判断是否共圆。


	ThreePointsToCircle 平面三点确定圆方程
	*	[公式]: 圆方程: (x - cx)² + (y - cy)² = R²
	*	[算法]: 三点确定圆方程: 即 解行列式:
				| x²+y²    x   y   1 |  =  0
				| x1²+y1²  x1  y1  1 |
				| x2²+y2²  x2  y2  1 |
				| x3²+y3²  x3  y3  1 |
			即.目标三点和圆上(x,y)应该满足方程组:
				(x²+y²)·a + x·b + y·c + 1·d = 0
	*	[推导]:
				M11(x²+y²) - M12 x + M13 y - M14 = (x²+y²)·a + x·b + y·c + 1·d = 0
				=> (x² + b/a x) + (y² + c/a y) = - d/a
				=> (x + b/2a)² + (y + c/2a)² = -d/a + b²/4a² + c²/4a²
								CircumCircle 三角形外接圆
	*	外接圆圆心: 即. 三点确定圆方程问题, 也是任意两边的垂直平分线的交点.直接用 ThreePointsToCircle()方法

            
\section{交点}
    \subsection{射线 \& 平面交点}
		平面方程: $Ax + By + Cz + D = 0$
		直线方程: $(x - x0) / a = (y - y0) / b = (z - z0) / c = K$
		联立平面,直线方程, 解K, K即直线到射线平面交点的距离
		\begin{align*}
			x = K a + x0 , ...
			A(K a + x0) + B(K b + y0) + C(K c + z0) + D = 0
			=> K =  - (A x0 + B y0 + C z0) / (A a + B b + C c)
		\end{align*}
		
	\subsection{射线 \& 球面交点}
		球方程: $(X - Xs)² + (Y - Ys)² + (Z - Zs)² = R²$
		线球交点: $K²(Al² + Bl² + Cl²) + 2K(Al ΔX + Bl ΔY + Cl ΔZ) + (ΔX² + ΔY² + ΔZ² - R²) = 0$
		\begin{align*}
				  ΔX = Xl - Xs
				  Δ = b² - 4ac = 4(Al ΔX + Bl ΔY + Cl ΔZ)² - 4(Al² + Bl² + Cl²)(ΔX² + ΔY² + ΔZ² - R²)
				  若Δ≥0 有交点.
				  K = ( -b ± sqrt(Δ) ) / 2a	即光线走过线距离
		\end{align*}
		
    \subsection{射线 \& 三角形交点}
		射线: $P = O + t D$
		射线三角形交点:
		\begin{align*}
				O + t D = (1 - u - v)V0 + u V1 + v V2
				[ -D  V1-V0  V2-V0] [ t  u  v ]' = O - V0
				T = O - V0    E1 = V1 - V0    E2 = V2 - V0
				[ -D  E1  E2 ] [ t  u  v ]' = T
				t = | T  E1  E2| / |-D  E1  E2|
				u = |-D   T  E2| / |-D  E1  E2|
				v = |-D  E1  E2| / |-D  E1  E2|
		(混合积公式): |a  b  c| = a×b·c = -a×c·b
				t = (T×E1·E2) / (D×E2·E1)
				u = (D×E2· T) / (D×E2·E1)
				v = (T×E1· D) / (D×E2·E1)
        \end{align*}


						ConvexHull 凸包
*	[算法]: Graham 扫描法
*	[时间复杂度]: O(n logn)
*	[流程]:
		[1] 选择y最小点 p0, 若多个则选其中x最小
		[2] sorted by polar angle in counterclockwise order around p0
			(if more than one point has the same angle, remove all but the one that is farthest from p0)
			* 几何可知,排序后 P1 和最后一点 Pn-1 一定是凸包上的点
		[3] P0,P1 入栈S,P2 为当前点
			if n<2, error "Convex Hull is empty"
		[4] 遍历剩余点 P3 -> Pn-1
			[4.1] while the angle formed by points NEXT-TO-TOP(S),TOP(S),and p makes a nonleft turn
					POP(S)
			[4.2] Pi 入栈
		[5] 最后栈中元素,即结果
*	[Referance]:
		[1] Introduction Algorithms.THOMAS H.CORMEN,CHARLES E.LEISERSON,RONALD L.RIVEST,CLIFFORD STEIN
		[2] Thanks for https://www.cnblogs.com/aiguona/p/7232243.html		


						Delaunay 三角剖分
*	[定义]:
		[1] Delaunay三角剖分: 每个三角形的外接圆内不包含V中任何点
	[流程]:
		[1] 将点按坐标x从小到大排序
		[2] 确定超级三角形
			将超级三角形保存至未确定三角形列表 trianglesTemp
		[3] 遍历每一个点
			[3.1] 初始化边缓存数组 edgeBuffer
			[3.2] 遍历 trianglesTemp 中的每一个三角形
				[3.2.1] 计算该三角形的圆心和半径
				[3.2.2] 如果该点在外接圆的右侧
					则该三角形为Delaunay三角形,保存到triangles,并在temp里去除掉,跳过
				[3.2.3] 如果该点在外接圆外(即也不是外接圆右侧)
					则该三角形为不确定,跳过
				[3.2.4] 如果该点在外接圆内
					则该三角形不为Delaunay三角形,将三边保存至edgeBuffer,在temp中去除掉该三角形
			[3.3] 对edgeBuffer进行去重
			[3.4] 将edgeBuffer中的边与当前的点进行组合成若干三角形并保存至temp triangles中
		[4] 将triangles与trianglesTemp进行合并, 并除去与超级三角形有关的三角形
*	[Referance]:
		[1] http://paulbourke.net/papers/triangulate/




\section{分形}
Mandelbrot集
*	[定义]: Zn+1 = Zn² + C
			所有能使Zn+1不发散的复数点C, 所构成的集合,即 Mandelbrot Set
			(不发散,不一定收敛,有可能在几个不同点来回跳)
*	[性质]: |Zn|>2不可能收敛, 即Mandelbrot Set在半径为2的圆内.


*								Julia集
*	[定义]: Zn+1 = Zn² + C
			对于某复数值C,所有能使Zn+1不发散的Z0的集合,即 Julia Set
			类似于. Mandelbrot 曼德布洛特集


*								Hilbert 曲线
*	[定义]: 一种自相似的分形曲线
*	[生成方法]: 四象限复制四分,翻转左下、右下, 使左下末同左上初、右下初同右上末,能够最短连接.
*	[性质]:
		* 边长: nth 2^n
		* 长度: nth 2^n - 1 / 2^n
		* 因为是四进制自相似, 所以曲线上位置 distance, 可以不断判断子象限,按二进制在位上叠加
*	[用途]: 1D to 2D 的映射算法, 随 n 增加, 映射的位置趋于收敛
*	[1,2阶曲线]: "┌┐"	   "┌┐┌┐"    "┌┐┌┐ "
					︱︱  =>	︱︱︱︱ =>   ︱└┘︱
							┌┐┌┐      └┐┌┘
							︱︱︱︱       -┘└-
*	[程序]: rotation : 翻转坐标, 使Hilbert曲线性质, 自相似地适用于左下、右下象限
		[1] xy2d(): 坐标 -> 曲线位置    [2] d2xy(): 曲线位置 -> 坐标
*	[Author]: 1891.Hilbert
*	[Reference]: [1] Wikipedia.Hilbert curve


*								Perlin Noise
*	Function to linearly interpolate between a0 and a1 , Weight w should be in the range [0.0, 1.0]
*	[流程]:
		[1] 格点随机梯度矢量
		[2] (x,y)与格点距离,梯度点积
		[3] 插值

		

Marching Cubes 三维等高面绘制


\end{document}