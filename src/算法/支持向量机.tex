* 支持向量机
	\Algorithm
		- 目的
			$
				\max_{\.w, b} \quad& \min_i d_i = \frac{1}{||\.w||} \min_i |\.w^T \.x_i + b|
				s.t. \quad& y_i (\.w^T \.x_i + b) ≤ 1
			$
			找到一个超平面, 最大化样本点与超平面的最小距离, 且使得不同类别$y_i \in \{-1,1\}$的点分居超平面的两侧. 
		- 解法
			$
				\max_{\.\lambda} \quad& \sum \lambda_i - \frac{1}{2} \sum_i \sum_j \lambda_i \lambda_j y_i y_j \.x_i^T \.x_j  \tag{对偶问题}
				s.t. \quad& \sum_i \lambda_i^* y_i = 0
					& \lambda_i &≥ 0
			$
			利用Sequential Minimal Optimization算法求解$\lambda_i^*$, 代入解出$\.w^*, b^*$.
			\Proof
				优化问题简化, 可令$\min_i |\.w^T \.x_i + b| = 1$, 且$ \arg\max_{\.w, b} \frac{1}{||\.w||} => \arg\min_{\.w, b} \frac{||\.w||^2}{2}$
				$
					\max_{\.w, b} \quad& \frac{||\.w||^2}{2}
					s.t. \quad&	y_i (\.w^T \.x_i + b) ≤ 1
				$
				Lagrange函数,
				$
					L(\.w, b,\.\lambda) &= \frac{||\.w||^2}{2} + \sum_i \lambda_i (1 - y_i (\.w^T \.x_i + b)) \tag{Lagrange函数}
					\frac{∂ L(\.w, b,\.\lambda)}{∂ \.w} |_{\.w = \.w^*} &= \.w^* - \sum_i \lambda_i x_i y_i = 0
					\frac{∂ L(\.w, b,\.\lambda)}{∂ b} |_{b = b^*} &= - \sum_i \lambda_i y_i = 0
				$
				$
					=>\quad& \.w^* = \sum_i \lambda_i y_i x_i
					& \sum_i \lambda_i^* y_i = 0
				$
				Lagrange对偶函数, 对偶问题,
				$
					G(\.\lambda) &= \inf_{\.w, b} L(\.w, b, \.\lambda)  \tag{Lagrange对偶函数}
						&= L(\.w^*, b^*, \.\lambda)
						&= \frac{1}{2} (\sum_i \lambda_i y_i x_i)^T (\sum_j \lambda_j y_j x_j) + (-\sum_i \lambda_i y_i (\sum_j \lambda_j y_j x_j)^T x_i + \sum_i \lambda_i)  \tag{$\.w^*, b^*$代入}
						&= \sum \lambda_i - \frac{1}{2} \sum_i \sum_j \lambda_i \lambda_j y_i y_j \.x_i^T \.x_j
				$
				$
					\max_{\.\lambda} \quad& \sum \lambda_i - \frac{1}{2} \sum_i \sum_j \lambda_i \lambda_j y_i y_j \.x_i^T \.x_j  \tag{对偶问题}
					s.t. \quad& \sum_i \lambda_i^* y_i = 0
						& \lambda_i &≥ 0
				$
				对偶问题是凸二次规划问题, 可利用Sequential Minimal Optimization算法求解$\lambda_i^*$, 代入解出$\.w^*, b^*$.
				KKT条件,
				$
					y_i (\.w^{*T} \.x_i + b^*) &≤ 1
					\lambda_i^* &≥ 0
					\lambda_i^* (1 - y_i (\.w^{*T} \.x_i + b^*)) = 0
					\.w^* &= \sum_i \lambda_i^* x_i y_i
					\sum_i \lambda_i^* y_i &= 0
				$
* 核方法
	* 核函数
		\def
			$\kappa(\.x_i, \.x_j) = \phi(\.x_i)^T \phi(\.x_j)$
		\Example
			- 线性核: $\kappa(\.x_i, \.x_j) = \.x_i^T \.x_j$
			- 高斯核: $\kappa(\.x_i, \.x_j) = e^{-\frac{1}{2}(\.x_i - \.x_j)^T \Sigma^{-1} (\.x_i - \.x_j)}$
			- 多项式核: $\kappa(\.x_i, \.x_j) = (\.x_i^T \.x_j)^d$
			- Laplace核: $\kappa(\.x_i, \.x_j) = e^{-\frac{\|\.x_i-\.x_j\|}{\sigma}}$
			- Sigmoid核: $\kappa(\.x_i, \.x_j) = \tanh (\beta \.x_i^T \.x_j + \theta)$

