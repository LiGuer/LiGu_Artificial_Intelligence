* 优化问题
	\def
		$
			\min \quad& f_0(x)  \tag{目标函数}
			s.t. \quad& f_i(x) ≤ 0  \tag{不等式约束}
						& h_i(x) = 0  \tag{等式约束}
		$
		优化问题的最优解: $p^*= \inf \{f_{0}(x) | f_i(x) ≤ 0, h_i(x) = 0 \}$
	\Example
		* 可行性问题
			\def
				$
					\min \quad& x 
					s.t. \quad& f_i(x) ≤ 0
								& h_i(x) = 0
				$
				若目标函数恒等于零, 则最优解是0 (可行集非空) 或$\infty$ (可行集为空集).
	* 凸优化问题
		\def
			$
				\min \quad& f_0(x) \tag{$f_0$为凸} 
				s.t. \quad& f_i(x) ≤ 0 \tag{$f_i$为凸} 
							& a_i^T x = b_i  \tag{仿射函数} 
			$
		\Example
			* 线性规划
				\def
					$
						\min \quad& c^T x + d
						s.t. \quad& G x ⪯ h
							& A x = b
					$
					目标函数和约束函数都是仿射的优化问题.
					* 标准形式线性规划
						$
							\min \quad& c^T x
							s.t. \quad& A x = b
								& x ⪰ 0
						$
					* 不等式形式线性规划
						$
							\min \quad& c^T x
							s.t. \quad& A x ⪯ b
						$
				\Note
					- 可行域是多面体, 等位曲线是与向量$c^T$正交的超平面, 最优解是多面体中在$-c^T$方向最远的顶点.
					- 若线性规划问题存在两个最优解, 则其必然存在无穷多个最优解. 
				\Algorithm{线性规划转换为标准形式}
					- 步骤
						- 为不等式引入松弛变量
							$
								\min \quad& c^T x + d
								s.t. \quad& G x + s = h
									& A x = b
									& s ⪰ 0
							$
						- 将x表示为两个非负变量的差$x = x^+ - x^-, x^+ ⪰ 0, x^- ⪰ 0$, 代入即可转换为标准形式.
							$
								\min \quad& c^T x^+ - c^T x^- + d
								s.t. \quad& G x^+ - G x^- + s = h
									& A x^+ - A x^- = b
									& s ⪰ 0, x^+ ⪰ 0, x^- ⪰ 0
							$
			* 线性分式规划
				\def
					$
						\min \quad& \frac{a^T x+ b}{c^T x + d}
						s.t. \quad& Gx ⪯ 0
							& Ax = b
					$
					该问题可以等价转化为线性规划.
			* 二次规划
				\def
					$
						\min \quad& \frac{1}{2} x^T P x + q^T x + r
						s.t. \quad& Gx ⪯ 0
							& Ax = b
					$
				\Example
					最小二乘法 $\min ||Ax + b||_2^2$
			* 二次约束二次规划
				\def
					$
						\min \quad& \frac{1}{2} x^T P_0 x + q_0^T x + r_0
						s.t. \quad& \frac{1}{2} x^T P_i x + q_i^T x + r_i ⪯ 0
							& Ax = b
					$
			* 二次锥规划
				\def
					$
						\min \quad& f^T x
						s.t. \quad& ||A_i x + b_i|| ≤ c_i^T + d_i
							& Fx = g
					$
			* 几何规划
				\def
					$
						\min \quad& f_0(x)
						s.t. \quad& f_i(x) ≤ 1
							& h_i(x) = 1
					$
					自然形式不是凸的, 但可通过变换转换为凸优化问题.
			* 半正定规划
			\Note
				$二次锥规划 \supset \{二次规划 \supset \{ 线性规划 \} , 二次约束二次规划\}$
		* 凸优化问题算法
			* 线性规划求解
				* 单纯形法
					* 大M法
			* 无约束凸优化问题求解
				* 下降法
					* 梯度下降法
						$x_{k+1} = x_k - \lambda ∇ f(x_k)$
					* 最速下降法
						$x_{k+1} = x_k + \arg\min \{ (∇ f(x_k))^T v\ |\ ||v|| = 1 \}$
					* 牛顿迭代法
						$x_{k+1} = x_k - \frac{1}{∇^2 f(x)} ∇ f(x_k)$
					* 拟牛顿法
			* 等式约束凸优化问题求解
			* 等式加不等式约束优化问题求解
				* 内点法
					* 障碍函数法
					* 原始对偶法
	* 整数规划
		\def
			优化问题中存在变量只能取整数的规划.
		* 混合整数规划
			\def
				优化问题中既有连续变量, 又有整数变量的规划.
		* 0-1规划
			\def
				优化问题中变量仅取值0或1的规划.
		\Algorithm{分枝定界法}
			- 目的
				求解纯整数或混合的整数规划问题.
	* 非凸优化问题