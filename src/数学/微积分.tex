* 微积分
	* 极限
		* 数列极限
			\def{数列极限}
				设$\{x_n\}$为一数列,如果存在常数a,对于任意给定的正数ε (不论它多么小),总存在正整数N,使得当n>N时,不等式$|x_n-a|<\varepsilon$都成立,那么就称常数a是数列$\{x_n\}$的极限, 或者称数列$\{x_n\}$收敛于a,记为$\lim{n \to \infty} x_n = a$. 若不存在常数a, 则数列无极限, 数列\bf{发散}.
				$$\lim_{n \to \infty} x_n=a <=> \forall \varepsilon>0, \exists \text { 正整数 } N \text {, 当 } n>N \text { 时, 有 }|x_n-a|<\varepsilon$$

			\bf{性质}: 
				* 极限唯一性
				* 收敛数列有界性
				* 收敛数列保号性

		* 函数极限
			\def{函数极限}
				设函数$f(x)$在点$x_0$的某一去心邻域内有定义. 如果存在常数$A$ , 对 于任意给定的正数$\varepsilon$(不论它多么小 ), 总存在正数$\delta$ , 使得当$x$满足不等式$0<|x-x_0|<\delta$时, 对应的函数值$f(x)$都满足不等式$|f(x)-A|<\varepsilon$, 那么常数$A$就叫做函数$f(x)$当$x \to x_0$时的极限, 记作$\lim_{x \to x_0} f(x)=A \text { 或 } f(x) arrow A(\text { 当 } x \to x_0) .$

	* 微分
		* 导数 -- 一元函数微分
			\def{导数}
				$\frac{df}{dx} = lim_{x \to 0}  \frac{ f(x + Δx) - f(x - Δx) }{ 2 Δx }$
				$\frac{d^n f}{dx^n} = lim_{x \to 0} \frac{ f^{n - 1}(x + Δx) - f^{n - 1}(x - Δx)} { 2 Δx }$

			\bf{算法}: 中心差分公式
				f'(x) = ( f(x+Δx) -  f(x-Δx) ) / (2 Δx) + Err(f,Δx)
				* 截断误差: Err(f,Δx) = h^2  f^(3)(c) / 6 = O(h^2 )
				* 精度: O(h^2 )
				f'(x) = ( -f(x+2Δx) + 8·f(x+Δx) - 8·f(x-Δx) + f(x-2Δx) ) / (12 Δx) + Err(f,Δx)
				* 截断误差: Err(f,Δx) = h^4 f^(5)(c) / 6 = O(h^4)
				* 精度: O(h^4)

			* 曲率
				\def{曲率}
					单位弧段弯曲的角度, 也即等效曲率圆的半径的倒数.
					$ K = |\frac{Δ\alpha}{Δs}| = 1 / R = \frac{ |y''|}{ (1 + y'^2)^{3/2}} $

		* 偏导数 -- 多元函数微分
			\def{偏导数}
				$ ∂f/∂x_i = [ f(...,x_i + Δx_i,...) -  f(..., x_i - Δx_i, ...) ] / (2 Δx_i)$
				$∂^2 f/∂x_i^2  = [ f'(..,xi+Δxi,..) -  f'(..,xi-Δ,..) ] / Δxi = f(..,xi+Δxi) - 2·f'(..,x) + f'(..,xi-Δxi) / Δxi^2 $
						

			* 梯度, 散度, 旋度
				\def{梯度} $\nabla f $ 矢量, 函数在该点处变化率最大的方向.
				\def{散度} $\nabla · \. f $ 标量, 矢量场在该点发散的程度, 表征场的有源性(>0源,<0汇,=0无源)
				\def{旋度} $\nabla × \. f $ 矢量, 矢量场在该点旋转的程度, 方向是旋转度最大的环量的旋转轴, 旋转的方向满足右手定则, 大小是绕该旋转轴旋转的环量与旋转路径围成的面元面积之比.

				\Property
					(直角坐标系)
					$\nablaf		= ∂f/∂x \. x + ∂f/∂y \. y + ∂f/∂z \. z + ...$
					$\nabla · \. f =  ∂fx/∂x + ∂fy/∂y + ∂fz/∂z + ...$
					$\nabla × \. f = (∂fz/∂y - ∂fy/∂z) \. x
							+ (∂fx/∂z - ∂fz/∂x) \. y
							+ (∂fy/∂x - ∂fx/∂y) \. z$

			* Hamilton算子, Laplace算子
				\def{Hamilton算子} 
					$\nabla \equiv ∂/∂x \. x + ∂/∂y \. y + ∂/∂z \. z + ... $
					Hamilton算子可以看成一个矢量.

				\def{Laplace算子} 
					$\nabla^2 \equiv ∂^2/∂x^2 + ∂^2 /∂y^2  + ∂^2 /∂z^2  + ...$
					$\nabla^2 = \nabla · \nabla$

	* 积分
		\def{积分}

		\bf{算法} NewtonCotes 公式
			∫_a^b f(x) = (b - a) Σ_(k=0)^n  C_k^(n) f(xi)
			C_k^(n) = (-1)^(n-k) / (n·k!(n-k)!) ∫_0^n Π_(k≠j) (t-j)dt 
			n = 1: C = {1/2, 1/2}
			n = 2: C = {1/6, 4/6, 1/6}
			n = 4: C = {7/90, 32/90, 12/90, 32/90, 7/90}
			* NewtonCotes 公式在 n > 8 时不具有稳定性
			复合求积法: 将积分区间分成若干个子区间, 再在每个子区间使用低阶求积公式.

		* 多元函数积分
			* 重积分
				\def{重积分} ∫∫∫ f(r) dr³
		
* 微分方程

	* 常微分方程

		\bf{算法} 
			\bf{Runge Kutta 方法}
				常微分方程组
					$
						\. y' = f(\. x, \. y) 
						\. y(\. x_0) = \. y_0
					$
					迭代求解 $\. y(x)$ 的一点/一区间的数值解.
					$
						y(x + dx) = y(x) + dx/6 · (k_1 + 2·k_2 + 2·k_3 + k_4)
						k_1 = f(x_n , y_n)						\tag{区间开始斜率}
						k_2 = f(x_n + dx/2, y_n + dx/2·k_1)		\tag{区间中点斜率,通过欧拉法采用k1决定y在xn+dx/2值}
						k_3 = f(x_n + dx/2, y_n + dx/2·k_2)		\tag{区间中点斜率,采用k2决定y值}
						k_4 = f(x_n + dx, y_n + dx·k_3)			\tag{区间终点斜率}
					$

				解常微分方程组
						[ y1'(x) = f1(y1 , ... , yn , x)		  ->   ->
						| y2'(x) = f2(y1 , ... , yn , x)	=>	y' = f(x , y)
						| ...
						[ yn'(x) = fn(y1 , ... , yn , x)

				\Property
					* RK4法是四阶方法,每步误差是h⁵阶,总积累误差为h⁴阶

				\Code
					void RungeKutta(Mat& y, double dx, double x, function f, int enpoch) 
						Mat k1, k2, k3, k4

						while (enpoch--) 
							k1 = f(x, y)
							k2 = f(x + dx / 2, y + (dx / 2) * k1)
							k3 = f(x + dx / 2, y + (dx / 2) * k2)
							k4 = f(x + dx,	 y +  dx	  * k3)

							y = y + h/6 * (k1 + 2 * k2 + 2 * k3 + k4)

	* 偏微分方程

		\Example
			* Poisson's方程
				\def{Poisson's方程} 
					$ Δ \phi = f $
					$ \nabla^2 \phi = f \quad\text{(Euclidean空间)}$  
					三维直角坐标系中 (∂^2 /∂x^2  + ∂^2 /∂y^2  + ∂^2 /∂z^2 ) φ(x,y,z) = f(x,y,z)
					当f ≡ 0, 得到 Laplace's方程
					
				\bf{算法} Green's函数  φ(r) = - ∫∫∫ f(rt) / 4π|r-rt| d³rt	,r rt为矢量


			* 波动方程
				\def{波动方程} a \nabla^2 u = ∂^2 u/∂t^2 

				\bf{算法} 
					有限差分法
					$u(t+1,...) = 2·u(t,...) - u(t-1,...) + Δt^2 ·a{[u(x+1,...) - 2·u(x,...) + u(x-1,...)]/Δx^2  + ...}$

			* 扩散方程
				\def{扩散方程} 
					$a \nabla^2 u = \frac{∂u}{∂t}$

				\bf{算法} 
					有限差分法
					u(t+1,...) = u(t,r)
							+ Δt·a{[u(x+1,...) - 2·u(x,...) + u(x-1,...)]/Δx^2  + ...}

* 级数
	* Taylor级数
		\def{Taylor级数}
			$f(x) = \sum_{n=0}^\infty \frac{f^{(n)}(x_0)}{n!} · (x - x_0)^n$

	* Fourier级数
		\def{Fourier级数}
			$f(x) = \sum_{n=0}^{N-1} e^(-\frac{j 2 π n k}{N}) · x_n$
		
		\bf{算法}
			快速Fourier变换
			* 时间复杂度: O(N·logN)

	* Laplace 变换
		\Notes
			积分实际上就是内积运算, 而我们知道, 向量a与方向向量i的内积, 就是a在i方向上的投影. 如果向量a和不同方向的向量分别求内积, 得到的就是在各个方向上的分量. 如果方向向量满足正交完备, 则这组向量是一组基. 不论是傅里叶, 拉普拉斯, 还是z,  都在做积分运算, 而且都是和某一类函数做积分, 就是投影。换句话说, 原函数被分解为一系列函数的线性叠加, 那么这一系列函数, 其实就是基向量. 变换的好处就是, 便于运算。
			Fourier 变换是分解到正弦函数; Laplace 变换是分解到幅度指数变化的正弦函数; Z 变换是分解到周期变化的离散序列.

