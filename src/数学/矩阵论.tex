* 矩阵论
	* 线性空间
		\def
			一个带有加法和数乘的非空集合, 且满足下列条件,
			- 加法封闭 $x+y \in V$
			- 数乘封闭 $k x \in V$
			- 存在零元 $x+0=x$
			- 存在负元 $x+(-x) = 0$
			- $1x = x$
			- 交换律 $x+y = y+x$
			- 分配律 $(k+l)x = kx+lx$
			- 加法结合律 $x+(y+z) = (x+y) +z$
			- 数乘结合律 $k(Lx) = (kl)x$
	* 线性子空间
		\def
			线性空间中的一个非空集合, 且对线性运算的封闭.
			- 加法封闭 $x,y\in V_1 ,\quad x+y \in V_1$
			- 数乘封闭 $x \in V_1, k x \in V_1$
	* 线性变换
		\def
			线性空间$V$到自身的一类映射, 对于所有$x \in V$都有唯一的$y \in V$与之对应, 且满足
			$T(k x + l y) = k(T x) + l(T y)$
		\Property
			* 特征值、特征向量
				\def
					$T x = \lambda x$
					$x$ 特征向量, 是线性变换前后方向不改变的向量; $\lambda$ 特征值, 是特征向量在线性变换后长度变化的倍率.
		\Example
			* 恒等变换
				\def $T x = x \quad ;(\forall x \in V)$
			* 零变换
				\def $T x = 0 \quad ;(\forall x \in V)$
			* 正交变换
				\def
					$<x, x> = <T x, T x>$
					内积空间中, 保持任意向量的长度不变的线性变换.
					正交矩阵:
						$A A^T = I$
						$A A^H = I$
				* 初等旋转变换
					\def
						初等旋转变换矩阵:
						$T_{ij} = (\mb
							\. I \\ & cos\theta|_{(i,i)}&  & \sin\theta|_{(i,j)} \\ & & \. I \\ & -\sin\theta|_{(j,i)} & & \cos\theta|_{(j,j)} \\ & & & & \. I
						\me)$
				* 初等反射变换
					\def
						$y = H x = (I - 2 e_2 e_2^T) x$
						\Proof
							$
								x - y &= e_2 · (e_2^T x)
								=> y &= (I-2 e_2 e_2^T) x
							$
			* 对称变换
				\def
					$<T x, y> = <x, T y>$
					对称矩阵:
						$A^T = A$
						$A^H = A$
			* 投影变换
				\def
					令线性空间分为不交的子空间L,M, 投影变换是将线性空间沿M到L的投影的变换.
					投影矩阵: 
						$P_{L,M} = (X & 0)\ (X & Y)^{-1}$
				* 正交投影变换
					\def
						设线性空间的子空间L, 将线性空间沿$L_\bot$到L的投影的变换, 称投影变换.
						正交投影矩阵:
							投影后子空间的基 $X = (x_1, ... , x_r) ,$ 则正交投影矩阵 $P_L = X(X^H X)^{-1}X^H$.
			* 斜切变换
				\def
					斜切变换矩阵: 
						单位矩阵的第(i,j)个元素改为斜切比率 $a_{ij}$
			* 缩放变换
				\def
					缩放变换矩阵:
					$T = (\mb dx_1 \\ & dx_2 \\ & & \ddots \\ & & & dx_n \me)$
	* 范数
		* 向量范数
			\def
				一类函数, 且满足
				- 非负性, $||A|| ≥ 0$, 当且仅当$A = 0, ||A|| = 0$
				- 齐次性, $||k A|| = |k| ||A||$
				- 三角不等式, $||A + B|| e ||A|| + ||B||$
			\Example 
				* $p$-范数 $||x||_{p}=(\sum_{i=1}^{n}|x_i|^p)^{1 / p}$
				* $\infty$-范数 $||x||_\infty = \max|x_i|$
				* 椭圆范数 $||x||_{A}=(x^T A x)^{\frac{1}{2}}$
		* 矩阵范数
			\def
				一类函数, 且满足
				- 非负性, $||A|| \ge 0$, 当且仅当$A = 0, ||A|| = 0$
				- 齐次性, $||k A|| = |k| ||A||$
				- 三角不等式, $||A + B|| e ||A|| + ||B||$
				- 相容性, $||A B|| e ||A||\ ||B||$
			\Example 
				* $||A||_{m_1} = \sum_{i,j} |a_{ij}|$
				* $||A||_{m_2} = (\sum_{i,j} a_{ij}^2)^{\frac{1}{2}}$
				* $||A||_{m_\infty} = n·\max_{i,j}|a_{ij}|$
				* 列和范数 $||A||_1	  = \max_j \sum_i |a_{ij}|$
				* 行和范数 $||A||_\infty = \max_i \sum_j |a_{ij}|$
				* 谱范数   $||A||_2 = \sqrt{\max\ \lambda_i} \quad ,(\lambda_i)$为$A^H A$特征值.
		\Note
			矩阵范数, 向量范数相容: $||A x||_V \le ||A||_M ||x||_V$
	* 矩阵分解
		\Example
			* 上下三角分解
				\Algorithm
					- 目的
						将矩阵A化成上三角矩阵R与下三角矩阵L的乘积.$A = L R$
			* 上下三角对角分解
				\Algorithm
					- 目的
						将矩阵A化成上三角矩阵R, 对角矩阵D, 下三角矩阵L的乘积.$A = L D R$
			* 对称三角分解
				\Algorithm
					- 目的
						将对称正定矩阵化成对称的两个上下三角矩阵. $A = G G^T$
					- 步骤
						- 先上下三角分解 $A = L D U = L D L^T \tag{因为对称正定矩阵}$
						- 
							$
								A = L (\sqrt(D))^2 L^T
									= (L \sqrt(D)) (\sqrt(D) L^T)
									= (L \sqrt(D)) (L \sqrt(D))^T
									= G G^T
							$
			* 正交三角分解
				\Algorithm
					- 目的
						将非奇异矩阵A化成正交矩阵Q与非奇异上三角矩阵R的乘积. $A = Q R$
					- Schmidt正交化方法
						- 步骤
							- $A = (a_1, ..., a_n)$
							- Schmidt正交化 $b_i = a_i - \sum_{k=1}^{i-1} \frac{<a_i,b_j>}{<b_j,b_j>}b_j$
							- 
								$
									Q = ( \frac{b_1}{|b_1|}, ... , \frac{b_n}{|b_n|} )
									R = (\mb |b_1|\\ & \ddots\\ && |b_{n}| ) ( 1 & k_{21} & ... & k_{n1} \\ & 1 & ... & k_{n2} \\& & \ddots & \vdots \\& & & 1 \me) \quad; k_ij = \frac{<a_i,b_j>}{<b_j,b_j>}
									A = Q R
								$
					- 初等旋转变换方法
						- 步骤
							- 对第1列, 初等旋转变换使其变为 $T_i a_1 = (b_{11}, 0,...,0)$
							- $T_i = \prod_{i=0}^{n-1} T_{i(n-1-j)}$
							- 重复上面步骤, 直至将 $A_i$ 化为上三角矩阵
							- 
								$
									R = A_{n-1}
									Q = (\prod_{i=0}^{n-1} T_{n-1-i} )^T
									A = Q R
								$
					- 初等反射变换
						- 步骤
							- 对第1列, 初等旋转变换使其变为 $H_i a_1 = (b_{11}, 0,...,0)$
								$
									u_i = \frac{b_i - |b_i|}{| b_i - |b_i| |}
									H_i = I - 2 u u^T
									A_{i+1} = H_i A_i
								$
							- 重复上面步骤, 直至将 $A_i$ 化为上三角矩阵
							- 
								$
									R = A_{n-1}
									Q = (\prod_{i=0}^{n-1} H_{n-1-i} )^T
									A = Q R
								$
			* 满秩分解
				\Algorithm
					- 目的
						将矩阵A化成F G的乘积. $A = F G$
						\Proof
							$A=P^{-1} B= (\mb F & S \me) (\mb G\\ 0 \me) = F G$
					- 步骤
						初等行变换$A \to (\mb G\\ 0 \me)$, 取A左侧rank(A)列作为F, 则$A = F G$
			* 奇异值分解
				\Algorithm
					- 目的
						将矩阵A化成两个Unitary矩阵$U, V$, 和一个非零奇异值组成的矩阵$\Sigma$的乘积. $A = U (\Sigma & 0 \\ 0 & 0) V^T$
					- 步骤
						- $A^T A$ 计算特征值 $\lambda$, 特征向量$x$
						- $V = ( \frac{x_1}{|x_1|}, ... ,\frac{x_n}{|x_n|} ), \quad \Sigma = diag(\sqrt(\lambda_1), ... ,\sqrt(\lambda_n))$
						- $U_1 = A V \Sigma^{-1}$, 计算正交矩阵$U$
						- $ A = U (\mb \Sigma & 0 \\ 0 & 0 \me) V^T $
