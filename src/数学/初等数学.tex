* 初等函数
	* 三角函数
		\def{三角函数}
			单位圆 $x^2 + y^2 = 1$
			$\theta$ 半径线段x与x轴正向所成的夹角
			$
				cos \theta = x
				sin \theta = y
			$

	* 双曲函数
		\def{双曲函数}
			单位双曲线 $x^2 - y^2 = 1$
			$
				cosh \theta = x
				sinh \theta = y
			$

	\Property
		$
			\sin  x = \frac{e^{ix} - e^{-ix}}{2 i}
			\cos  x = \frac{e^{ix} + e^{-ix}}{2}
			\sinh x = \frac{e^{x}  - e^{-x} }{2}
			\cosh x = \frac{e^{x}  + e^{-x} }{2}
		$

* Euclidean几何
	* \bf{公理}
		* 直线公理: 过相异两点, 能作且只能作一直线.
		* 任意线段能无限延长成一条直线.
		* 圆公理:   以任一点为圆心、任意长为半径, 可作一圆.
		* 角公理:   凡是直角都相等.
		* 平行公理: 在平面上, 过直线外一点, 有且只有一条直线与该直线不相交.

	* 
		* 反射
		* 折射

	* 三角形
		* \Theorem{正弦定理}
			$2 R = \frac{a}{\sin \alpha} = \frac{b}{\sin \beta} = \frac{c}{\sin \gamma}$
			其中, R为三角形外接圆半径,

		* 外接圆
			* 外接圆半径
				\Theorem{Heron-秦九韶公式}
					设平面内一个三角形, 边长分别为a、b、c, p为三角形周长的一半, 则三角形的面积S, 
					$
						S = \sqrt{p(p-a)(p-b)(p-c)}
						p = \frac{a+b+c}{2}
					$

			* 外接圆圆心

				\Algorithm{判断四点共圆}
					*	三点确定圆方程: 即 解行列式:
						$ |\mb
							x^2+y^2 & x & y & 1
							x_1^2+y_1^2 & x_1 & y_1 & 1
							x_2^2+y_2^2 & x_2 & y_2 & 1
							x_3^2+y_3^2 & x_3 & y_3 & 1
						\me| ?= 0
						$
					* 几何解释: 通过把平面点提升到三维的抛物面中, 由于抛物面被平面所截的截面为圆形, 四点共面即使共圆, 也可以用四面体的体积判断是否共圆。

				\Algorithm{平面三点确定圆方程}
					* 圆方程: $(x - x_c)^2 + (y - y_c)^2 = R^2$
					* 三点确定圆方程: 即 解行列式:
						$ |\mb
							x^2+y^2 & x & y & 1
							x_1^2+y_1^2 & x_1 & y_1 & 1
							x_2^2+y_2^2 & x_2 & y_2 & 1
							x_3^2+y_3^2 & x_3 & y_3 & 1
						\me| = 0
						$
						即.目标三点和圆上(x,y)应该满足方程组:
						$(x^2+y^2)·a + x·b + y·c + 1·d = 0$

					\Proof
						$
							M_{11}(x^2+y^2) - M_{12} x + M_{13} y - M_{14} = (x^2+y^2)·a + x·b + y·c + 1·d = 0
							=> (x^2 + b/a x) + (y^2 + c/a y) = - d/a
							=> (x + b/2a)^2 + (y + c/2a)^2 = -d/a + b^2/4a^2 + c^2/4a^2
						$
													
			* 三角形外接圆
				\Algorithm{三角形外接圆}
					外接圆圆心: 即. 三点确定圆方程问题,  也是任意两边的垂直平分线的交点.直接用平面三点确定圆方程方法
				
	* 交点
		* 射线、平面交点
			$
				d = - (A x_0 + B y_0 + C z_0 + D) / (A a + B b + C c)
			$

			\Proof
				平面方程: $A x + B y + C z + D = 0$
				直线方程: $(x - x_0) / a = (y - y_0) / b = (z - z_0) / c = K$
				联立平面,直线方程, 解K, K即直线到射线平面交点的距离
				$
					x = K a + x_0 , ...
					A(K a + x_0) + B(K b + y_0) + C(K c + z_0) + D = 0
					=> K =  - (A x_0 + B y_0 + C z_0) / (A a + B b + C c)
				$

		* 射线、三角形交点

			\Proof
				射线: $P = O + t D$
				射线三角形交点:
				$
						O + t D = (1 - u - v)V0 + u V1 + v V2
						[ -D  V1-V0  V2-V0] [ t  u  v ]' = O - V0
						T = O - V0	E1 = V1 - V0	E2 = V2 - V0
						[ -D  E1  E2 ] [ t  u  v ]' = T
						t = | T  E1  E2| / |-D  E1  E2|
						u = |-D   T  E2| / |-D  E1  E2|
						v = |-D  E1  E2| / |-D  E1  E2|
				(混合积公式): |a  b  c| = a × b·c = -a × c·b
						t = (T × E1·E2) / (D × E2·E1)
						u = (D × E2· T) / (D × E2·E1)
						v = (T × E1· D) / (D × E2·E1)
				$

		* 射线、圆交点

			\Codes
				double RayCircle(Mat RaySt, Mat Ray, Mat Center, double R, Mat n)
					double D = -(n[0] * Center[0] + n[1] * Center[1] + n[2] * Center[2]),
						d = RayPlane(RaySt, Ray, n[0], n[1], n[2], D)
					if (d == DBL_MAX)
						return DBL_MAX
					return norm(RaySt + d * Ray - Center) <= R ? d : DBL_MAX

			* 射线、球面交点

			\Proof
				球方程: $(X - Xs)^2 + (Y - Ys)^2 + (Z - Zs)^2 = R^2$
				线球交点: $K^2(Al^2 + Bl^2 + Cl^2) + 2K(Al ΔX + Bl ΔY + Cl ΔZ) + (ΔX^2 + ΔY^2 + ΔZ^2 - R^2) = 0$
				$
						ΔX = Xl - Xs
						Δ = b^2 - 4ac = 4(Al ΔX + Bl ΔY + Cl ΔZ)^2 - 4(Al^2 + Bl^2 + Cl^2)(ΔX^2 + ΔY^2 + ΔZ^2 - R^2)
						若Δ≥0 有交点.
						K = ( -b ± sqrt(Δ) ) / 2a	即光线走过线距离
				$

		* 射线、矩体交点

			\Codes
				double RayCuboid(Mat RaySt, Mat Ray, Mat pmin, Mat pmax)
					double t0 = -DBL_MAX, t1 = DBL_MAX
					for dim in range(0, 3)
						if (fabs(Ray[dim]) < EPS && (RaySt[dim] < pmin[dim] || RaySt[dim] > pmax[dim])) 
							return DBL_MAX

						double
							t0t = (pmin[dim] - RaySt[dim]) / Ray[dim],
							t1t = (pmax[dim] - RaySt[dim]) / Ray[dim]
						if (t0t > t1t)
							swap(t0t, t1t)

						t0 = max(t0, t0t)
						t1 = min(t1, t1t)
						if (t0 > t1 || t1 < 0)
							return DBL_MAX

					return t0 >= 0 ? t0 : t1
			
	* 凸包
		\Algorithm{Graham 扫描法}
			\Property
				* 时间复杂度 $O(n \log n)$

			* 流程:
				* 选择y最小点 p0, 若多个则选其中x最小
				* sorted by polar angle in counterclockwise order around p0 (if more than one point has the same angle, remove all but the one that is farthest from p0)
					* 几何可知, 排序后 P1 和最后一点 Pn-1 一定是凸包上的点
				* P0,P1 入栈S, P2 为当前点, if n<2, error "Convex Hull is empty"
				* 遍历剩余点 P3 -> Pn-1
					* while the angle formed by points NEXT-TO-TOP(S),TOP(S),and p makes a nonleft turn, POP(S)
					* Pi 入栈
				* 最后栈中元素, 即结果

	* 三角剖分
		\Algorithm{Delaunay三角剖分} 
			* 原理:
				每个三角形的外接圆内不包含V中任何点

			* 流程:
				* 将点按坐标x从小到大排序
				* 确定超级三角形, 将超级三角形保存至未确定三角形列表 trianglesTemp
				* 遍历每一个点
					* 初始化边缓存数组 edgeBuffer
					* 遍历 trianglesTemp 中的每一个三角形
						* 计算该三角形的圆心和半径
						* 如果该点在外接圆的右侧, 则该三角形为Delaunay三角形, 保存到triangles,并在temp里去除掉,跳过
						* 如果该点在外接圆外(即也不是外接圆右侧), 则该三角形为不确定,跳过
						* 如果该点在外接圆内, 则该三角形不为Delaunay三角形,将三边保存至edgeBuffer,在temp中去除掉该三角形
					* 对edgeBuffer进行去重
					* 将edgeBuffer中的边与当前的点进行组合成若干三角形并保存至temp triangles中
				* 将triangles与trianglesTemp进行合并, 并除去与超级三角形有关的三角形

* 分形
	* Mandelbrot集
		\def{Mandelbrot集}
			所有能使$Z_{n+1}$不发散的复数点C, 所构成的集合,即 Mandelbrot Set. (不发散,不一定收敛,有可能在几个不同点来回跳)
			$Z_{n+1} = Z_n^2 + C$
			
		\Property
			* $|Z_n|>2$不可能收敛, 即Mandelbrot Set在半径为2的圆内.

	* Julia集
		\def{Julia集}
			对于某复数值C,所有能使Z_n+1不发散的Z0的集合,即 Julia Set. 类似于. Mandelbrot 曼德布洛特集
			$Z_{n+1} = Z_n^2 + C$

	* Hilbert曲线
		\def{Hilbert 曲线} 一种自相似的分形曲线

	 	\Algorithm
	 		四象限复制四分, 翻转左下、右下, 使左下末同左上初、右下初同右上末,能够最短连接.
			翻转坐标, 使Hilbert曲线性质, 自相似地适用于左下、右下象限
			* 1,2阶曲线
				"┌┐"	   "┌┐┌┐"	"┌┐┌┐ "
				︱︱  =>	︱︱︱︱ =>   ︱└┘︱
						┌┐┌┐	  └┐┌┘
						︱︱︱︱	   -┘└-

		\Property
			* 边长: nth 2^n
			* 长度: nth 2^n - 1 / 2^n
			* 因为是四进制自相似, 所以曲线上位置 distance, 可以不断判断子象限, 按二进制在位上叠加
			* 用途: 1D to 2D 的映射算法, 随 n 增加, 映射的位置趋于收敛

	* Perlin噪音

		\Algorithm
			* 格点随机梯度矢量
			* (x,y)与格点距离,梯度点积
			* 插值

\Algorithm{Marching Cubes 三维等高面绘制}

* 
	* 坐标系
		* 直角坐标系

		* 极坐标系
			\Example

		* 参数坐标系

	* 曲线
			* 心脏线
				\def{心脏线}
					* 极坐标: $ r = a (1 - \sin \theta) $

				\Property
					* 面积: $S = 3 \frac{π a^2}{2}$

			* Descartes 叶形线
				\def{Descartes 叶形线}
					* 直角坐标: $x^3 + y^3 - 3 a x y = 0$
					* 极坐标: $r = \frac{3 a \sin \theta \cos \theta}{\sin^3 \theta + \cos^3 \theta}$

				\Property
					* 渐近线: $x + y + a = 0$
					* 圈套顶点: $A(\frac{3a}{2}, \frac{3a}{2})$
					* 圈套的面积: $S_1 = 3 \frac{3 a^2}{2}$
					* 曲线与渐近线的面积: $S_2 = S_1 = 3 \frac{3 a^2}{2}$




