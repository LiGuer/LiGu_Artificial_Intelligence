* 概率论
    * 概率空间
        \def{概率空间}
            概率空间是一个三元组 $<\Omega, \mathcal F, \P>$
            * $\Omega$, 样本空间.
            * $\mathcal F$: 被选择的 $\Omega$的子集系列. 且满足
                * 包含空集、样本空间全集 $\emptyset, \Omega \in \mathcal F$
                * 取补封闭,如果一个事件A在其中,那么补集也需要在其中.  
                    $A \in \mathcal F \Rightarrow A^C \in \mathcal F$
                * 可列并封闭 
                    $A_1, A_2, \cdots \in \mathcal F \Rightarrow \bigcup_{i=1}^\infty A_i \in \mathcal F$
            * $\P : \mathcal F \to [0, 1]$ 概率测度, 且满足Kolmogorov公理.

    * 公理
        \def{Kolmogorov公理}
            * 非负性 $\P(A) \in [0, 1] \quad ; \forall A \in F$
            * 规范性 $\P(\Omega) = 1$
            * 可列可加性 $\P (\bigcup_i A_i)=\sum_i \P(A_i)$

    * 概率
        * \Theorem{全概率公式}

        * \Theorem{Bayes公式}
            $P(A | B)=\frac{P(B | A) P(A)}{P(B)}$

    * 概率分布

        * 离散概率分布

            \Example
                * 0-1分布
                    \def{0-1分布} 随机变量X只能取0或1
                        $\P{X = k} = p^k (1 - p)^k \quad; k \in \{0,1\}$

                    \Property
                        * 均值 $\E(x) = p$
                        * 方差 $D(x) = p (1 - p)$

                * 二项分布
                    \def{二项分布} 
                        $\P{X = k} = C^k_n p^k (1-p)^{n-k}$

                    \Property
                        * 均值 $\E(x) = n p$
                        * 方差 $D(x) = n p(1 - p)$

                * 几何分布
                    \def{几何分布} 描述在连续独立重复实验中, 首次成功所进行的实验次数.
                        $\P{X = k} = p (1-p)^{n-1}$

                    \Property
                        * 均值 $\E(x) = \frac{1}{p}$
                        * 方差 $D(x) = \frac{1 - p}{p^2}$

                * 超几何分布
                    \def{超几何分布} 
                        $\P{X = k} = \frac{C_M^k C_{N_M}^{n-k}}{C_N^n}$

                    \Property
                        * 均值 $\E(x) = \frac{n M}{N}$
                        * 方差 $D(x) = \frac{n M}{N} (1 - \frac{M}{N})\frac{N - M}{N - 1}$

                * Poisson分布
                    \def{Poisson分布} 
                        $\P{X = k} = \frac{\lambda^k}{k!} e^{-\lambda}$

                    \Property
                        * 均值 $\E(x) = \lambda$
                        * 方差 $D(x) = \lambda$

                    \Theorem{Poisson定理} Poisson分布是二项分布的极限情景($\lambda = n p , n \to \infty, p \to 0$)
                        $\lim_{n \to \infty, p \to 0} \frac{C_M^k C_{N_M}^{n-k}}{C_N^n} = \frac{\lambda^k}{k!} e^{-\lambda}$

        * 连续概率分布
            \Example
                * 均匀分布
                    \def{均匀分布} 
                        $
                            f(x) = \{\mb \frac{1}{b-a} &\quad a < x < b \\ 0 &\quad other \me\right.
                            F(x) = \{\mb 0 &\quad x < a \\ \frac{1}{b-a} &\quad a ≤ x < b \\ 1 &\quad b ≤ x \me\right.
                        $

                    \Property
                        * 均值 $\E(x) = \frac{a + b}{2}$
                        * 方差 $D(x) = \frac{(b - a)^2}{12}$

                * Normal分布
                    \def{Normal分布} 
                        $
                            f(x) = \frac{1}{\sqrt{2 \pi \sigma^2}} e^{-\frac{(x - \mu)^2}{2 \sigma^2}} \quad; x \in (-\infty, +\infty)
                        $

                    \Property
                        * 均值 $\E(x) = \mu$
                        * 方差 $D(x) = \sigma^2$

                    * 标准Normal分布
                        $
                            f(x) = \frac{1}{\sqrt{2 \pi }} e^{-\frac{x^2}{2}} \quad; x \in (-\infty, +\infty)
                        $

                * $\Gamma$分布
                    \def{$\Gamma$分布} 
                        当$\alpha=1$时,$\Gamma$分布退化为指数分布;
                        当$\alpha=n/2, \beta=\frac{1}{2}$时,$\Gamma$分布退化为$\chi^2$分布.
                        $
                            f(x) = \{\mb \frac{1}{\beta^\alpha \Gamma(\alpha)} x^{a^{-1}} e^{-x / \beta} &\quad x \in (0, +\infty) \\ 0 &\quad x \in (-\infty, 0] \me\right.
                        $

                    \Property
                        * 均值 $\E(x) = \alpha \beta$
                        * 方差 $D(x) = \alpha \beta^2$

                    * 指数分布
                        \def{指数分布} 
                            $
                                f(x) = \{\mb \lambda e^{-\lambda x} &\quad x \in (0, +\infty) \\ 0 &\quad x \in (-\infty, 0] \me\right.
                                F(x) = \{\mb \lambda 1 - e^{-\lambda x} &\quad x \in (0, +\infty) \\ 0 &\quad x \in (-\infty, 0] \me\right.
                            $
                        \Property
                            * 均值 $\E(x) = \theta$
                            * 方差 $D(x) = \theta^2$

                    * $\chi^2$分布
                        \def{$\chi^2$分布} 
                            $\frac{1}{2^{n / 2} \Gamma(n / 2)} x^{n / 2 - 1} e^{-x / 2}$
                        \Property
                            * 均值 $\E(x) = n$
                            * 方差 $D(x) = 2 n$

* 随机过程
    * 平稳性
        \def{平稳性}
            时间平移不变性, 即统计特性不随随时间推移而改变. 在时间序列上随意的任意间隔任意顺序的采样, 都具有时间平移不变性.
            $ 
                \P(x_{t_1}, \ldots, x_{t_n}) = \P(x_{t_1+\tau}, \ldots, x_{t_n+\tau}) \quad \text { for all } \tau, t_1, \ldots, t_n \in \mathbb R, \quad n \in \mathbb N
            $

    * Markov性
        \def{Markov性}