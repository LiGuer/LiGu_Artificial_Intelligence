* 概率论
    * 概率空间
        \def{概率空间}
            概率空间是一个三元组 $<\Omega, \mathcal F, \P>$
            * $\Omega$, 样本空间.
            * $\mathcal F$: 被选择的 $\Omega$的子集系列. 且满足
                * 包含空集、样本空间全集 $\emptyset, \Omega \in \mathcal F$
                * 取补封闭,如果一个事件A在其中,那么补集也需要在其中.  
                    $A \in \mathcal F \Rightarrow A^C \in \mathcal F$
                * 可列并封闭 
                    $A_1, A_2, \cdots \in \mathcal F \Rightarrow \bigcup_{i=1}^\infty A_i \in \mathcal F$
            * $\P : \mathcal F \to [0, 1]$ 概率测度, 且满足Kolmogorov公理.

    * 公理
        \def{Kolmogorov公理}
            * 非负性 $\P(A) \in [0, 1] \quad ; \forall A \in F$
            * 规范性 $\P(\Omega) = 1$
            * 可列可加性 $\P (\bigcup_i A_i)=\sum_i \P(A_i)$

    * \Theorem{Bayes公式}
        $P(A | B)=\frac{P(B | A) P(A)}{P(B)}$


* 随机过程
    * 平稳性
        \def{平稳性}
            时间平移不变性, 即统计特性不随随时间推移而改变. 在时间序列上随意的任意间隔任意顺序的采样, 都具有时间平移不变性.
            $ 
                \P(x_{t_1}, \ldots, x_{t_n}) = \P(x_{t_1+\tau}, \ldots, x_{t_n+\tau}) \quad \text { for all } \tau, t_1, \ldots, t_n \in \mathbb R, \quad n \in \mathbb N
            $

    * Markov性
        \def{Markov性}