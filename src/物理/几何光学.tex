* 几何光学
	* 基本原理
		- 光沿直线传播.
		- 光路可逆.
		- 两束光传播中互不干扰,会聚于同一点时光强简单相加.
		- 介质分界面的光传播规律.
	* 介质分界面光传播规律
		* 反射
			\Theorem{反射定律}
				$\theta_i = \theta_o$
				入射角 = 反射角. 
			\Algorithm{反射光线}
				$\.L_o = \.L_i - 2 (\.N · \.L_i) \.N$
				面矢$\.N$, 入射光$L_i$, 出射光$L_o$
		* 折射
			\Theorem{折射定律} 
				$n_i·\sin \theta_i = n_o·\sin \theta_o$
				全反射时,
			\Algorithm{折射光线}
				$\.L_o = \.L_i + \.N· (\cos\theta_i - \frac{\sqrt{1 - n^2·\sin^2\theta_i}}{n})$
				面矢\.N, 入射光\.L_i, 出射光$L_o$, 折射率 $n = \frac{n_i}{n_o}$, 折射角$ \theta_i = <\.L, \.N>$
				\Proof
					$
						\.L_o &= \.L_i + \.N·\frac{\sin(\theta_o - \theta_i)}{\sin\theta_o}
						&= \.L_i + \.N· \frac{\cos\theta_i·\sin\theta_o - \cos\theta_o·\sin\theta_i}{\sin\theta_o}
						&= \.L_i + \.N· (\cos\theta_i - \frac{\cos\theta_o}{n})
						&= \.L_i + \.N· (\cos\theta_i - \frac{\sqrt{1 - n^2·\sin^2\theta_i}}{n})
					$
			\Example
				* 凸透镜
					\def
						边缘薄、中间厚的透镜. 主要对光起会聚的作用.
					\Property
						- 物距 $u > 2 F$, 像距$v \in (F, 2F)$, 倒立缩小的实像, 物像异侧.
						- 物距 $u = 2 F$, 像距$v = 2 F$, 倒立等大的实像, 物像异侧.
						- 物距 $u \in (F, 2F)$, 像距$v > 2 F$, 倒立放大的实像, 物像异侧.
						- 物距 $u = F$, 不成像.
						- 物距 $u < f$, 像距$v > F$, 正立放大的虚像, 物像同侧.
				* 凹透镜
		* 漫反射
			在面矢半球内, 面积均匀的随机取一射线, 作为反射光线.
	* 雾
		\Theorem{雾图模型}
			$L_o(x) = L_i(x) · t(x) + A · (1 - t(x))$
			$L_o(x)$ 观测到的亮度, $L_i(x)$ 原始无雾时亮度, $t(x)$ 透射率, $A$大气光亮度. 
			当大气密度、成分是均匀同质时, 透射率 $t(x) = e^{-\beta d(x)}$
			\Notes
				雾图模型, 即每个原始像素亮度和大气光亮度, 按比例$t : (1-t)$调配(透射率). 大气光的比例越多, 雾感越强. 得到每个像素的透射率和大气光亮度, 即可实现雾化和去雾的功能.
				- 当大气密度、成分是均匀同质时, 透射率与像素所在的景物深度相关. 景深越深, 大气光占比越大. 
				- 当像素景深近似无限远时, 可认为该点透射率$t = 0$, 像素亮度全由大气光提供. 可由此得到大气光亮度值$A$.
	\Theorem{光照模型}
		\Theorem{双向反射分布函数}
			$L_o(\.x, \omega_o, \lambda, t) = L_e(\.x, \omega_o, \lambda, t) + \int_{\Omega} f_r(\.x, \omega_i, \omega_0, \lambda, t) L_i(\.x, \omega_i, \lambda, t)(\omega_i · \.n) \d \omega_i$
			一个表面上某一点的亮度由以下几部分组成, 表面的自发光$L_e$, 所有射向该表面的入射光线$L_i$经过表面特性作用后的出射光线.
		\Algorithm{光线追踪}
			- 步骤
				- 遍历屏幕中每个像素, 计算像素矢量、射入光线初始矢量
					- 计算该像素内射入光线的颜色, 对射入光线进行反向追踪
						- 在对象集合中, 搜索与当前反向光线相交的最近的对象, 计算相交的交点和相交的面矢,
						- 根据相交面的特征, 计算反向光线在介质分界面传播后的光线, 并作为新的反向光线.
						- 直至得到当前反向光线的颜色数据, 如反向光线到达光源、无限远、追踪阈值上限等.
						- 回溯追踪过程, 得到原始射入光线的颜色.
