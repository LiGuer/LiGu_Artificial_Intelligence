
* 复数
	\def{复数}
		复数是一对有序的实数$(a, b)$, 复数集合 $\.C = \{ (a, b) | a , b \in \.R$.
		因为$(0, 1) = \i$虚数单元, 所以复数也记作 $z = a + b · \i$, 且$\i · \i = -1$.

	* 运算
		* 模、幅角 
			$
				r = |z| = \sqrt{a^2 + b^2}
				\theta = arg\ z = arctan(\frac{b}{a})
			$
			因此复数也可以表示为
			$z = a + b · \i = r (\cos \theta + \i · \sin \theta) = r e^{\i · \theta}$
		* 加 $z_1 + z_2 = (a + b · \i) + (c + d · \i) = (a + b) + (c + d) · \i$
		* 减 $z_1 - z_2 = (a - b · \i) + (c - d · \i) = (a - b) + (c - d) · \i$
		* 乘 $z_1 · z_2 = (a - b · \i) · (c - d · \i) = (a c - b d) + (a d + b c) · \i$
		* 除 $\frac{ z_1 }{ z_2 } = \frac{a - b · \i}{c - d · \i} = \frac{a c + b d}{c^2 + d^2} + \frac{b c + a d}{c^2 + d^2} · \i$
		* 共轭 $\bar z = a + (-b) · \i$
		* 幂  
			\Theorem{De Moiver公式}
				$z^p = r^p (\cos \theta + \i · \sin \theta)^p = r^p (\cos (p · \theta) + \i · \sin(p · \theta))$

			$z_1^{z_2} = $

	* 初等函数
		* 指数函数
			$e^z = e^{a + b · \i} = e^a (\cos a + \i · \sin b)$
		* 对数函数
			$\ln(z) = \ln(r e^{\i · \theta}) = \ln(r) + \i ·\theta$
		* 三角函数、双曲函数
			$
				\cos z  = \frac{e^{\i · z} + e^{\i · z}}{2}
				\sin z  = \frac{e^{\i · z} - e^{\i · z}}{2 · \i}
				\sinh z = \frac{e^{z} - e^{-z} }{2}
				\cosh z = \frac{e^{z} + e^{-z} }{2}
			$
			
* 复变函数
	* 解析函数
		\def{解析函数}
			复可微指$\lim_{z \to z_0} \frac{f(z)-f(z_0)}{z-z_0}$存在. 函数$f$在域D中的没一点都可微, 则函数$f$称域D上的解析函数.

		* \Theorem{留数定理}