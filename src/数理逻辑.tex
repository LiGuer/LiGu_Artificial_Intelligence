
* 命题
	\def{命题} 非真即假的陈述句.

		\bf{命题联结词}
			* \def{否定} $\neg$
			* \def{合取} $\wedge$
			* \def{析取} $\vee$
			* \def{蕴含} $\to$
			* \def{等价} $ <=> $

			\Property
				* 关系: 优先级顺序, 从上到下, 同级运算符从左往右算.

				* 基本等值公式
					* \bf{双重否定律} $\neg \neg p = p$
					* \bf{结合律} 
						$
							(P \wedge Q) \wedge R &= P \wedge(Q \wedge R) 
							(P \vee Q) \vee R &= P \vee(Q \vee R) 
							(P <=> Q) <=> R &= P  <=> (Q <=> R)
						$
					* \bf{交换律}
						$
							P \wedge Q &= Q \wedge P 
							P \vee Q &= Q \vee P 
							P <=> Q &= Q <=> P
						$
					* \bf{分配律}
					* \bf{吸收律} $P \vee (P \wedge Q) = P \wedge(P \vee Q) = P$
					* \bf{反演律}
						$
							\neg(P \vee   Q) &= \neg P \wedge \neg Q 
							\neg(P \wedge Q) &= \neg P \vee \neg Q
						$

		\def{范式} 每一类中的命题公式, 能逐步化为一种统一规范的形式.
		
		\def{谓词} 描述个体的属性, 以及个体之间的关系.

		\def{量词} 用来对个体的数量进行约束.
			* \def{全称量词} $\forall$
			* \def{存在量词} $\exists$