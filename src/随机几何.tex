\section{随机几何}
    \section{点过程}
        \bf{注}: “点过程”一词中的“过程”并不意味着随时间的动态演变。从历史上看,该理论的第一批研究人员考虑的是时间事件的随机序列,例如电话交换机的呼叫到达瞬间、排队的客户到达、地震或矿难的发生等。然而,在R2或R3的点过程应用中,时间的概念通常是不存在的。“随机点场”将是一个更精确的术语.
        $$\text{点过程} = \text{随机点场}$$
        $$\text{静止} = \text{统计均匀}$$
        以Poisson点过程为例,

        * \bf{二项过程}$X \sim B(n,p)$, 
            * 输入: X发生次数
            * 参数: \{$n, p$\} 实验次数, 事件发生概率.
            * 意义: n次实验, 发生概率p, 事件所发生的次数X.

            
        * \bf{Poisson过程}$X \sim Poisson(\lambda)$是二项过程($\lambda = np , n \to \infty, p \to 0$)的极限情景,
            * 输入: X 发生次数
            * 参数: $\lambda$ 单位事件内平均发生的次数.
            * 意义: 单位时间内, 事件发生的次数X.
            
            
        * \bf{Poisson点过程}, 是Poisson过程的高维几何空间的延申. Poisson过程是单位时间上的情景, 可看作一维空间下随机点分布; 延申到高维, 即在高维空间中的随机点分布.
            * 参数: $\lambda$ 单位Lebesgue测度内(eg.长度,面积,体积)随机点的密度.
            * 意义: 单位Lebesgue测度内, 随机点的个数.
            
        
        
        \section{Poisson点过程}
            \bf{Define}: $\forall$紧集$B \set \mathbb{R}^d$, 其内部的点数, 是一个Poisson随机变量, 且不相交集中的点数是独立的.\\
            poisson分布概率函数\quad $P(x = k) = \frac{\lambda^k}{k!} e^{-\lambda}$\\
            
        \section{Neyman–Scott过程}
            \section{Matern簇过程}
        
        \section{硬核点过程}
        
        \section{Gibbs点过程}
        
        \section{散粒噪声场}
    


\section{随机几何下-无线通信物理量解析解}
    \section{基站的Poisson点过程建模}
        \bf{Example:}\\
            对一终端的半径r的圆内,除了其指定基站外,没有其它基站的概率为: $P(x = 1) = e^{-\lambda \pi R^2}$\\
            针对变量r而言,\\
            概率分布函数, 则有\quad $F(R) = P(r \le R) = 1 - e^{-\lambda \pi R^2}$\\
            概率密度函数, 则有\quad $f(r) = dF(r) / dr = e^{-\lambda \pi R^2}\ 2 \pi \lambda r$
    \section{小区平均面积}
        随机几何小区内平均面积概率密度,应当符合$\Gamma$分布:
            $$f(y)=\frac{b^{a}}{\Gamma(a)} y^{a-1} e^{-by}$$
        而在二维空间中,有 $a = b = 3.5$:
            $$f_{2 D}(y)=\frac{3.5^{3.5}}{\Gamma(3.5)} y^{2.5} e^{-3.5 y}$$
    \section{小区内平均用户数}
        $$P[N=n]=\frac{3.5^{3.5} \Gamma(n+3.5)\left(\lambda_{u} / \lambda_{b}\right)^{n}}{\Gamma(3.5) n !\left(\lambda_{u} / \lambda_{b}+3.5\right)^{n+3.5}}$$
        Proof:
        \begin{align*}
        P[N=n]  
            &=\int_{0}^{\infty} P[N=n \mid S=s] \cdot f_{2D}(s) d s =\int_{0}^{\infty} \frac{\left(\lambda_{u} \frac{s}{\lambda_{b}}\right)^{n}}{n !} e^{-\lambda_{u} \frac{s}{\lambda_{b}}} \cdot f_{2D}(s) d s \\
            &=\frac{3.5^{3.5}}{\Gamma(3.5)} \frac{\left(\lambda_{u} / \lambda_{b}\right)^{n}}{n !} \int_{0}^{\infty} s^{n+2.5} e^{-\left(\lambda_{u} / \lambda_{b}+3.5\right) s} d s =\frac{3.5^{3.5} \Gamma(n+3.5)\left(\lambda_{u} / \lambda_{b}\right)^{n}}{\Gamma(3.5) n !\left(\lambda_{u} / \lambda_{b}+3.5\right)^{n+3.5}}
        \end{align*}
        (对不同面积下的用户数量求期望, 然后代入Poisson点过程概率$P(N=k)$, 和 小区平均面积概率密度$f_{2D}$)

    \section{信干噪比}
        Rayleigh衰落信道下,
        * 信号功率:\quad $p = h r^{-\alpha} \quad ,h\sim exp(\mu)$
        * 干扰功率:\quad $I_r = \sum h_i r_i^{-\alpha} \quad$
        * 噪声功率:\quad $\sigma^2$
        

        信干噪比:\quad $SINR = \frac{h r^{-\alpha}}{\sigma^2 + I_r}$\\
        SINR分布: ($\mathbb{P}\left(SINR > T\right)$也称, 容量概率)
        \begin{align*}
            F_{SINR}(T) 
            &= \mathbb{P}\left(SINR \le T\right) & \text{(定义)}\\
            &= \mathbb{P}\left(\frac{P h r^{-\alpha}}{I_r +\sigma^{2}} \leq T\right) & (\text{代入}) \\
            &= \mathbb{P}\left(h \leq \frac{T r^\alpha}{P} (I_r + \sigma^2) \right) & (\text{移项}) \\
            &= \int_{0}^{\infty} \mathbb{P}\left(h \leq \frac{T r^\alpha}{P} (I_r + \sigma^2) \mid I_r \right) f(I_r) \mathrm{~d} I_r & (\text{对干扰功率$I_r$求期望}) \\
            &= \int_{0}^{\infty} 1 - e^{- \frac{\mu T r^\alpha}{P}(I_r + \sigma^2)}  f(I_r) \mathrm{~d} I_r & (h \sim exp(\mu))\\
            &= 1 - e^{- \frac{\mu T r^\alpha \sigma^2}{P}} \int_{0}^{\infty} e^{- \frac{\mu T r^\alpha I_r}{P}} f(I_r) \mathrm{~d} I_r & (\text{移项})\\
            &= 1 - e^{- \frac{\mu T r^\alpha \sigma^2}{P}} \mathcal{L}_{I_{r}}\left(\frac{\mu T r^{\alpha}}{P}\right) & (\text{简写}\mathcal{L}(s) = \int_0^{\infty} f(t) e^{st} \mathrm{d} t)
        \end{align*}
    
