
* 哲学家
	* Θαλῆς   	624BC-548BC
		* 第一个提出"世界本原是什么?"(本体论)
		* 世界的本原是水,水生万物,万物复归于水.
	* Ἀναξίμανδρος	610BC-546BC
		* 世界的本原是ἄπειρον(Apeiron,无定).
	* Ἀναξιμένης	586BC-526BC
		* 世界的本原是水,ἄπειρον,气.
	* Pythagoras	570BC-495BC
		* 世界的本原是数.
	* 老子	----
		* 道家创始人.
		* 无为而治: "道常无为,而无不为"
		* 希望民众能安居乐业,生活于自己的小世界, 没有外界打扰,也没有向外扩张、迁徙,自给自足、稳定的田园理想世界.
			"甘其食,美其服,安其居,乐其俗"  "民至老死,不相往来."
		* 认为民众、万物皆有属于自己演化的道路 “道”,天不会干涉,而是顺应自然.  "天地不仁,以万物为刍狗."
	* 孔子	551BC-479.4.11
		* 儒家创始人.
	* Ἡράκλειτος	535BC-475BC
		* 世界是不断被变化的.  “人不能两次踏入同一条河流”.
		* 世界的本原是火.
	* Ζήνων  	495BC-430BC
		* 二分法 * 阿基里与龟 * 飞矢不动
	* Πρωταγόρας	490BC-420BC
		* 人是万物的尺度.
	* Śākyamuni	----
	* 墨子	476BC-390BC
		* 兼爱: "兼相爱、交相利."  中心思想,主张天下所有人都应当不分高低,彼此相爱.
		* "三表"准则: [1] 事物之本—是否符合天神意志. [2] 事物之原: 是否能加以验证. [3] 事物之用: 是否对民有利.
	* Σωκράτης	469BC-399BC
		* "认识你自己"  "知识的助产士"  通过不断诘问,使他人了解自己的无知,从而进一步认识自己.
		* "美德即知识"
	* Δημόκριτος	460BC-370BC
		* 原子论: 万物由原子构成. 原子Atoms: 物理上不可分割的基本微粒.
		* 万物的本原是原子和虚空
	* Πλατών	427BC-347BC
		* 理念世界 + 现实世界
		* 洞穴寓言
	* Αριστοτέλης	384BC-322.3.7BC
		* 形而上: 对世界本质的研究,研究一切存在的现象(尤指抽象概念)的本源.
	* 孟子	372BC-289BC
		* 民贵君轻: "民为贵, 社稷次之, 君为轻"
			国家由圣人治理,但当统治者违背了人性,缺乏了道德,则民众拥有将君主废除、杀死,进行革命的权力.
			此时, 统治者不再是君主,而只是“独夫”
		* 王道、霸道: [1] 以道德教诲来贯彻 (王道)  [2] 以暴力手段来强制 (霸道)  的两种统治方法
			圣王应遵行王道,以道德来教化民众。而极力避免霸道,来压抑、迫使民众。 
		* 井田制: 认为最健全的经济制度基础是给予农民实现土地分配
			四周八块私田各自耕种,收获归于己,中央公田共同耕种,收获归于国
			"五亩之宅,树之以桑,五十者可以衣帛矣;鸡豚狗彘之畜,无失其时,七十者可以食肉矣"
	* 庄子	369BC-286BC
		* 道家
		* 庄周梦蝶
	* 荀子	313BC-238BC
		* 性恶论, 但可教化: "人之性恶,其善者伪也"  
			也相信人可以根据后天的学习、教化而变得善良、有德行.  "涂之人可以为禹"
			著《劝学》,劝世人要坚持刻苦、善始善终地去求学好学  "学不可以已"
			唯有学习前人哲思,才能净化自我德行,并在前人基础上更进一步.  “青,取之于蓝,而青于蓝”。
	* 韩非子	280BC-233BC
		* 法家
		* 性恶论: 人与人之间只有利益关系,人性皆好为自己谋私利.
			君主予臣子利益,臣子就要效忠君主。父母予子女利益,子女就要赡养父母.
			“产男而相贺,产女而杀之…计之长利也”
		* 君主治国, 在于权术.    ”二柄”
			君主将职务授予官员,而不关心任务如何完成. 官员恪尽职守则奖赏,任务失败则惩戒.
			君主如何知道谁适合官职呢?在于赏罚分明,无能之辈便不敢担任其职责,于是被淘汰
			所以,君主掌握赏罚大权“二柄”,依顺人性来进行治理,便能做到令行禁止,进而统治好国家。 
	* Descartes	1596.3.31-1650.2.11
	* John Locke	1632.8.29-1704.10.28
		* 白板说,人出生时心灵像白板,只是通过经验的途径,心灵中才有了观念,因此,经验是观念的唯一来源.
	* Spinoza 	1632.11.24-1677.2.21
	* George Berkeley	1685.3.12-1753.1.14
	* David Hume	1711.4.2-1776.8.25
	* Rousseau	1712.6.28-1778.7.2
	* Immanuel Kant	1724.4.22-1804.2.12
		* 二元论 不可知论 现代哲学的开端 
		* 人的心灵不是一张白纸,人的心灵有一套自己的形式和结构.
		* 现象是可知的, 本质是不可知的(和柏拉图相反)
		* 空间与时间感性的先天直观形式,是感知事物的先决条件,而不是感知的结果.
			人们可以想象没有任何事物的空的空间和时间、但是却不能想象有某种不在空间和时间中的事物.
			空间与时间是不是通过感性经验而获得的经验概念.
		* 人的直观能力、空间感是先天的.
		* 空间与时间不是物自体的存在方式,而是感性的先天直观形式.
		* 康德坚决反对人有理智直观,他认为人只有感性直观.
		* 人类只有一种直观形式,没有理智直观.
		按照知识必须符合对象的传统观念,无法解决科学普遍性和必然性的问题,
		尝试把主体和客体的关系颠倒,而是客体必须按照主体的认识形式来形成知识,
		* 经验论的基本原则即一一切知识来源于感觉经验.
		* 知识需要有经验,但是只有经验不够,它需要有主体认识形式对感觉经验进行加工整理形成知识.
		* 三大批判: 纯粹理性批判,实践理性批判,判断理性批判
			纯粹理性批判{ 先验感性论 , 先验逻辑论{先验分析,先验辩证论} }
			康德理论: 先验分析论一一关于真理的逻辑    先验辩证论一一关于假象或幻想的逻辑
		* 康德把人理性的认识能力区分为三个环节: 感性、知性、理性
		* 知识的基本单位是判断.
		* 判断: 
			[1]分析判断: 判断的宾词原本就蕴含于主词中,是从主词之中抽出来的. (没有带来新的东西) eg.物体是有广延的.
			[2]综合判断: 宾词是后来通过我们的经验加在主词之上的. eg.物质是热胀冷缩的.
				科学知识不仅要有经验添加的新内容, 而且还必须具有普遍必然性.
			[2.1]后天的综合判断
			[2.2]先天的综合判断
				先天综合判断如何可能?
				* 纯粹数学如何可能?	——先验感性论
				* 纯粹自然科学如何可能?	——先验分析论
				* 一般形而上学如何可能?	——先验辩证论
		* 先验(transzendental): 研究先天认识形式的哲学.
		超验(transzendent): 超越经验之外的不可知领域.
			先验和超验是形而上学的研究内容. 康德将二者区分开.
		* 康德认为数学可以分为几何学和算数.
	* G.W.F.Hegel	1770.8.27-1831.11.14
	* Kierkegaard	1813.5.5-1855.11.11
	* Karl Max	1818.5.5-1883.3.14
	* Max Weber	1864.4.21-1920.6.14
	* Wittgenstein	1889.4.26-1951.4.29
