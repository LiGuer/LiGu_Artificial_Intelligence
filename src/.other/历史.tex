
* 地球史
	* 元古宙 Proterozoic	(~25亿年前-~5.41亿年前)
		* 古元古代 Paleoproterozoic	(~25亿年前-~16亿年前)
			* 成铁纪 Siderian	(~25亿年前-~23亿年前)
				* 大氧化事件	(~24亿年前-~20亿年前)
				* 休伦冰河期	(~24亿年前-~21亿年前)
			* 层侵纪 Rhyacian	(~23亿年前-~20.5亿年前)
			* 造山纪 Orosirian	(~20.5亿年前-~18亿年前)
			* 固结纪 Statherian	(~18亿年前-~16亿年前)
		* 中元古代 Mesoproterozoic	(~16亿年前-~10亿年前)
			* 盖层纪 Calymmian	(~16亿年前-~14亿年前)
			* 延展纪 Ectasian	(~14亿年前-~12亿年前)
			* 狭带纪 Stenian	(~12亿年前-~10亿年前)
		* 新元古代 Neoproterozoic	(~10亿年前-~6.30亿年前)
			* 拉伸纪 Tonian	(~10亿年前-~6.30亿年前)
			* 成冰纪 Cryogenian	(~8.50亿年前-~6.30亿年前)
				* 雪球地球	(~6.5亿年前)
			* 埃迪卡拉纪 Ediacaran (~6.30亿年前-~5.41亿年前)
				* 第一批多细胞动物出现, 埃迪卡拉生物群发展
	* 显生宙 Phanerozoic	(~5.41亿年前-)
		* 古生代 Paleozoic	(~5.41亿年前-~2.519亿年前)
			* 寒武纪 Cambrian	(~5.41亿年前-~4.854亿年前)
				* 寒武纪生命大爆发
			* 奥陶纪 Ordovices	(~4.854亿年前-~4.438亿年前)
				* 奥陶纪末大灭绝
			* 志留纪 Silurian	(~4.438亿年前-~4.192亿年前)
			* 泥盆纪 Devonian	(~4.192亿年前-~3.589亿年前)
			* 石炭纪 Carboniferous	(~3.589亿年前-~2.989亿年前)
				* 石炭纪雨林崩溃事件
			* 二叠纪 Permian	(~2.989亿年前-~2.519亿年前)
				* 二叠纪末大灭绝	(~2.519亿年前)
					* 峨眉山暗色岩事件
					* 西伯利亚暗色岩事件
		* 中生代 Mesozoic	(~2.519亿年前-~0.66亿年前)
			* 三叠纪 Triassic	(~2.519亿年前-~2.013亿年前)
				* 卡尼期洪积事件	(~2.3亿年前)
				* 三叠纪末大灭绝	(~2.013亿年前)
			* 侏罗纪 Jurassic	(~2.013亿年前-~1.45亿年前)
			* 白垩纪 Cretaceous	(~1.45亿年前-~0.66亿年前)
		* 新生代 Cenozoic	(~0.66亿年前-)
			* 古近纪 Paleogene	(~0.66亿年前-~0.2303亿年前)
				* 古新世-渐新世极热事件	(~0.55亿年前)
			* 新近纪 Neogene	(~0.2303亿年前-~258万年前)
			* 第四纪 Quaternary	(~258万年前-)
				* 新仙女木事件	(~1.28万年前)

* 演化史
	* 门类
		* 动物 Animalia
			* 棘头动物 Acanthocephala
			* 环节动物 Annelida
			* 节肢动物 Arthropoda
			* 腕足动物 Brachiopoda
			* 苔藓动物 Bryozoa
			*		   Cephalorhyncha
			* 毛颚动物 Chaetognatha
			* 脊索动物 Chordata
			* 刺胞动物 Cnidaria
			* 栉水母   Ctenophora
			* 环口动物 Cycliophora
			* 菱形虫   Dicyemida
			* 棘皮动物 Echinodermata
			* 螠虫动物 Echiura
			* 内肛动物 Entoprocta
			* 腹毛动物 Gastrotricha
			* 颚胃动物 Gnathostomulida
			* 半索动物 Hemichordata
			* 铠甲动物 Loricifera
			* 微颚动物 Micrognathozoa
			* 软体动物 Mollusca
			* 粘体动物 Myxozoa
			* 线虫动物 Nematoda
			* 线形动物 Nematomorpha
			* 纽形动物 Nemertea
			* 有爪动物 Onychophora
			* 直泳虫 Orthonectida
			* 帚形动物 Phoronida
			* 扁盘动物 Placozoa
				* 丝盘虫 Trichoplax adhaerens Schulze
			* 扁形动物 Platyhelminthes
			* 多孔动物 Porifera
			* 轮虫动物 Rotifera
			* 星虫动物 Sipuncula
			* 缓步动物 Tardigrada
			* 异涡动物 Xenacoelomorpha
		* 古菌 Archaea
			* 泉古菌 Crenarchaeota
			* 广古菌 Euryarchaeota
		* 细菌 Bacteria
			* Acidobacteria
			* Actinobacteria 放线菌
			* Aquificae
			* Armatimonadetes
			* Bacteroidetes
			* Caldiserica
			* Chlamydiae
			* Chlorobi
			* Chloroflexi
			* Chrysiogenetes
			* Cyanobacteria 蓝藻
			* Deferribacteres
			* Deinococcus-thermus
			* Dictyoglomi
			* Elusimicrobia
			* Fibrobacteres
			* Firmicutes
			* Fusobacteria
			* Gemmatimonadetes
			* Lentisphaerae
			* Nitrospira
			* Planctomycetes
			* Proteobacteria
			* Spirochaetae
			* Synergistetes
			* Tenericutes
			* Thermodesulfobacteria
			* Thermotogae
			* Verrucomicrobia
		* Chromista
			* Bigyra
			* Ciliophora
			* Cryptista
			* Foraminifera
			* Haptophyta
			* Miozoa
			* Ochrophyta
			* Oomycota
			* Radiozoa
		* 真菌 Fungi
			* 子囊菌 Ascomycota
			* 担子菌 Basidiomycota
			* 壶  菌 Chytridiomycota
			* 球囊菌 Glomeromycota
			* 接合菌 Zygomycota
		* 植物 Plantae
			* 角苔	 Anthocerotophyta
			* 苔藓植物 Bryophyta
			* 轮藻	 Charophyta
			* 绿藻	 Chlorophyta
			* 灰藻	 Glaucophyta
			* 地钱	 Marchantiophyta
			* 红藻	 Rhodophyta
			* 维管植物 Tracheophyta
		* 原生动物 Protozoa
			* Amoebozoa
			* Cercozoa
			* Choanozoa
			* Euglenozoa
			* Loukozoa
			* Metamonada
			* Microsporidia
			* Mycetozoa
			* Percolozoa
			* Sarcomastigophora
			* Sulcozoa
			* assigned Not
		* 病毒 Viruses
			* assigned Not


	* 演化树
		* 聚胞动物
			* 领鞭毛虫
			* 动物 Animalia

		* 动物 Animalia
			* 多孔动物 Porifera
				* 海绵
			* 真后生动物 Eumetazoa
				* 栉水母 Ctenophora
				* 
					* 
						* 扁盘动物 Placozoa
						* 刺胞动物 Cnidaria
					* 两侧对称动物 Bilateria
						* 异涡动物 Xenacoelomorpha
						* 肾管动物 Nephrozoa
							* 原口动物 Protostomia
								* 螺旋动物 Spiralia
									* 触手冠动物
										* 腕足动物 Brachiopoda
											* 海豆芽 Lingula
										* 外肛动物
										* 软舌螺
									* 软体动物 Mollusca
										* 腹足纲
											* 螺
										* 双壳纲
											* 贝
										* 头足纲
											* 角石
												* 菊石
												* 鹦鹉螺
												* 蛸类
													* 章鱼
													* 鱿鱼
										* 掘足纲
									* 环节动物 Annelida
								* 
									* 毛颚动物 Chaetognatha
									* 蜕皮动物 Ecdysozoa
										* 线虫动物 Nematoda
										* 节肢动物 Arthropoda
											* 奇虾
											* 三叶虫
											* 螯肢动物 Chelicerata
												* 鲎
												* 蜘蛛
												* 蜱螨亚纲
											* 有颚亚门 Mandibulata
												* 多足亚门 Myriapoda
												* 六足亚门 Hexapoda
													* 昆虫
														* 鞘翅目
														* 双翅目
														* 膜翅目
															* 蜂
																* 蚂蚁
														* 鳞翅目
															* 蝴蝶
														* 脉翅目
															* 丽蛉
												* 甲壳动物
													* 虾
														* 螃蟹
							* 后口动物
								* 脊索动物 Chordata
									* 头索动物 Leptocardii
									*
										* 被囊动物
											*   海鞘 Ascidiacea
											* 樽海鞘 Thaliacea
											* 尾海鞘 Appendicularia
										* 脊椎动物
											* 圆口鱼
												* 盲  鳗 Myxini
												* 七鳃鳗
											* 头甲鱼 Cephalaspidomorphi
											* 有颌鱼
												* 软骨鱼 Holocephali
													* 鲨鱼
												* 硬骨鱼
													* 辐鳍鱼 Actinopterygii
													* 肉鳍鱼 Sarcopterygii
														* 两栖动物 Amphibia
														* 羊膜动物
															* 合弓纲
																* 哺乳动物 Mammalia
																	* 啮齿动物
																		* 鼠
																	* 灵长类
																		* 人
																		* 猩猩
																		* 猴子
																	* 蝙蝠
																	* 奇蹄目
																		* 马
																		* 犀
																		* 貘
																	* 偶蹄目
																		* 牛
																		* 羊
																		* 猪
																		* 鲸
																		* 河马
																	* 食肉目
																		* 猫
																			* 猫
																			* 狮子
																			* 老虎
																			* 豹
																		* 犬
																			* 狼
																			* 狗
															* 蜥形纲
																* 鳞龙形类
																* 主龙形类
																	* 翼龙
																	* 鸟 Aves
																		* 古颚类
																			* 鸵鸟
																			* 恐鸟
								* 步带动物
									* 棘皮动物 Echinodermata
										* 海百合
										* 海扁果
										* 海蕾
										* 海胆
											* 海参
											* 海胆
										* 海星
											* 海蛇尾
											* 海星
									* 半索动物 Hemichordata

* 人类史
	* 时间序列
		1582.6.21	本能寺の変, 明智光秀叛杀织田信长于本能寺.
		1853	日本, 黑船事件, 美将Perry率铁甲军舰, 武力威胁幕府开国.
		1868-1890	日本明治维新
		1894.7.25-1895.4.17	甲午中日战争
		1904.2.8-1905.9.5	日俄战争
		1917.11.7	俄国十月革命
		1927.8.1	南昌起义
		1931.9.18	柳条湖事件, 日本占领东北三省
		1933.1.30	Adolf Hitler上台
		1934.10-1936.10	长征
		1936.3.7	进军Rhineland
		1936.2.26	日本二二六兵变
		1937.7.7	卢沟桥事变, 日本全面入侵中国
		1938.3.12	德奥合并
		1938.9.30	Munich协定, Czechoslovak割让Sudetenland给德国
		1939.8.23	苏德互不侵犯条约
		1939.9.1-1945.9.2	第二次世界大战
		1939.9.1-1939.10.9	波兰战役, 德国闪击败波兰
		1940.4.9	威瑟堡行动, 德国击败挪威和丹麦
		1940.5.10-1940.6.22	法国战役, 德国闪击法国
		1940.5.26-1940.6.4	敦刻尔克大撤退
		1941.6.22-1945.5.9	苏德战争
		1941.6.22	巴巴罗萨行动, 德国入侵苏联
		1941.9.8-1944.1.27	列宁格勒战役
		1941.9.30-1942.1.7	莫斯科保卫战
		1941.12.7-1945.8.15	日美太平洋战争
		1941.12.7	日本偷袭美国珍珠港
		1942.6.4-1942.6.7	中途岛战役
		1942.7.17-1943.2.2	斯大林格勒战役
		1942.10.23-1942.11.3	阿拉曼战役
		1943.7.5-1943.8.27	库尔斯克战役
		1944.6.6	诺曼底登陆
		1944.12.16-1945.1.25	阿登战役
		1945.4.16-1945.5.9	柏林战役, 苏联占领德国柏林
		1945.3.10,1945.5.25	东京大轰炸
		1945.5.9	德国二战投降
		1945.8.6,1945.8.9	核爆日本, 广岛、长崎原子弹
		1945.8.8-1945.9.2	八月风暴行动, 苏联对日本作战
		1945.8.15	日本投降
		1946.6-1950.6	中国解放战争
		1948.9.12-11.2	辽沈战役
		1948.11.6-1949.1.10	淮海战役
		1948.11.29-1949.1.31	平津战役
		1950.6.25-1953.7.27	朝鲜战争

	* 朝代史
		* 中国
			仰韶文化	5000BC-3000BC
			龙山文化	3000BC-2000BC
			夏	2070BC-1600BC	禹
			商	1600BC-1046BC	汤,纣
			西周	1046-771BC(275)	周武王姬发
			东周	770-221BC(549)	周平王姬宜臼
			春秋	770-476BC(295)	
			战国	475-221BC(254)	
			秦	221-206BC(16)	秦始皇嬴政
			西楚	206-202BC(5)	项羽
			西汉	202BC-8(210)	刘邦
			新	8-23(16) 	王莽
			东汉	25-220(195)	刘秀
			魏	220-265(45)	曹操,魏文帝曹丕
			蜀	221-263(42)	刘备
			吴	222-280(58)	孙权
			西晋	265-316(51)	晋武帝司马炎
			东晋	317-420(103)	晋元帝司马睿
			十六国	304-439(135)	
			南北朝	420-589(169)
			隋	581-619(38)
			唐	618-907 	唐高祖李渊,唐太宗李世民,武则天,唐玄宗李隆基
			五代十国	891-979(89)
			北宋	960-1127(167)	宋太祖赵匡胤
			南宋	1127-1279(152)	宋高宗赵构
			辽	916-1125(210)	耶律阿保机
			西夏	1038-1227(190)	李元昊
			金	1115-1234(120)	完颜阿骨打
			大理国	937-1253	段思平
			元	1206-1368	成吉思汗,元世祖忽必烈
			明	1368-1644(277)	明太祖朱元璋
			清	1616-1912	清太宗皇太极
			中华民国	1912-1949(38)	孙中山,蒋介石
			中华人民共和国	1949-	毛泽东

		* 日本
			绳文時代	xxBC-300BC
			弥生時代	300BC-250	卑弥呼
			古坟時代	250-592
			飞鸟時代	673-710
			奈良時代    710-794
			平安時代	794-1192	桓武天皇
			鎌倉幕府	1192-1333	源赖朝
			建武新政	1333-1336
			南北朝時代	1336-1392
			室町時代	1336-1573
			战国時代	1467-1603	
			安土桃山時代	1573-1590	织田信长,丰臣秀吉
			江户時代	1603-1868	德川家康
			明治时代	1868-1912	
			(现代)
		
