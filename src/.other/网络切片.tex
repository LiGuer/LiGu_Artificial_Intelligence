* 网络切片定义和原则
	* 网络切片定义和意义
		\def{5G网络切片}{
			是一种网络架构, 利用虚拟化技术的优势, 支持在同一通用物理网络基础设施平台上建立多个逻辑独立且相互隔离的端到端的子网络(即网络切片) , 从而实现为不同性能需求的特定应用场景定制其专门的解决方案。
		}
	   
		\bf{意义}
			传统4G网络"一刀切"的架构方法, 无法满足从覆盖全球的通用宽带接入到传感器或极端移动性的专用网络的应用场景, 以及不同垂直行业和特定应用场景中在延迟、可扩展性、可用性和可靠性方面的不同性能要求。通过将一整个同一的物理网络切割成多个虚拟的相互隔离的逻辑子网切片, 让每个网络切片都是一个隔离的端到端专用网络, 使运营商能够创建网络, 定制为需要不同要求的不同市场场景提供优化的解决方案, 便能解决这个问题。

			在NGMN组织(Next Generation Mobile Networks, NGMN)编写的\textit{5G白皮书}的“5G架构”一章中, 对5G网络切片的定义进行了详细的描述和解释: “\textit{利用软件和硬件之间的结构分离, 以及软件定义网络SDN和网络资源虚拟化NFV提供的可编程性, 因此, 5G体系结构是一种本地的SDN/NFV体系结构, 涵盖了从设备、(移动/固定) 基础设施、网络功能、价值实现能力到协调5G系统的所有管理功能的各个方面}”、“\textit{一种网络切片, 被命名为'5G切片', 以特定的方式处理特定连接类型的通信业务的C-平面和U-平面。为此, 5G片由一系列5G网络功能和特定RAT配置组成, 这些功能和RAT配置结合在一起, 用于特定的用}

		
	* 网络切片服务的三大场景
		国际电信联盟(ITU) 和第五代公私合作伙伴关系(5G-PPP) 对于5G技术可能应用到的案例,划分了三大类广泛的应用场景:

		* \bf{增强型移动宽带}(Enhanced Mobile Broadband, eMBB), 5G非独立部署的初始阶段侧重于eMBB, eMBB提供了更大的数据带宽用来服务数据驱动型大流量增强移动宽带业务, 诸如增强现实及虚拟现实、3D超高清视频流传输、可感知的互联网等。eMBB的预计实现目标是热点地区的流量为每平方米10Mbps,用户经历的数据传输速率高达1Gbps,峰值数据传输速率在数十Gbps,整个流量至少为每平方公里1Tbps,用户体验数据交换的延迟为 1 毫秒,连接密度高达每平方公里100万个连接,高速列车的机动性高达500公里/小时,飞机时速可达1000公里/小时,用户体验增强等。

		* \bf{超高可靠低时延通信}(Ultra-Reliable Critical Low Latency Communication Services, URLLC), 用来服务对任务关键型通信的延迟和可靠性有严格的要求的业务, 诸如无人驾驶、远程医疗、工业自动化等。主要特征是 (1) 高可靠, 网络保持稳定性, 保证在运行的过程中不受干扰。(2) 低延时, 可支持低至5ms的端到端延迟。

		* \bf{海量机器类通信}(Massive Machine Type Communication, mMTC), 用来服务大规模超密集的传感器等物联网业务, 帮助物联网设备突破人与人之间的通信以实现人与机器、机器与机器的通信。物联网设备具有低成本、低功耗、低范围的特点,(1) 低功耗, 物联网无法使用固定电源供电只能使用电池, 同时物联网尽在特定时间的数据采集和发送需要能源,其余绝大部分时间可处于休眠状态。 (2) 海量接入, 支持数十亿级的物联网终端设备的接入,这对接入网及其信令调度的负担是一个挑战。

		因为业务形式的不同, 业务对网络要求的侧重点也不尽相同, Xenofon Foukas, Georgios Patounas在文章中给出这三类主要业务场景对通信性能指标需求的雷达统计图, 如图2所示。从图中我们可以清晰地看到mMTC仅仅对连接密度有较强的需求, 而对其他通信指标要求较弱,因为mMTC典型特点就在海量机器的连接密度上。而URLLC对时延、移动性、可靠性需要较强,因为无人驾驶、远程医疗等任务关键型业务不能容忍通信连接的中断,当手术过程中和无人汽车驾驶中任何通信中断都会瞬间造成不可挽回的人身健康和财产的重大损失,也因此当通信出现困难时URLLC服务可以通过抢占eMBB传输来减少下行链路延迟以保证自身的通信服务质量。由此也可以明白,5G业务具有很强的异构性和性能需求差异,需要网络切片技术帮助其调度来解决这些问题。
		
	* 网络切片的原则
		网络切片建立在以下几条原则之上: 
		* \bf{自动化}: 支持按需配置网络切片, 无需固定的合同协议和手动干预, 其依赖于基于信令的机制, 该机制允许第三方发出切片创建请求, 除了传统的SLA之外, 该请求还将反映所需的容量、延迟、抖动等, 考虑开始和结束时间的计时信息以及网络切片的持续时间或周期性。

		* \bf{隔离}: 是网络切片的基本属性, 可确保每个租户的性能保证和安全性(以捍卫网络对第三方的开放性) , 即使不同的租户将网络切片用于具有冲突性能要求的服务也是如此。但是, 隔离可能以降低多路复用增益为代价, 具体取决于显式使用的资源分离方式, 这可能导致网络资源利用率低下。隔离的概念不仅涉及数据平面, 还涉及控制平面, 而其实现定义了资源分离的程度。可以部署隔离: (1)  使用不同的物理资源, (2)  当通过虚拟化分离意味着共享资源时, 以及 (3) 通过在定义每个租户访问权限的相应策略的指导下共享资源来部署隔离。

		* \bf{自定义}: 确保分配给特定租户的资源得到有效利用, 以便最好地满足相应的服务要求。可以实现切片定制 (1) 在考虑抽象拓扑以及数据与控制平面分离的网络范围, (2) 在具有服务定制网络功能和数据转发机制的数据平面上, (3) 在控制平面上引入可编程策略, 操作和协议, 以及(4) 通过大数据和上下文感知等增值服务。

		* \bf{弹性}: 是与分配给特定网络片的资源相关的基本操作, 以确保在不断变化的(1) 无线电和网络条件下所需的SLA, (2) 服务用户的数量, 或(3) 由于用户移动性而导致的地理服务区域。这种资源弹性可以通过通过扩展/缩减或重新定位VNF和增值服务来重塑已分配资源的使用, 或者通过调整应用的策略并重新编程某些数据和控制平面元素的功能来实现。弹性还可以采取通过修改物理和虚拟网络功能来改变最初分配的资源量的形式, 例如, 通过添加不同的RAN技术或新的VNF, 或者通过增强无线电和网络容量。但是, 此过程需要切片间协商, 因为它可能会影响共享相同资源的其他切片的性能。

		* \bf{可编程性}: 允许第三方通过公开网络功能的开放API控制分配的切片资源, 即网络和云资源, 从而促进面向服务的按需定制和资源弹性。

		* \bf{端到端}: 是网络切片的固有属性, 用于促进从服务提供商到最终用户/客户的全程服务交付。端到端属性有两个扩展: (1) 跨越不同的管理域, 即一个结合了属于不同基础架构提供商的资源的切片, 以及(2) 统一了各种网络层和异构技术, 例如, RAN、核心网络、传输和云。特别是端到端网络切片整合了各种资源, 实现了覆盖的服务层, 这为高效的网络和服务融合提供了新的机会。

		* \bf{分层抽象}: 是网络切片的一种属性, 其根源在于递归虚拟化, 其中资源抽象过程在分层模式上重复, 每个级别都连续更高, 从而提供具有更广范围的更大抽象。换句话说, 分配给特定租户的网络切片的资源可以进一步部分或全部交易给另一个第三方玩家, 这与网络切片租户以这种方式促进前一个网络切片服务之上的另一个网络切片服务有关。例如, 从基础结构提供商处获取网络切片的虚拟移动运营商提供部分此类资源, 以使使用其虚拟网络的实用工具提供商能够形成 IoT 切片。


* 网络切片的架构

	NGMN对于5G网络切片架构提倡灵活的软加密网络方法, 并主张端到端切片的范围包括无线接入网RAN和核心网CN。整个NGMN体系结构分为三层: 基础架构资源、业务支持和业务应用程序。而5G-PPP对于5G网络切片架构的与观点NGMN一致,潜在的5G架构必须支持本地软件化,并利用切片来支持不同的应用场景,只是编排/MANO被视为单独一层。5G-PPP的架构方案分为五层: 基础设施层、网络功能层、编排层、业务功能层和服务层。

	* \bf{基础设施层}: 广义上是指跨越RAN和CN的物理网络基础设施, 包括基础设施的部署、控制和管理;将资源(计算、存储、网络、无线)分配给片;以及这些资源向上层展示和管理的方式。相关工作侧重于两个主要主题: 网络基础架构的组成及其虚拟化。(1) 网络基础设施的构成方面,对于核心网提出了中央和边缘云计算基础设施的混合,其中资源可以根据切片要求分配给其中任何一个,对于无线接入网提出了部署由集中式基带处理单元和远程无线电头组成的通用软件定义基站。(2) 虚拟化方面,核心网的虚拟化可以利用基于内核的虚拟机(KVM) 和 Linux 容器(LXC) 等技术可以在操作系统(OS) 或进程级别提供处理、存储和网络资源方面的隔离保证,且与OpenStack等资源池化的功能相结合以简化虚拟化核心网的动态创建,例如虚拟核心网络功能-虚拟移动性管理实体(MME) ,虚拟服务网关(SGW) 等。无线接入网虚拟化,分为为每个虚拟基站(片) 提供专用的频谱块,以在其上部署完整的虚拟网络堆栈、通过采用通用的底层物理和低介质访问控制 (MAC)  层,在不同的虚拟基站实例(切片) 之间动态共享频谱,这两种方法。

	* \bf{网络功能层}: 封装了链接在虚拟基础架构上与网络功能配置和生命周期管理相关的所有操作, 以提供端到端服务和满足网络切片服务设计中的约束和要求。研究主要围绕可以推动网络功能部署和管理的技术, 与有关已部署功能的粒度和类型的问题。软件定义网络SDN和网络资源虚拟化NFV是其关键技术,分别解决网络功能生命周期管理与编排、标准化协议配置与控制路由及转发平面两个部分。

	* \bf{服务层 和 管理及网络编配MANO}: 5G网络切片与其它切片的主要区别是端到端性质, 以及通过高级描述表达服务并将其灵活映射到适当的基础设施元素和网络功能的要求。由此产生两个新概念: (1)直接链接到创建网络切片背后的业务模型的服务层; (2)用于切片生命周期超视觉的网络切片编排。由于该层在概念和思想方面引入了新颖性, 因此该领域的相关研究自然侧重于回答有关网络切片架构的基本问题。更具体地说, 所考虑的主题与描述服务的方式以及如何将它们映射到底层网络组件以及网络切片管理器和业务流程协调程序的体系结构有关。
	

	
* 网络切片的关键技术
	* 软件定义网络
		软件定义网络(Software Defined Network, SDN), 是一种网络设计理念, 区别于传统网络中的各个路由转发节点各自为政、独立工作的现状, 只要是将控制平面与数据平面分离并通过逻辑集中式可编程化软件管理的网络智能, 则可以认为这个网络是一个SDN网络。
		
		SDN起源于2006年斯坦福大学的Clean Slate研究课题, Mckeown教授于2009年正式提出了SDN概念。SDN具有灵活性、面向服务的适应性、可扩展性和稳健性等优势。引入了中枢控制节点---控制器, 用来统一指挥下层设备的数据发送, 以实现控制和转发分离, 使得网络灵活性和可扩展性大为增强,其抽象的网络视图也为第三方和多租户支持提供了便利。

		开发研究中心(Open Networking Research Center, ONRC) 对SDN技术描述到:  “\textit{SDN正在改变我们设计、构建和运营网络的方式, 以更好地支持增长、灵活性和创新。SDN的关键属性包括: 数据和控制平面的分离;控制平面和数据平面之间统一的、与供应商无关的接口, 称为OpenFlow;以及逻辑上集中的控制平面, 为编程人员或操作员提供一致的、网络范围的视图。逻辑集中控制平面是使用网络操作系统实现的, 该网络操作系统构造并呈现整个网络到在其上实现的服务或控制应用程序的逻辑映射。通过SDN, 网络运营商或第三方可以通过编写一个简单的软件程序来操纵网络切片的逻辑图, 从而引入新服务或定制网络行为。其余部分由网络操作系统负责。SDN的这一观点支持无限创新, 具有真正的模块化体系结构, 允许提供商、运营商和最终用户混合和匹配他们希望最好地满足其需求的应用程序、网络操作系统、虚拟机监控程序和交换机。网络管理员最终可以运行应用程序来管理其网络, 并为其组织提供独特的性能、规模、服务和功能需求。}”

		* 软件定义网络对网络切片的作用
			SDN控制器提供了用于建立切片的资源和控制逻辑的抽象集合,并且切片可以被视为SDN客户端上下文。因此,SDN有助于切片蓝图的预定义,以及根据相应的业务特征和需求按需创建切片实例。
	
	* 网络资源虚拟化
		网络资源虚拟化(Network function virtualization, NFV), 随着通用服务器处理能力的大幅增强, 有余力拿出一部分资源作为虚拟化层, 把网络中的计算、存储、网络等资源进行统一管理, 按需划分。多台物理服务器的硬件形成了资源池, 可以按照需要划分成若干逻辑服务器, 供各种应用来使用。虚拟化是网络切片的关键过程,可以在切片之间实现有效的资源共享。虚拟化是使用适当的技术对资源进行抽象。

		* 网络资源虚拟化对网络切片的作用
			NFV架构可以有效管理网络切片及其组成资源的生命周期,NFV 架构由网络功能虚拟化基础架构、断续器、管理和编排(MANO)(包括虚拟化基础架构管理器,VNF 管理器和Orchestrator)、网络管理系统、基础设施 SDN 控制器、租户 SDN 控制器共同构成。
			
			NFV在实现网络切片方面补充了SDN,因为SDN建立了支持切片的控制平面功能,而NFV通过实现VNF来提供服务和管理网络切片的生命周期并编排切片资源.

	* 云和边缘计算
		云计算是通过网络交付计算服务(包括服务器、存储、数据库、网络、软件、分析和智能) ,以提供更快的创新、灵活的资源和规模经济。其理想是能够想电力一样将计算生产设备集中起来并能够随时随地向用户提供其所需的计算力。帮助公司避免或最小化前期IT基础架构成本。

		云服务类型分为基础架构即服务IaaS、平台即服务PaaS、无服务器计算和软件即服务SaaS,IaaS是最基本的云计算服务类别以从云提供商处以即用即付的方式租用 IT 基础架构(服务器和虚拟机 (VM) 、存储、网络、操作系统) 。PaaS指为开发、测试、交付和管理软件应用程序提供按需环境的云计算服务,使开发人员能够更轻松地快速创建 Web 或移动应用程序,而不必担心设置或管理开发所需的底层基础结构。无服务器计算与 PaaS 重叠,无服务器计算侧重于构建应用功能,而无需花费时间持续管理执行此操作所需的服务器和基础结构。SaaS是一种通过 Internet 按需交付软件应用程序的方法,通常以订阅方式交付。借助 SaaS,云提供商可以托管和管理软件应用程序和底层基础架构,并处理任何维护,例如软件升级和安全补丁。

		边缘计算是一种分布式计算方法,它使计算和数据存储更接近数据源,从而缩短响应时间并节省带宽。因为云计算无法集中式解决所有庞大的计算服务而提出了边缘计算来缓解。与其同义的还有雾计算。


* 网络切片的挑战
	* \bf{资源与安全隔离}, 服务质量上需要控制和避免某个切片中的业务突发或异常流量影响到同一网络中的其他切片,做到不同网络切片内的业务之间互不影响, 能够提供确定性时延。传输安全上,网络切片中的业务或用户信息不希望被其他网络切片的用户访问或者获取。隔离类型分为业务隔离、资源隔离、运维隔离。对于资源隔离,是某一网络切片所使用的网络资源与其他网络切片所使用的资源之间相互隔离。分为硬隔离和软隔离,硬隔离是指为不同的网络切片在网络中分配完全独享的网络资源;软隔离是指不同的网络切片既拥有部分独立的资源,同时对网络中的另一些资源也存在共享,从而在提供满足业务需求的隔离特性的同时也可保持一定的统计复用能力。

	* \bf{切片安全与隐私}, 由于软件化网络可编程性的开放接口、提供切片资源的技术域之间的交互层不同、租户共享资源以及网络切片和托管租户之间存在的不同暴露级别,如何保障用户的安全和隐私是一个重要难题。有研究建议使用额外的定量或定性参数来区分单个切片所需的安全级别作为提供保障的等级指标。

	* \bf{切片设计优化问题}, 网络切片中面临的最困难的挑战之一是如何以最佳方式描述、定义和动态调整网络切片模板(即动态切片资源分配和最佳VNF配置) 。传统的虚拟网络通过静态分配一组预期数量的网络资源,时常会预配过多或不足所以不是最佳方法。如何动态地网络切片资源扩展,以最佳方式为所有用户提供服务,避免其他任何活动网络切片的资源量不足使得性能产生负面影响,是一个重要挑战。

	* \bf{定价计费}, 5G网络切片的定价计费可以取代传统的按流量计费方式,如何确定最优的定价策略能够帮助运营商和网络切片的销售商获得更可观的收入和销量。

	* \bf{用户终端切片}, 5G网络应根据其各自的特征和类型来处理UE,即每个UE都不应再连接到单一尺寸的网络,而应连接到专门为该类型UE创建的自定义片,从而给用户带来更多的自由、定制和广泛的应用程序可用性。

	* \bf{管理和编排问题}, 鉴于切片带来的动态性和可伸缩性,为了灵活地将动态资源分配到切片,管理的优化策略必须处理资源需求在相对较短的时间尺度内变化很大的情况,使得多租户方案中的管理和编排成为一个挑战。

	* \bf{自动化切片管理}: 业务种类和规模持续增加,网络管理复杂度快速增长,难以再依赖人工管理.

	* \bf{灵活定制拓扑连接}: 提供按需定制的逻辑网络, 用户无需感知基础网络的全量拓扑,只需看到自己网络切片的逻辑拓扑与连接. 运营商避免将基础网络内部信息暴露,提高安全性。

	
* 网络切片的其他技术
	* 随机几何, Vincenzo Sciancalepore, Marco Di Renzo等人$^{[5]}$利用随机几何方法, 对网络切片的基站分布进行建模并计算其信道容量和信干噪比等通信物理量, 将其作为网络性能度量的基础指标并进一步利用凸优化方法对带宽和功率的资源分配的性能进行计算, 验证了网络切片带宽和功率的合理分配能够提高资源利用率, 同时也给出了达到最优性能的切片策略。
		   
	* 强化学习和深度学习, Mu Yan, Gang Feng在论文\textit{Intelligent Resource Scheduling for 5G Radio Access Network Slicing}$^{[7]}$以协作方式利用"深度学习"和"强化学习"来解决无线接入网上5G网络切片的资源调度问题,预测和在线决策模块之间的重要性可以动态调整,以协助 RAN 做出准确决策,使得收敛性满足在线资源调度要求,在保证片间性能隔离的同时,能够显著提高资源利用率。

	  
	
\section{参考文献}
	* Ordonez-Lucena J, Ameigeiras P, Lopez D, et al. Network Slicing for 5G with SDN/NFV: Concepts, Architectures and Challenges[J]. IEEE Communications Magazine, 2017, 55(5):80-87.
	* NGMN 5G White Paper, NGMN Alliance, Frankfurt, Germany, Feb. 2015.
	* Foukas X, Patounas G, Elmokashfi A, et al. Network Slicing in 5G: Survey and Challenges[J]. IEEE Communications Magazine, 2017, 55(5):94-100.
	* Shunliang Zhang.An Overview of Network Slicing for 5G[J]. IEEE Wireless Communications, 2019, PP(3):1-7.
	* Sciancalepore V, MD Renzo, Costa-Perez X. STORNS: Stochastic Radio Access Network Slicing[C]. IEEE, 2019.
	* Andrews J G, Baccelli F, Ganti R K. A Tractable Approach to Coverage and Rate in Cellular Networks[J]. 2010.
	* Yan M, Feng G, Zhou J H, et al. Intelligent Resource Scheduling for 5G Radio Access Network Slicing[J]. IEEE Transactions on Vehicular Technology, 2019, 68(99):7691-7703.
	* David Silver, Guy Lever. Deterministic Policy Gradient Algorithms[J].In Proceedings of the 31st International Conference on International Conference on Machine Learning - Volume 32 (ICML'14). JMLR.org, I–387–I–395.
	* Mnih V, Kavukcuoglu K, Silver D, et al. Playing Atari with Deep Reinforcement Learning[J]. Computer Science, 2013.
	* Lillicrap T P, Hunt J J, Pritzel A, et al. Continuous control with deep reinforcement learning[J]. Computer ence, 2015.
	
