
* 几何光学
    * 基本定律
		* 光沿直线传播.
		* 光路可逆.
		* 两束光传播中互不干扰,会聚于同一点时光强简单相加.
		* 介质分界面的光传播规律.

	* 介质分界面光传播规律
		* 反射
			\Theorem{反射定律} 入射角 = 反射角. $\theta_I = \theta_O$
			
			\bf{算法}
				设面矢$\. F$, 入射光$L$
				$\. L_O = \. L_I - \. F·2 \cos \theta_I$
				
		* 折射
			\Theorem{折射定律} $n_I·\sin \theta_I = n_O·\sin \theta_O$
				
			\bf{算法}
				设面矢\. F, 入射光L $n = \frac{n_I}{n_O}, \theta_I = <\. L,\. F>$
				$
					\. L_O &= \. L_I + \. F·\frac{\sin(\theta_O - \theta_I)}{\sin\theta_O}
					&= \. L_I + \. F· \frac{\cos\theta_I·\sin\theta_O - \cos\theta_O·\sin\theta_I}{\sin\theta_O}
					&= \. L_I + \. F· (\cos\theta_I - \frac{\cos\theta_O}{n})
					&= \. L_I + \. F· (\cos\theta_I - \frac{\sqrt{1 - n^2·\sin^2\theta_I}}{n} )
				$
	

* 外接圆
	* 外接圆半径
		\bf{Heron-秦九韶公式}
			$S = \sqrt{p(p-a)(p-b)(p-c)}$
			$p = \frac{a+b+c}{2}$

	* 外接圆圆心

		\bf{算法}
			* 判断四点共圆
				*	三点确定圆方程: 即 解行列式:
					$ |\mb
						x^2+y^2 & x & y & 1 \\
						x_1^2+y_1^2 & x_1 & y_1 & 1 \\
						x_2^2+y_2^2 & x_2 & y_2 & 1 \\
						x_3^2+y_3^2 & x_3 & y_3 & 1
					\me| ?= 0
					$
				* 几何解释: 通过把平面点提升到三维的抛物面中,由于抛物面被平面所截的截面为圆形,四点共面即使共圆,也可以用四面体的体积判断是否共圆。

			* 平面三点确定圆方程
				* 圆方程: (x - cx)^2 + (y - cy)^2 = R^2
				* 三点确定圆方程: 即 解行列式:
					$ |\mb
						x^2+y^2 & x & y & 1 \\
						x_1^2+y_1^2 & x_1 & y_1 & 1 \\
						x_2^2+y_2^2 & x_2 & y_2 & 1 \\
						x_3^2+y_3^2 & x_3 & y_3 & 1
					\me| = 0
					$
					即.目标三点和圆上(x,y)应该满足方程组:
					$(x^2+y^2)·a + x·b + y·c + 1·d = 0$

				\Proof
					$
						M_{11}(x^2+y^2) - M_{12} x + M_{13} y - M_{14} = (x^2+y^2)·a + x·b + y·c + 1·d = 0
						=> (x^2 + b/a x) + (y^2 + c/a y) = - d/a
						=> (x + b/2a)^2 + (y + c/2a)^2 = -d/a + b^2/4a^2 + c^2/4a^2
					$
											
			* CircumCircle 三角形外接圆
				外接圆圆心: 即. 三点确定圆方程问题, 也是任意两边的垂直平分线的交点.直接用平面三点确定圆方程方法
            
* 交点
    * 射线 \& 平面交点
		平面方程: $Ax + By + Cz + D = 0$
		直线方程: $(x - x0) / a = (y - y0) / b = (z - z0) / c = K$
		联立平面,直线方程, 解K, K即直线到射线平面交点的距离
		$
			x = K a + x0 , ...
			A(K a + x0) + B(K b + y0) + C(K c + z0) + D = 0
			=> K =  - (A x0 + B y0 + C z0) / (A a + B b + C c)
		$

    * 射线 \& 三角形交点
		射线: $P = O + t D$
		射线三角形交点:
		$
				O + t D = (1 - u - v)V0 + u V1 + v V2
				[ -D  V1-V0  V2-V0] [ t  u  v ]' = O - V0
				T = O - V0    E1 = V1 - V0    E2 = V2 - V0
				[ -D  E1  E2 ] [ t  u  v ]' = T
				t = | T  E1  E2| / |-D  E1  E2|
				u = |-D   T  E2| / |-D  E1  E2|
				v = |-D  E1  E2| / |-D  E1  E2|
		(混合积公式): |a  b  c| = a × b·c = -a × c·b
				t = (T × E1·E2) / (D × E2·E1)
				u = (D × E2· T) / (D × E2·E1)
				v = (T × E1· D) / (D × E2·E1)
        $

	* 射线 \& 球面交点
		球方程: $(X - Xs)^2 + (Y - Ys)^2 + (Z - Zs)^2 = R^2$
		线球交点: $K^2(Al^2 + Bl^2 + Cl^2) + 2K(Al ΔX + Bl ΔY + Cl ΔZ) + (ΔX^2 + ΔY^2 + ΔZ^2 - R^2) = 0$
		$
				  ΔX = Xl - Xs
				  Δ = b^2 - 4ac = 4(Al ΔX + Bl ΔY + Cl ΔZ)^2 - 4(Al^2 + Bl^2 + Cl^2)(ΔX^2 + ΔY^2 + ΔZ^2 - R^2)
				  若Δ≥0 有交点.
				  K = ( -b ± sqrt(Δ) ) / 2a	即光线走过线距离
		$
		



* 凸包
	\bf{算法}
		* Graham 扫描法
			\Property
				* 时间复杂度 O(n logn)

			\bf{流程}
				* 选择y最小点 p0, 若多个则选其中x最小

				* sorted by polar angle in counterclockwise order around p0 (if more than one point has the same angle, remove all but the one that is farthest from p0)

					* 几何可知,排序后 P1 和最后一点 Pn-1 一定是凸包上的点

				* P0,P1 入栈S,P2 为当前点, if n<2, error "Convex Hull is empty"

				* 遍历剩余点 P3 -> Pn-1
					* while the angle formed by points NEXT-TO-TOP(S),TOP(S),and p makes a nonleft turn, POP(S)
					* Pi 入栈

				* 最后栈中元素,即结果




* Delaunay 三角剖分
 	\def{Delaunay三角剖分} 每个三角形的外接圆内不包含V中任何点

	 \bf{流程}
		* 将点按坐标x从小到大排序
		* 确定超级三角形, 将超级三角形保存至未确定三角形列表 trianglesTemp
		* 遍历每一个点
			* 初始化边缓存数组 edgeBuffer
			* 遍历 trianglesTemp 中的每一个三角形
				* 计算该三角形的圆心和半径
				* 如果该点在外接圆的右侧, 则该三角形为Delaunay三角形,保存到triangles,并在temp里去除掉,跳过
				* 如果该点在外接圆外(即也不是外接圆右侧), 则该三角形为不确定,跳过
				* 如果该点在外接圆内, 则该三角形不为Delaunay三角形,将三边保存至edgeBuffer,在temp中去除掉该三角形
			* 对edgeBuffer进行去重
			* 将edgeBuffer中的边与当前的点进行组合成若干三角形并保存至temp triangles中
		* 将triangles与trianglesTemp进行合并, 并除去与超级三角形有关的三角形





* 分形
	* Mandelbrot集
		\def{Mandelbrot集}
			所有能使Zn+1不发散的复数点C, 所构成的集合,即 Mandelbrot Set. (不发散,不一定收敛,有可能在几个不同点来回跳)
			$Zn+1 = Zn^2 + C$
			
		\Property
			* |Zn|>2不可能收敛, 即Mandelbrot Set在半径为2的圆内.


	* Julia集
		\def{Julia集}
			对于某复数值C,所有能使Zn+1不发散的Z0的集合,即 Julia Set. 类似于. Mandelbrot 曼德布洛特集
			$Zn+1 = Zn^2 + C$

	* Hilbert曲线
		\def{Hilbert 曲线} 一种自相似的分形曲线

	 	\bf{算法}
	 		四象限复制四分,翻转左下、右下, 使左下末同左上初、右下初同右上末,能够最短连接.
			翻转坐标, 使Hilbert曲线性质, 自相似地适用于左下、右下象限
			* 1,2阶曲线
				"┌┐"	   "┌┐┌┐"    "┌┐┌┐ "
				︱︱  =>	︱︱︱︱ =>   ︱└┘︱
						┌┐┌┐      └┐┌┘
						︱︱︱︱       -┘└-

		\Property
			* 边长: nth 2^n
			* 长度: nth 2^n - 1 / 2^n
			* 因为是四进制自相似, 所以曲线上位置 distance, 可以不断判断子象限,按二进制在位上叠加
			* 用途: 1D to 2D 的映射算法, 随 n 增加, 映射的位置趋于收敛

	* Perlin噪音

		\bf{算法}
			* 格点随机梯度矢量
			* (x,y)与格点距离,梯度点积
			* 插值

		

* Marching Cubes 三维等高面绘制


