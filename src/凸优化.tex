
* 凸优化
	* 仿射、凸、锥
		\bf{基础集合}
			* \def{仿射集}
					集合内任意两点所画直线仍在该集合内的集合. 
					$
						\forall \vec x_i \in C, \theta_i \in R, \sum \theta_i = 1
						then\ \sum \theta_i x_i \in C 
					$ 
			* \def{仿射包} 
					集合内所有点的仿射组合的集合构成一个仿射包. 
					$ 
						aff\ C =   \{\sum \theta_i x_i\ |\ x_i\in C,theta_i \in R, \sum \theta_i = 1  \}
					$
			* \def{凸集}
					集合内任意两点间的线段仍在该集合内的集合. 
					$
						\forall \vec x_i \in C, \theta_i \in [0,1], \sum \theta_i = 1 
						then\ \sum \theta_i x_i \in C 
					$
			* \def{凸包}
					集合内所有点的凸组合的集合构成一个凸包. 该凸包也是包含给定集合内所有点的最小凸集.
					$
						conv\ C = \{\sum \theta_i x_i\ |\ x_i\in C, \theta_i \in [0,1], \sum \theta_i = 1 \}
					$
			* \def{锥}
					以零点为起点的射线的集合.
					$\forall \vec x \in C, \theta > 0, then\ \theta \vec x \in C$
			* \def{凸锥}
					凸的锥.
					$
						\forall \vec x_1, \vec x_2 \in C, \theta_1,theta_2 \ge 0
						then\ \theta_1 \vec x_1 + \theta_2 \vec x_2 \in C
					$
		
	* 凸函数
		\def{凸函数}
			定义域是凸集, 且$(x_1, f(x_1))$和$(x_2, f(x_2))$间的线段始终在函数图像上方的函数.
			$
				f: \mathbb R^n \to \mathbb R
				dom\ f \text{是凸集}
				\forall x_1, x_2 \in dom\ f, \forall \theta \in [0,1]
				f(\theta x_1 + (1 - \theta) x_2) \le \theta f(x_1) + (1 - \theta)f(x_2)
			$
			
		\Property
			* 证明条件:
				* \bf{一阶条件}: $f(y) \ge f(x)+\nabla f(x)^{T}(y-x)$
				* \bf{二阶条件}: $\nabla^{2} f(x) \succeq 0$
		
		\def{拟凸函数}
			函数$f: \mathbf{R}^n \to \mathbf{R}$是拟凸函数$\Leftrightarrow$ 对$\forall \alpha \in R , \{x \in \mathbf{dom} f \mid f(x) \le \alpha\}$都是凸集.
		
	* 凸优化问题
		\def{优化问题}
			目标函数$f_0(x)$, 不等式约束$f_i(x)$, 等式约束$h_i(x)$.
			$
				\min \quad& f_0(x) 
				s.t. \quad& f_i(x) \le 0
						  & h_i(x) = 0
			$
			\bf{最优解}: $p^{\star}=\inf  \{f_{0}(x) \mid f_i(x) \le 0, h_i(x) = 0 \}$\

		\Example
			\def{可行性问题}
				如果目标函数恒等于零,则最优解是0或∞ (若可行集非空或空集).
				$
					\min \quad& x 
					s.t. \quad& f_i(x) \le 0
								& h_i(x) = 0
				$

		\def{凸优化问题}
			目标函数、不等式约束函数是凸函数, 且等式约束函数是仿射函数的优化问题.
			$
				\min \quad& x
				s.t. \quad& f_i(x) \le 0
						  & a_i^T x = b_i
			$
		\Property
			* 凸优化问题的局部最优解就是全局最优解.

		\Example
			* \def{线性规划}
					目标函数和约束函数都是放射的优化问题.
					$
						\min \quad& a^T x + b
						s.t. \quad& Gx \preceq 0
									& Ax = b
					$
					几何意义: 可行解集是多面体, 等位曲线是与向量$a^T$正交的超平面, 最优解是多面体中在$-a^T$方向最远的点.
			* \def{线性分式规划}
					$
						\min \quad& \frac{a^T x+ b}{c^T x + d}
						s.t. \quad& Gx \preceq 0
									& Ax = b
					$
					该问题可以等价转化为线性规划.
			* \def{二次规划}
					$
						\min \quad& \frac{1}{2} x^T P x + q^T x + r
						s.t. \quad& Gx \preceq 0
									& Ax = b
					$

				\Example
					最小二乘法$\min ||Ax + b||_2^2$

			* \def{二次约束二次规划}
					$
						\min \quad& \frac{1}{2} x^T P_0 x + q_0^T x + r_0
						s.t. \quad& \frac{1}{2} x^T P_i x + q_i^T x + r_i \preceq 0
									& Ax = b
					$

			* \def{二次锥规划}
					$
						\min \quad& f^T x
						s.t. \quad& ||A_i x + b_i|| \le c_i^T + d_i
									& Fx = g
					$

			* \def{几何规划}

			* \def{半正定规划}

			\df{关系} $二次锥规划 \supset \{二次规划 \supset \{ 线性规划 \} , 二次约束二次规划\}$
			
	* 凸优化算法
		\def{Lagrange函数} $L(x, \lambda, \nu) = f_0(x) + \sum_i \lambda_i f_i(x) + \sum_i \nu_i h_i(x)$
		\def{Lagrange对偶函数} $g(\lambda, \nu)=\inf _{x \in \mathcal{D}} L(x, \lambda, \nu)=\inf _{x \in \mathcal{D}} (f_{0}(x)+\sum_{i=1}^{m} \lambda_i f_i(x)+\sum_{i=1}^p \nu_i h_i(x) )$

		\Property
			* Lagrange对偶函数一定是凹函数.
			* 若$\lambda \succeq 0$, 则$g(\lambda, \nu) \le p^*$, 即给出原问题最优解的一个不平凡下界.

		\def{对偶问题}
			$
				\max \quad& g(\lambda, \nu)
				s.t. \quad& \lambda \succeq 0
			$
			最优解$d^* = max g(\lambda, \nu)$

		\Property
			* \bf{弱对偶} $p^* \ge d^*$ 恒成立
			* \bf{强对偶} $p^* = d^*$ 保证强对偶成立的条件为被称约束准则.
			
		\def{Slater 约束准则}
		\def{KKT 条件}


